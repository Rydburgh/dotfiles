%%%%%%%%%%%%%%%%%%%%%%%%%%%%%%%%%%%%%%%%%
% University/School Laboratory Report
% LaTeX Template
% Version 3.1 (25/3/14)
%
% This template has been downloaded from:
% http://www.LaTeXTemplates.com
%
% Original author:
% Linux and Unix Users Group at Virginia Tech Wiki
% (https://vtluug.org/wiki/Example_LaTeX_chem_lab_report)
%
% License:
% CC BY-NC-SA 3.0 (http://creativecommons.org/licenses/by-nc-sa/3.0/)
%
%%%%%%%%%%%%%%%%%%%%%%%%%%%%%%%%%%%%%%%%%

%----------------------------------------------------------------------------------------
%	PACKAGES AND DOCUMENT CONFIGURATIONS
%----------------------------------------------------------------------------------------

\documentclass{article}

\usepackage[version=3]{mhchem} % Package for chemical equation typesetting
\usepackage{siunitx} % Provides the \SI{}{} and \si{} command for typesetting SI units
\usepackage{graphicx} % Required for the inclusion of images
\usepackage{natbib} % Required to change bibliography style to APA
\usepackage{amsmath} % Required for some math elements

\setlength\parindent{0pt} % Removes all indentation from paragraphs

\renewcommand{\labelenumi}{\alph{enumi}.} % Make numbering in the enumerate environment by letter rather than number (e.g. section 6)

%\usepackage{times} % Uncomment to use the Times New Roman font

%----------------------------------------------------------------------------------------
%	DOCUMENT INFORMATION
%----------------------------------------------------------------------------------------

\title{Coulomb's Law Lab\\ PHY-222-AC01} % Title

\author{Jasper \textsc{Runco}} % Author name

\date{\today} % Date for the report

\begin{document}

\maketitle % Insert the title, author and date

\begin{center}
\begin{tabular}{l r}
\end{tabular}
\end{center}

% If you wish to include an abstract, uncomment the lines below
% \begin{abstract}
% Abstract text
% \end{abstract}

%----------------------------------------------------------------------------------------
%	SECTION 1
%----------------------------------------------------------------------------------------

\section{Theory}

Coulomb's law describes a mathematical expression for the interaction of electrically charged objects.
The electrical charge generates a force pair, $F_{E}$, with a magnitude that has been experimentally determined
to such a degree that it is accepted as a universal law.

\subsection{Definitions}
\label{definitions}
\begin{description}
	\item[Electrically charged object -]
		Matter with more or less electrons than protons is negatively or
		positively charged respectively.
	\item[Force pair -]
		The principal of Newton's third law in which every action force has a corresponding reaction force
		equal in magnitude and opposite in direction.
	\item[Coulomb's law -]
		\[F_{E} = k \frac{q_1 q_2}{r^2}\]
		\begin{description}
			\item[r -]
				The distance between the charged objects measured in meters.
			\item[$q_1 q_2$ -]
				The sum of the electrical charges of each object, measured in Coulombs.
			\item[k -]
				The coulomb constant, the experimentally determined proportionality constant with a value
				of $k = 9.0 \times 10^{9} \frac{Nm^2}{C^2}$

		\end{description}
\end{description}

%----------------------------------------------------------------------------------------
%	SECTION 2
%----------------------------------------------------------------------------------------
\section{Objectives}


% If you have more than one objective, uncomment the below:
\begin{description}
\item[First Objective] \hfill \\
	Experimentally confirm Coulomb's law.
\item[Second Objective] \hfill \\
	Study how distance and charge affect the electric force.
\item[Third Objective] \hfill \\
	Experimentally determine the value of the electric constant, k.
\end{description}


%----------------------------------------------------------------------------------------
%	SECTION 3
%----------------------------------------------------------------------------------------

\section{Experimental Data}

\subsection{Part One}%
\label{sub:part_one}


\begin{table}[htpb]
	\centering
	\caption{}
	\label{tab:label}
	\begin{tabular}{| c |  c | c |  c | }
		\hline
		$q_1= 2 \mu C$ & &  $q_2 = 4 \mu C$  &\\
		\hline
		\textbf{r(cm)} & \textbf{$r^2(m^2)$} & $\frac{1}{r^2}(\frac{1}{m^2})$ & $F_{E}(N)$ \\
		\hline
		10 & $1.0 \times 10^{-2}$ & $1\times 10^{2}$& $7.190$ \\
		\hline
		9 & $8.1 \times  10^{-3}$& $1.2 \times 10^{2}$ & $8.877$\\
		\hline
		8 & $6.4 \times 10^{-3}$& $1.6 \times 10^{2}$ & $11.234$\\
		\hline
		7 & $4.9 \times  10^{-3}$& $2.0 \times  10^{2}$ & $14.674$ \\
		\hline
		6 & $3.6 \times 10^{-3}$& $2.8 \times 10^{2}$& $19.972$\\
		\hline
		5 & $2.5 \times 10^{-3}$ & $4.0 \times  10^2$& $28.760$ \\
		\hline
		4 & $1.6 \times  10^{-3}$& $6.3 \times  10^2$ & $44.938$ \\
		\hline
		3 & $9.0 \times  10^{-4}$& $1.1 \times  10^{3}$ & $ 79.889$\\
		\hline
	\end{tabular}
\end{table}

\subsection{Part Two}%
\label{sub:part_two}

\begin{table}[htpb]
	\centering
	\caption{}
	\label{tab:label}
	\begin{tabular}{| c | c |}
		\hline
	 $q_1 = 5 \mu C$ & $ r = 6 cm$   \\
	 \hline
	 $q_2 (\mu C)$ & $F_{E}(N)$\\
	 \hline
	 10 & $124.827$  \\
	 \hline
	 9 &  $112.344$ \\
	 \hline
	 8 &  $99.862$ \\
	 \hline
	 7 &  $87.379$ \\
	 \hline
	 6 &   $74.896$\\
	 \hline
	 5 &  $62.414$ \\
	 \hline
	 4 &   $49.931$\\
	 \hline
	 3 &  $37.448$ \\
	 \hline
	\end{tabular}
\end{table}


%----------------------------------------------------------------------------------------
%	SECTION 4
%----------------------------------------------------------------------------------------

\section{Data Analysis}


%----------------------------------------------------------------------------------------
%	SECTION 5
%----------------------------------------------------------------------------------------

\section{Results and Conclusions}

The atomic weight of magnesium is concluded to be \SI{24}{\gram\per\mol}, as determined by the stoichiometry of its chemical combination with oxygen. This result is in agreement with the accepted value.

\begin{figure}[h]
\begin{center}
\includegraphics[width=0.65\textwidth]{placeholder} % Include the image placeholder.png
\caption{Figure caption.}
\end{center}
\end{figure}

%----------------------------------------------------------------------------------------
%	SECTION 6
%----------------------------------------------------------------------------------------

\section{Discussion of Experimental Uncertainty}

The accepted value (periodic table) is \SI{24.3}{\gram\per\mole} \cite{Smith:2012qr}. The percentage discrepancy between the accepted value and the result obtained here is 1.3\%. Because only a single measurement was made, it is not possible to calculate an estimated standard deviation.

The most obvious source of experimental uncertainty is the limited precision of the balance. Other potential sources of experimental uncertainty are: the reaction might not be complete; if not enough time was allowed for total oxidation, less than complete oxidation of the magnesium might have, in part, reacted with nitrogen in the air (incorrect reaction); the magnesium oxide might have absorbed water from the air, and thus weigh ``too much." Because the result obtained is close to the accepted value it is possible that some of these experimental uncertainties have fortuitously cancelled one another.

%----------------------------------------------------------------------------------------
%	SECTION 6
%----------------------------------------------------------------------------------------

\section{Answers to Definitions}

\begin{enumerate}
\begin{item}
The \emph{atomic weight of an element} is the relative weight of one of its atoms compared to C-12 with a weight of 12.0000000$\ldots$, hydrogen with a weight of 1.008, to oxygen with a weight of 16.00. Atomic weight is also the average weight of all the atoms of that element as they occur in nature.
\end{item}
\begin{item}
The \emph{units of atomic weight} are two-fold, with an identical numerical value. They are g/mole of atoms (or just g/mol) or amu/atom.
\end{item}
\begin{item}
\emph{Percentage discrepancy} between an accepted (literature) value and an experimental value is
\begin{equation*}
\frac{\mathrm{experimental\;result} - \mathrm{accepted\;result}}{\mathrm{accepted\;result}}
\end{equation*}
\end{item}
\end{enumerate}

%----------------------------------------------------------------------------------------
%	BIBLIOGRAPHY
%----------------------------------------------------------------------------------------

\bibliographystyle{apalike}

\bibliography{sample}

%----------------------------------------------------------------------------------------


\end{document}
