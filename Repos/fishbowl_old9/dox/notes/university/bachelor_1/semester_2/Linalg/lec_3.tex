%%%\documentclass[a4paper,11pt,twoside]{report}

%%%%| |_) | |_) |  _|   / _ \ | |\/| |  _ \| |   |  _|
%|  __/|  _ <| |___ / ___ \| |  | | |_) | |___| |___
%|_|   |_| \_\_____/_/   \_\_|  |_|____/|_____|_____|
% Jasper Runco
% last updated: 2020-05-20
%%%%%%%%%%%%%%%%%%%%%%%%%%%%%%%%%%%%%%%%%%%%%%%%%%%%%%%%%%%%%%%%%%%%%%%%%%%%%%%%%%%%%%%%%%
%%% PACKAGES
%%%%%%%%%%%%%%%%%%%%%%%%%%%%%%%%%%%%%%%%%%%%%%%%%%%%%%%%%%%%%%%%%%%%%%%%%%%%%%%%%%%%%%%%%%
\pdfminorversion=7						% to prevent errors when building pdf
% some basic packages
\usepackage{amsthm, amsmath}
\usepackage{url}						% to format hyperlink text
\usepackage{float}						% for custom figure/table environment
\usepackage{xifthen}					% to handle tests
\usepackage{booktabs}					% for table commands and optimisation
\usepackage{enumitem}					% to format enumerate, itemize, and description
\usepackage{textcomp}					% to support different glyphs
\usepackage{graphicx}					% to support \includegraphics
\usepackage[T1]{fontenc}				% for unicode encoding
\usepackage[utf8]{inputenc}				% for unicode input
\setlength{\headheight}{13.6pt}
\usepackage[top=1.5in,bottom=1in,right=1in,left=1in,headheight=45pt]{geometry}
\usepackage{fancyhdr}					% for adding different headers
\pagestyle{fancy}
% for list of equations
\usepackage{tocloft}					% for custom lists
\usepackage{ragged2e} 					% to undo \centering
\usepackage{hyperref} 					% to make references hyperlinks
\usepackage{glossaries}
% for figures
\usepackage{import}						% for file control
\usepackage{pdfpages}					% for pdf, graphics, and hypertext
\usepackage{transparent}				% for color stack transparency
\usepackage{xcolor}						% for arbitrary color mixing
%%%%%%%%%%%%%%%%%%%%%%%%%%%%%%%%%%%%%%%%%%%%%%%%%%%%%%%%%%%%%%%%%%%%%%%%%%%%%%%%%%%%
% Commands
%%%%%%%%%%%%%%%%%%%%%%%%%%%%%%%%%%%%%%%%%%%%%%%%%%%%%%%%%%%%%%%%%%%%%%%%%%%%%%%%%%%%
% to make a new figure
\newcommand{\incfig}[2][1]{%
	\def\svgwidth{#1\columnwidth}
	\import{./figures/}{#2.pdf_tex}
}
\pdfsuppresswarningpagegroup=1
% define list of equations
\newcommand{\listequationsname}{\Large{List of Equations}}
\newlistof{myequations}{equ}{\listequationsname}
\newcommand{\myequations}[1]{
	\addcontentsline{equ}{myequations}{\protect\numberline{\theequation}#1}
}
\setlength{\cftmyequationsnumwidth}{2.3em}
\setlength{\cftmyequationsindent}{1.5em}
% command to box, label, refference, and include
% noteworthy equations in list of equations
\newcommand{\noteworthy}[2]{
\begin{align} \label{#2} \ensuremath{\boxed{#1}} \end{align}
\myequations{#2} \centering \small \textit{#2} \normalsize \justify }
%%%%%%%%%%%%%%%%%%%%%%%%%%%%%%%%%%%%%%%%%%%%%%%%%%%%%%%%%%%%%%%%%%%%%%%%%%%%%%%%%%%%
% Theorems
%%%%%%%%%%%%%%%%%%%%%%%%%%%%%%%%%%%%%%%%%%%%%%%%%%%%%%%%%%%%%%%%%%%%%%%%%%%%%%%%%%%%
\newtheorem{definition}{Definition}
\newtheorem{theorem}{Theorem}
\newtheorem{lemma}{Lemma}
\newtheorem{corollary}{Corollary}


%%%\begin{document}

\chapter{Systems of Linear Equations and Matrices (Cont'd)}%
\label{cha:systems_of_linear_equations_and_matrices_cont_d_}


\LARGE\textsc{Date: 2020-08-21} \\ \\ \LARGE\textsc{Announcements:} \\
\small
\textbf{Assignment: 1.1 (5-10) 1.2 (1-22)}
Solve them using Gaus-Jordan Elimination, as was shown, not gausian elimination.


\paragraph \hrule \paragraph \\ \fancyhead[R]{Lesson 3} \fancyhead[L]{Week 1}
%  %  %  %  %  %  %  %  %  %  %  %  %  %  %  %  %  %  %  %  %  %  %  %  %  %  %  %
\begin{example}[solve]
	\begin{align*}
		2y + 3z &=  8 \\
		2x + 3y + z &= 5 \\
		x - y -2z &= -5 \\
	\end{align*}
\end{example}

\begin{definition}[Matrix]
	A rectangular array of numbers.
\end{definition}


coeficient matrix
\[	\begin{bmatrix}
	0 & 2 & 3 \\
	2 & 3 & 1 \\
	1 & -1 & -2 \\
\end{bmatrix}
\]

Constan matrix
\[	\begin{bmatrix}
	8 \\
	5 \\
	-5 \\
\end{bmatrix}
\]

Augmented matrix
\[	\begin{bmatrix}
	0 & 2 & 3 & 8 \\
	2 & 3 & 1 & 5 \\
	1 & -1 & -2 & -5 \\
\end{bmatrix}
\]

\section{Reduced Row Echelon Form}%
\label{sec:reduced_row_echelon_form}

\begin{enumerate}
	\item If a row does not consist entirely of zeros, then the first non zero entry is a 1. (Leading one).
	\item If any rows that are all zero, they appear at the bottom of the matrix.
	\item In any two successive rows that are not all zeros, the leading one in lower row is further to the right than
		the 1 in the higher row. (This qualifies as Row Eschelon Form)
	\item Each column that contains a leading 1 has zeros everywhere else in that column.
\end{enumerate}

\begin{solution}[1]

\end{solution}
$R_1\leftrightarrow R_3$

\[	\begin{bmatrix}
	1 & -1 & -2 & -5 \\
	2 & 3 & 1 & 5 \\
	0 & 2 & 3 & 8 \\
\end{bmatrix}
\]
$-2 R_1 + R_2 \to R_2$


\[	\begin{bmatrix}
	1 & -1 & -2 & -5 \\
	0 & 5 & 5 & 15 \\
	0 & 2 & 3 & 8 \\
\end{bmatrix}
\]

$\frac{1}{5}R_2 \to R_2$
\[	\begin{bmatrix}
	1 & -1 & -2 & -5 \\
	0 & 1 & 1 & 3 \\
	0 & 2 & 3 & 8 \\
\end{bmatrix}
\]
$-2 R_2 + R_3 \to R_3$

\[	\begin{bmatrix}
	1 & -1 & -2 & -5 \\
	0 & 1 & 1 & 3 \\
	0 & 0 & 1 & 2 \\
\end{bmatrix}
(Row Exchelon Form)
\]

$-R_3 + R_2 \to R_2$

\[	\begin{bmatrix}
	1 & -1 & -2 & -5 \\
	0 & 1 & 0 & 1 \\
	0 & 0 & 1 & 2 \\
\end{bmatrix}
\]

$2R_3 + R_1 \to R_1$

\[	\begin{bmatrix}
	1 & -1 & 0 & -1 \\
	0 & 1 & 0 & 1 \\
	0 & 0 & 1 & 2 \\
\end{bmatrix}
\]

$R_2 + R_1 \to R_1$

\[	\begin{bmatrix}
	1 & 0 & 0 &  0\\
	0 & 1 & 0 & 1 \\
	0 & 0 & 1 & 2 \\
\end{bmatrix}
(Reduced)
\]
\begin{align*}
	x&= 0 \\
	y &= 1 \\
	z &= 2 \\
\end{align*}
$\boxed{(0,1,2)}$

\begin{example}[Is this Row-eschelon or Reduced?]
	$\begin{bmatrix}
		1 & 1 & 3 & 4 & 9 \\
		0 & 0 & 1 & 7 & 7 \\
		0 & 0 & 0 & 0 & 0
	\end{bmatrix} $
\end{example}

\begin{solution}[]
	Answer: Just row eschelon, there must be a zero above and below every leading one.

	$-2R_2 + R_1 \to R_1$
	\[
		\begin{bmatrix}
			1 & 1 & 0 & -10 & -5 \\
			0 & 0 & 1 & 7 & 7 \\
			0 & 0 & 0 & 0 & 0
		\end{bmatrix}
		Reduced Row Eschelon Form
	\]

	$(R_1)$
	$x_1+x_2 - 10x_3 = -5$

	$(R_2)$
	$x_3+7x_4  = 7$

	$(R_1)$
	$x_1= -5-x_2 + 10x_3 $

	$(R_2)$
	$x_3  = 7-7x_4$

\end{solution}

	\begin{definition}[n-tuple]
		Ordered pair with n entries in it.

		E.g 4-tuple of solution above.
		\[(x_1,x_2,x_3,x_4) = (-5-x_2+10x_4, x_2, 7-7x_4, x_4)\]
	\end{definition}

	The tuple is always going to be parameterized, replacing $x_1$ with s's and $x_2$ with t's, etc.

		\[(-5-t+10s, t, 7-7s, s)\]





%  %  %  %  %  %  %  %  %  %  %  %  %  %  %  %  %  %  %  %  %  %  %  %  %  %  %  %
\newpage
%%%\end{document}


%LEAVE EMPTY ROW ABOVE THIS ONE
