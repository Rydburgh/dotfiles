%%%\documentclass[a4paper,11pt,twoside]{report}

%%%% ____  ____  _____    _    __  __ ____  _     _____
%|  _ \|  _ \| ____|  / \  |  \/  | __ )| |   | ____|
%| |_) | |_) |  _|   / _ \ | |\/| |  _ \| |   |  _|
%|  __/|  _ <| |___ / ___ \| |  | | |_) | |___| |___
%|_|   |_| \_\_____/_/   \_\_|  |_|____/|_____|_____|
%
% last updated: 2020-05-21
%%%%%%%%%%%%%%%%%%%%%%%%%%%%%%%%%%%%%%%%%%%%%%%%%%%%%%%%%%%%%%%%%%%%%%%%%%%%%%%%%%%%%%%%%%
%%% PACKAGES
%%%%%%%%%%%%%%%%%%%%%%%%%%%%%%%%%%%%%%%%%%%%%%%%%%%%%%%%%%%%%%%%%%%%%%%%%%%%%%%%%%%%%%%%%%


\pdfminorversion=7						% to prevent errors when building pdf

% some basic packages
\usepackage{amsmath, amsthm, amssymb}
\usepackage{mhchem}						% for chemical symbols
\usepackage{url}						% to format hyperlink text
\usepackage{float}						% for custom figure/table environment
\usepackage{xifthen}					% to handle tests
\usepackage{booktabs}					% for table commands and optimisation
\usepackage{enumitem}					% to format enumerate, itemize, and description
\usepackage{textcomp}					% to support different glyphs
\usepackage{graphicx}					% to support \includegraphics
\usepackage[T1]{fontenc}				% for unicode encoding
\usepackage[utf8]{inputenc}				% for unicode input
\setlength{\headheight}{13.6pt}
\usepackage[top=1.5in,bottom=1in,right=1in,left=1in,headheight=45pt]{geometry}
\usepackage{fancyhdr}					% for adding different headers
\pagestyle{fancy}
% for list of equations
\usepackage{tocloft}					% for custom lists
\usepackage{ragged2e} 					% to undo \centering
\usepackage{hyperref} 					% to make references hyperlinks
\usepackage{glossaries}

% for figures
\usepackage{import}						% for file control
\usepackage{pdfpages}					% for pdf, graphics, and hypertext
\usepackage{transparent}				% for color stack transparency
\usepackage{xcolor}						% for arbitrary color mixing



\author{Jasper Runco}
\date{2020 // Fall}

%%%%%%%%%%%%%%%%%%%%%%%%%%%%%%%%%%%%%%%%%%%%%%%%%%%%%%%%%%%%%%%%%%%%%%%%%%%%%%%%%%%%
% Commands
%%%%%%%%%%%%%%%%%%%%%%%%%%%%%%%%%%%%%%%%%%%%%%%%%%%%%%%%%%%%%%%%%%%%%%%%%%%%%%%%%%%%

% to make a new figure
\newcommand{\incfig}[2][scale=1]{%
	% \def\svgwidth{#1\columnwidth}
	\import{./figures/}{#2.pdf_tex}
}
\pdfsuppresswarningpagegroup=1

% define list of equations
\newcommand{\listequationsname}{\Large{List of Equations}}
\newlistof{myequations}{equ}{\listequationsname}
\newcommand{\myequations}[1]{
	\phantomsection
	\addcontentsline{equ}{myequations}{\protect\numberline{\theequation}#1}
}

\setlength{\cftmyequationsnumwidth}{2.3em}
\setlength{\cftmyequationsindent}{1.5em}

% command to box, label, reference, and include
% noteworthy equations in list of equations
\newcommand{\noteworthy}[2]{
\begin{align} \label{#2} \ensuremath{\boxed{#1}} \end{align}
\myequations{#2} \centering \textit{#2} \justify}

\newtheorem{definition}{Definition}
\newtheorem{theorem}{Theorem}
\newtheorem{lemma}{Lemma}
\newtheorem{corollary}{Corollary}
\newtheorem{example}{Example}
\newtheorem{solution}{Solution}
\newtheorem{constant}{Constant}
\newtheorem{note}{Note}


%%%\begin{document}

\chapter{Matrices and Matric Operations}%
\label{cha:matrices_and_matric_operations}


\LARGE\textsc{Date: 2020-08-26} \\ \\ \LARGE\textsc{Announcements:} \\
\small

\textbf{Assignment: Set 1.3 (1-6, 11-16, 23, 24) }


\paragraph \hrule \paragraph \\ \fancyhead[R]{Lesson 4} \fancyhead[L]{Week 2}
%  %  %  %  %  %  %  %  %  %  %  %  %  %  %  %  %  %  %  %  %  %  %  %  %  %  %  %

\section{Matrices}%
\label{sec:matrices}


\begin{definition}[Matrix]
	A rectangular array of numbers.
\end{definition}

\[A = \begin{bmatrix}  1 & 3 & 5 \\ 4 & 7 & 9 \end{bmatrix}
\]

\[B = \begin{bmatrix} 1 & 2 \\ 3 & 4 \\ 5 & 6  \end{bmatrix} \]


Size: \# Rows x \# Columns

A : 2x3

B : 3x2

\begin{example}[A General matrix]


A is $m \times n$ elements

m-rows

n-columns

\[A = \begin{bmatrix} a_{11} & a_{12} & a_{13} & \ldots & a_1n \\
a_{21} & a_{22} & a_{23} & \ldots & a_2n \\
\ldots & \ldots & \ldots & \ldots & \ldots\\
a_{m_1} & a_{m_2} & a_{m_3} & \ldots & a_{mn}\end{bmatrix} \]

\end{example}

\subsection{Square matrix}%
\label{sub:square_matrix}

\begin{definition}[Square Matrix]
	a matrix where \#rows = \#columns

	\[\begin{bmatrix} a_{11} & a_{22} & \ldots &a 1n \\
	a_{21} & a_{22} & \ldots & a_{2n} \\
\ldots & \ldots & \ldots & \ldots \\
a_{m_1} & a_{m_2} & \ldots & a_{mn}\end{bmatrix} \]

where $a_{11} \to a_{mn}$ is the main diagonal
\end{definition}

\section{Matrix operations}%
\label{sec:matrix_operations}


\begin{theorem}[Matrix equality]
	Two matrices are defined to be equal if they have the same size and their corresponding entries are equal.
\end{theorem}

\subsection{addition and subtraction}%
\label{sub:addition_and_subtraction}

\begin{theorem}[]
The sum of matrices A and B is written $A + B$ and it is the matrix obtained by adding corresponding
entries of two matrices of the same size.
\end{theorem}

\begin{example}[matrix addition]

	\[\begin{bmatrix} 1& 2 & 3&4 \\5&6&7&8 \end{bmatrix} + \begin{bmatrix} -3&4&-7&8 \\ -1& 0 & 5 & 9 \end{bmatrix}  \]
	\[= \begin{bmatrix}  -2 & 6 & -4 & 12 \\ 4 & 6 & 12 & 17 \end{bmatrix} \]
\end{example}

\subsubsection{Notation}%
\label{ssub:notation}

\begin{description}
	\item[A -] entire matrix
		\item[$a_{ij}$ -] individual entries
		\item[$(A+B)_{ij}$ -] notation of entry addition $(A)_{ij} + (B)_{ij}$
		\item[$(A-B)_{ij}$ -] $(A)_{ij - (B)_{ij}}$
\end{description}


\begin{example}[Matrix addition]
	\[\begin{bmatrix}  1&3&4 \end{bmatrix} + \begin{bmatrix} -5&-7&9 \end{bmatrix} = \begin{bmatrix} -4&-4&13 \end{bmatrix} \]
	\[\begin{bmatrix}  1&3&4 \end{bmatrix} - \begin{bmatrix} -5&-7&9 \end{bmatrix} = \begin{bmatrix} 6&10&-5 \end{bmatrix} \]
\end{example}


\subsection{Product of a scalar, c, and a Matrix, A}%
\label{sub:product_of_a_scalar_c_and_a_matrix_a}

The product of a scalar and a matrix, cA, is produced by multiplying each entry of A by c.

\[(cA)_{ij} = c(A)_{ij}\]

\begin{example}[]
	\[A = \begin{bmatrix}  1 & 2 \\ 3 & 4 \\ 5 & 6 \end{bmatrix} \]
	\[ B = \begin{bmatrix} -1 & 5 \\ -3 & 0 \\ 9 & 7 \end{bmatrix} \]
	\[3A-B\]
	\[   \begin{bmatrix} 3 & 6 \\ 9 & 12 \\ 15 & 18 \end{bmatrix} - \begin{bmatrix} -1 & 5 \\ -3 & 0 \\ 9 & 7 \end{bmatrix}  = \begin{bmatrix} 4 & 1 \\ 12 & 12 \\ 6 & 11 \end{bmatrix}     \]
\end{example}


\subsection{Product of matrices}%
\label{sub:product_of_matrices}

\begin{definition}[]
	The producto of two matrices A and B, written AB, is only defined when the number of columns
	of matrix A is equal to the number of rows of matrix B.
	\[A_{m \times r} B_{r\times n}\]

	The size of the product will be the rows of A by the columns of B.

	\[C_{m \times n}\]
\end{definition}

\begin{example}[]
	\[A_{3 \times 5} \:\text{and}\: B_{5\times 3}\]

	5 and 5: this can be done.

	3 and 3: the size of the result
\end{example}

\subsubsection{Getting the entries}%
\label{ssub:getting_the_entries}

To find the entries in Row i and Column j of AB, single out the ith row of A and the jth column of
B, multiply their corresponding entries and add the results.


\begin{example}[]
	\[\begin{bmatrix} 3 & 4 \\ 2 & 1 \\ 3 & 2 \end{bmatrix}_{3\times 2}  \begin{bmatrix} -1 & 1 & 2 & 3 \\ 1 & 5 & -2 & 2 \end{bmatrix}_{2\times 4}\]
\end{example}

\begin{solution}[]
	\[\begin{bmatrix} 3 \cdot -1 + 4 \cdot 1 & 3 \cdot 1 + 4 \cdot 5 & 3 \cdot 2 + 4 \cdot 2 & 9 + 8 \\ -2 + 1 & 2 + 5 & 4 + 2 & 6 + 2 \\
	-3 + 2 & 3 + 10 & 6 + 4 & 9 + 4\end{bmatrix} \]
	\[\begin{bmatrix} 1 & 23 & 14 & 17 \\ -1 & 7 & 6 & 8  \\ -1 & 13 & 10 & 13  \end{bmatrix} \]
\end{solution}

\begin{example}[]
	\[A = \begin{bmatrix} 2 & 1 &3 \\ 4 & -1 & 7 \\ 2 & 1 & 9 \end{bmatrix} \]
	\[X = \begin{bmatrix} x_1 \\ x_2 \\ x_3 \end{bmatrix} \]
	\[Ax\]
\end{example}
\begin{solution}[]
	\[Ax = \begin{bmatrix} 2x_1 + x_2 + 3x_3 \\ 4x_1 - x_2 +7 x_3 \\ 2x_1 + x_2 +9 x_3\end{bmatrix} \]
\end{solution}


\begin{definition}[Transpose of a matrix]
	$A^T$ is the matrix obtained when the rows and columns of A are interchanged.
\end{definition}

\begin{example}[]
	\[A = \begin{bmatrix}  1 & 2 & 5 \\ 9 & 3 & 1 \end{bmatrix} \]
\end{example}
\begin{solution}[]
	$A^T = \begin{bmatrix} 1 & 9 \\ 2 & 3 \\ 5 & 1 \end{bmatrix} $
\end{solution}






%  %  %  %  %  %  %  %  %  %  %  %  %  %  %  %  %  %  %  %  %  %  %  %  %  %  %  %
\newpage
%%%\end{document}


%LEAVE EMPTY ROW ABOVE THIS ONE
