%%%\documentclass[a4paper,11pt,twoside]{report}

%%%%| |_) | |_) |  _|   / _ \ | |\/| |  _ \| |   |  _|
%|  __/|  _ <| |___ / ___ \| |  | | |_) | |___| |___
%|_|   |_| \_\_____/_/   \_\_|  |_|____/|_____|_____|
% Jasper Runco
% last updated: 2020-05-20
%%%%%%%%%%%%%%%%%%%%%%%%%%%%%%%%%%%%%%%%%%%%%%%%%%%%%%%%%%%%%%%%%%%%%%%%%%%%%%%%%%%%%%%%%%
%%% PACKAGES
%%%%%%%%%%%%%%%%%%%%%%%%%%%%%%%%%%%%%%%%%%%%%%%%%%%%%%%%%%%%%%%%%%%%%%%%%%%%%%%%%%%%%%%%%%
\pdfminorversion=7						% to prevent errors when building pdf
% some basic packages
\usepackage{amsthm, amsmath}
\usepackage{url}						% to format hyperlink text
\usepackage{float}						% for custom figure/table environment
\usepackage{xifthen}					% to handle tests
\usepackage{booktabs}					% for table commands and optimisation
\usepackage{enumitem}					% to format enumerate, itemize, and description
\usepackage{textcomp}					% to support different glyphs
\usepackage{graphicx}					% to support \includegraphics
\usepackage[T1]{fontenc}				% for unicode encoding
\usepackage[utf8]{inputenc}				% for unicode input
\setlength{\headheight}{13.6pt}
\usepackage[top=1.5in,bottom=1in,right=1in,left=1in,headheight=45pt]{geometry}
\usepackage{fancyhdr}					% for adding different headers
\pagestyle{fancy}
% for list of equations
\usepackage{tocloft}					% for custom lists
\usepackage{ragged2e} 					% to undo \centering
\usepackage{hyperref} 					% to make references hyperlinks
\usepackage{glossaries}
% for figures
\usepackage{import}						% for file control
\usepackage{pdfpages}					% for pdf, graphics, and hypertext
\usepackage{transparent}				% for color stack transparency
\usepackage{xcolor}						% for arbitrary color mixing
%%%%%%%%%%%%%%%%%%%%%%%%%%%%%%%%%%%%%%%%%%%%%%%%%%%%%%%%%%%%%%%%%%%%%%%%%%%%%%%%%%%%
% Commands
%%%%%%%%%%%%%%%%%%%%%%%%%%%%%%%%%%%%%%%%%%%%%%%%%%%%%%%%%%%%%%%%%%%%%%%%%%%%%%%%%%%%
% to make a new figure
\newcommand{\incfig}[2][1]{%
	\def\svgwidth{#1\columnwidth}
	\import{./figures/}{#2.pdf_tex}
}
\pdfsuppresswarningpagegroup=1
% define list of equations
\newcommand{\listequationsname}{\Large{List of Equations}}
\newlistof{myequations}{equ}{\listequationsname}
\newcommand{\myequations}[1]{
	\addcontentsline{equ}{myequations}{\protect\numberline{\theequation}#1}
}
\setlength{\cftmyequationsnumwidth}{2.3em}
\setlength{\cftmyequationsindent}{1.5em}
% command to box, label, refference, and include
% noteworthy equations in list of equations
\newcommand{\noteworthy}[2]{
\begin{align} \label{#2} \ensuremath{\boxed{#1}} \end{align}
\myequations{#2} \centering \small \textit{#2} \normalsize \justify }
%%%%%%%%%%%%%%%%%%%%%%%%%%%%%%%%%%%%%%%%%%%%%%%%%%%%%%%%%%%%%%%%%%%%%%%%%%%%%%%%%%%%
% Theorems
%%%%%%%%%%%%%%%%%%%%%%%%%%%%%%%%%%%%%%%%%%%%%%%%%%%%%%%%%%%%%%%%%%%%%%%%%%%%%%%%%%%%
\newtheorem{definition}{Definition}
\newtheorem{theorem}{Theorem}
\newtheorem{lemma}{Lemma}
\newtheorem{corollary}{Corollary}


%%%\begin{document}

\chapter{Systems of Linear Equations and Matrices (Cont'd)}%
\label{cha:systems_of_linear_equations_and_matrices_cont_d_}


\LARGE\textsc{Date: 2020-08-19} \\ \\ \LARGE\textsc{Announcements:} \\
\small

20 minutes missed in lecture time will be posted as a video later today on blackboard.

\textbf{Assignment: Section 1.2 Numbers 5, 6, 7, 8
}

\paragraph \hrule \paragraph \\ \fancyhead[R]{Lesson 2} \fancyhead[L]{Week 1}
%  %  %  %  %  %  %  %  %  %  %  %  %  %  %  %  %  %  %  %  %  %  %  %  %  %  %  %
\section{Introduction}%
\label{sec:introduction}

Working towards solving systems of equations in a systematic way, vs. the algebra days when
you attack it at any angle with substitution and elimination. This systematic way is setting up
for the theoretical
\begin{example}[1]

\begin{enumerate}
	\item take a system of equation like this
		\[
			3x + 3y -2z = 13
		\]
		\[
			6x + 2y -5z = 13
			\] \[
			7x + 5y - 3z = 26
		\]
	\item Use the algebraic operations:
		\begin{enumerate}
			\item Multiply an equation by a nonzero constant
			\item  Interchange any equation
			\item Add a constant times one equation to another equation
		\end{enumerate}
		$-1  E2 + E3 \to E3$
		\begin{align*}
			-6x - 2y + 5z &= 13 \\
			7x + 5y - 3z &= 26\\
			\cline{1-2} \\
			x + 3y + 2z &= 13
		\end{align*}


		to get a variable in the first position with a coeficient of 1 (The equation you add to [ E3 ]
		is always the equation you replace.)
		\[
			6x + 2y - 5z = 13
			\] \[
			3x + 3y - 2z = 13
			\]	\[
			x + 3y + 2z = 13
		\]
		Interchange to get the equation with coeficient 1 on top, denoted like this:

		$E 1 \leftrightarrow E 3$
		\[
			x + 3y + 2z = 13
			\]\[
			6x + 2y - 5z = 13
			\]\[
			3x + 3y - 2z = 13
		\]
	\item Use the x to eliminate the x's from the other equations below.

		$-6  E 1 + E 2 \to E 3$

		$-3 E 1 + E 3 \to E 3$

		\begin{align*}
			x + 3y + 2z &= 13 \\
			-16y - 17z &= -65 \\
			-6y - 8z &= -26
		\end{align*}

	\item Get the next variable in column 2 with coeficient 1.

		$\frac{-1}{16}E 2 \to E 2$
		\begin{align*}
			x + 3y& + 2z = 13 \\
			y& + \frac{17}{16}z =  \frac{65}{16} \\
			-6y& - 8z =  -26 \\
		\end{align*}
	\item Use the y to elimintate the y's from below.
		\[
			6 E 1 + E 3 \to E 3
		\]
		\begin{align*}
			x + 3y + 2z& = 13 \\
			y + \frac{17}{16}z& = \frac{65}{16} \\
			-\frac{13}{8}z& =  -\frac{13}{8}\\
		\end{align*}

	\item Get the last variable in column 3 with coeficient 1. (Continue until diagonal of coeficient 1)

		$\frac{-8}{13}E 3 \to E 3$
		\begin{align*}
			x + 3y + 2z& = 13 \\
			y + \frac{17}{16}z& = \frac{65}{16} \\
			z& =  1\\
		\end{align*}
	\item Work to eliminate variables from the bottom up.

			$-\frac{17}{16}E 3 + E 2 \to E 2$

		$-2 E 3 + E 1 \to E 1$
		\begin{align*}
			x + 3y &= 11 \\
			y &= 3 \\
			z &= 1
		\end{align*}

			$-3 E 2 + E 1 \to E 1$
		\begin{align*}
			x &= 2 \\
			y &= 3 \\
			z &= 1
		\end{align*}

	\item Write solution as ordered tripple: $(2,3,1)$

\end{enumerate}

\end{example}
Because this system of equations could be written in this way, it has \textbf{only one} solution. There are
other cases where there are problems in solving the system this way, a situation where there is no soution
or infinite solutions.

\begin{example}[2]

	Solve:
	\begin{align*}
		x + y + z &= 1 \\
		-2x + y + z &= -2 \\
		3x + 6y + 6z &= 5
	\end{align*}

	$2E_1 + E_2 \to E_2$

	$-3E_1 + E_3 \to E_3$


	\begin{align*}
		x + y + z &= 1 \\
		 3y + 3z &= 0 \\
		3y + 3z &= 2
	\end{align*}

	$\frac{1}{3}E_2 \to E_2$

	\begin{align*}
		x + y + z &= 1 \\
		 y + z &= 0 \\
		3y + 3z &= 2
	\end{align*}

	$-3E_2+E_3 \to E_3$

	\begin{align*}
		x + y + z &= 1 \\
		 y + z &= 0 \\
		 0 &= 2  \leftarrow \:\text{false, no solution}\:
	\end{align*}

	You could make the determination of no solution eariler, as soon as you catch the false statement.
\end{example}

\begin{example}[3]
	\begin{align*}
		2x -y _ z &= -1 \\
		x +3y -2z &= 2 \\
		-5x + 6y -5z &= 5 \\
	\end{align*}

	$E_1 \leftrightarrow E_2$

	\begin{align*}
		x +3y -2z &= 2 \\
		2x -y + z &= -1 \\
		-5x + 6y -5z &= 5 \\
	\end{align*}


$-2E_1 + E_2 \to E_2$

$5E_1+E_3 \to E_3$

	\begin{align*}
		x +3y -2z &= 2 \\
		-7y + 5z &= -5 \\
		21y -5z &= 15 \\
	\end{align*}

	$\frac{-1}{7}E_2 \to E_2$

	\begin{align*}
		x +3y -2z &= 2 \\
		y - \frac{5}{7}z &= \frac{5}{7} \\
		21y -5z &= 15 \\
	\end{align*}

	$- 21 E_2 + E_3 \to E_3$

	\begin{align*}
		x +3y -2z &= 2 \\
		y - \frac{5}{7}z &= \frac{5}{7} \\
		0 &= 0 \\
	\end{align*}

Where the last equation is a \textbf{true} statement, but there is no coeficient of 1,
is a situation where there are \textbf{infinite solutions}. You must continue to elimintate up with
what you got.

$ -3E_2+ E_1 \to E_1$

	\begin{align*}
		x + \frac{1}{7}z &= - \frac{1}{7} \\
		y - \frac{5}{7}z &= \frac{5}{7} \\
		0 &= 0 \\
	\end{align*}

	This is now in as few variables as possible, and can give us a final solution

	\begin{align*}
		x + \frac{1}{7}z = - \frac{1}{7} & \to \boxed{x = - \frac{1}{7} z - \frac{1}{7}}\\
		y - \frac{5}{7}z =  \frac{5}{7} & \to \boxed{y =  \frac{5}{7} z + \frac{5}{7}}\\
		0 &= 0
	\end{align*}

	$ \boxed{(x,y,z) = \left( - \frac{1}{7}z - \frac{1}{7}, \frac{5}{7}z + \frac{5}{7}, z \right)} $
\end{example}

Thinking about this solution in terms of vectors, $z(-\frac{1}{7}, \frac{5}{7}, 1) + (-\frac{1}{7},
\frac{5}{5}, 0)$, this describes a line.

\begin{example}[4]

	\begin{align*}
		x + 3y + 4z &= 1 \\
		2x + 6y + 8z &= 2 \\
		3x + 9y + 12z &= 3 \\
	\end{align*}

$ -2 E_1 + E_2 \to E_2$

$-E_1+E_3\to E_3$

	\begin{align*}
		x + 3y + 4z &= 1 \\
		0 &= 0 \\
		0 &= 0 \\
	\end{align*}

	both equations with no variables are ture equations, therefore \textbf{infinite solutions}, and
	we have one equations to solve for x.


	$x + 3y + 4z = 1 \to x = -3y -4z +1$

	$\boxed{\left( x,y,z \right) = \left( -3y + 4z +1, y, z \right) }$


\end{example}

	You can look at this as the sum of three vectors,

	$(-3y, y, 0) + (-4z, 0, z) + (1, 0, 0)$
	or $y(-3, 1, 0) + z(-4, 0, 1 ) + (1, 0, 0)$

	This is just a geometric interpretation, whatever it means, but it is not a line


%  %  %  %  %  %  %  %  %  %  %  %  %  %  %  %  %  %  %  %  %  %  %  %  %  %  %  %
\newpage
%%%\end{document}


%LEAVE EMPTY ROW ABOVE THIS ONE
