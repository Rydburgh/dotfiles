%%%
\documentclass[a4paper,11pt,twoside]{report}
%%%
%| |_) | |_) |  _|   / _ \ | |\/| |  _ \| |   |  _|
%|  __/|  _ <| |___ / ___ \| |  | | |_) | |___| |___
%|_|   |_| \_\_____/_/   \_\_|  |_|____/|_____|_____|
% Jasper Runco
% last updated: 2020-05-20
%%%%%%%%%%%%%%%%%%%%%%%%%%%%%%%%%%%%%%%%%%%%%%%%%%%%%%%%%%%%%%%%%%%%%%%%%%%%%%%%%%%%%%%%%%
%%% PACKAGES
%%%%%%%%%%%%%%%%%%%%%%%%%%%%%%%%%%%%%%%%%%%%%%%%%%%%%%%%%%%%%%%%%%%%%%%%%%%%%%%%%%%%%%%%%%
\pdfminorversion=7						% to prevent errors when building pdf
% some basic packages
\usepackage{amsthm, amsmath}
\usepackage{url}						% to format hyperlink text
\usepackage{float}						% for custom figure/table environment
\usepackage{xifthen}					% to handle tests
\usepackage{booktabs}					% for table commands and optimisation
\usepackage{enumitem}					% to format enumerate, itemize, and description
\usepackage{textcomp}					% to support different glyphs
\usepackage{graphicx}					% to support \includegraphics
\usepackage[T1]{fontenc}				% for unicode encoding
\usepackage[utf8]{inputenc}				% for unicode input
\setlength{\headheight}{13.6pt}
\usepackage[top=1.5in,bottom=1in,right=1in,left=1in,headheight=45pt]{geometry}
\usepackage{fancyhdr}					% for adding different headers
\pagestyle{fancy}
% for list of equations
\usepackage{tocloft}					% for custom lists
\usepackage{ragged2e} 					% to undo \centering
\usepackage{hyperref} 					% to make references hyperlinks
\usepackage{glossaries}
% for figures
\usepackage{import}						% for file control
\usepackage{pdfpages}					% for pdf, graphics, and hypertext
\usepackage{transparent}				% for color stack transparency
\usepackage{xcolor}						% for arbitrary color mixing
%%%%%%%%%%%%%%%%%%%%%%%%%%%%%%%%%%%%%%%%%%%%%%%%%%%%%%%%%%%%%%%%%%%%%%%%%%%%%%%%%%%%
% Commands
%%%%%%%%%%%%%%%%%%%%%%%%%%%%%%%%%%%%%%%%%%%%%%%%%%%%%%%%%%%%%%%%%%%%%%%%%%%%%%%%%%%%
% to make a new figure
\newcommand{\incfig}[2][1]{%
	\def\svgwidth{#1\columnwidth}
	\import{./figures/}{#2.pdf_tex}
}
\pdfsuppresswarningpagegroup=1
% define list of equations
\newcommand{\listequationsname}{\Large{List of Equations}}
\newlistof{myequations}{equ}{\listequationsname}
\newcommand{\myequations}[1]{
	\addcontentsline{equ}{myequations}{\protect\numberline{\theequation}#1}
}
\setlength{\cftmyequationsnumwidth}{2.3em}
\setlength{\cftmyequationsindent}{1.5em}
% command to box, label, refference, and include
% noteworthy equations in list of equations
\newcommand{\noteworthy}[2]{
\begin{align} \label{#2} \ensuremath{\boxed{#1}} \end{align}
\myequations{#2} \centering \small \textit{#2} \normalsize \justify }
%%%%%%%%%%%%%%%%%%%%%%%%%%%%%%%%%%%%%%%%%%%%%%%%%%%%%%%%%%%%%%%%%%%%%%%%%%%%%%%%%%%%
% Theorems
%%%%%%%%%%%%%%%%%%%%%%%%%%%%%%%%%%%%%%%%%%%%%%%%%%%%%%%%%%%%%%%%%%%%%%%%%%%%%%%%%%%%
\newtheorem{definition}{Definition}
\newtheorem{theorem}{Theorem}
\newtheorem{lemma}{Lemma}
\newtheorem{corollary}{Corollary}

%%%
\begin{document}
\chapter{Two-dimensional projectile motion}%
\label{cha:two_dimensional_projectile_motion}


\LARGE\textsc{Date: 2020-05-25} \\ \LARGE\textsc{Announcements:}

\paragraph \hrule \paragraph \\ \fancyhead[R]{Lesson 3} \fancyhead[L]{Week 2}
%  %  %  %  %  %  %  %  %  %  %  %  %  %  %  %  %  %  %  %  %  %  %  %  %  %  %  %
\section{Horizontally launched projectile}%
\label{sec:horizontally_launched_projectile}
\begin{figure}[ht]
    \centering
    \incfig{horizontally-launched-projectile}
    \caption{Horizontally launched projectile}
    \label{fig:horizontally-launched-projectile}
\end{figure}
For example, a cliff diver runs off a cliff at $5 \frac{m}{s}$ from a height of $30 m$. How far
do they travel in the x-direction before landing? The key is that the vertical and horizontal
components of the velocity are independent. Their horizontal velocity is always  $5 \frac{m}{s}$, but
the vertical starts at 0 and accelerates due to gravity. This is how we write it up: \[
\begin{array}{l | l}
	x & y \\
	v_{0x} = 5 \frac{m}{s} & \Delta y = -30 m \\
	dx = ? & a_{y} = 09.8 \frac{m}{s^2} \\
		   & V_{oy} = 0 \frac{m}{s}
\end{array}
.\]
We need to find the time on the y side of the table.\[
	\Delta y = v_{oy}t + \frac{1}{2} a_{y} t^2
.\]\[
	-30 m = 0 + \frac{1}{2} (-9.8 \frac{m}{s^2}) t^2
.\] \[
	t = \sqrt{\frac{(-30m)(2)}{-9.8 \frac{m}{s^2}}} = 2.47 s
.\]
We could take this, the time to displace $30 m$ vertically, and use it in the x direction: \[
\Delta x = V_{ox}t + \frac{1}{2} a_{x}t^2
\] \[
5 \frac{m}{s} dx = (5 \frac{m}{s})(2.47 s)
\] \[
dx = 12.4 m
.\]

\section{Visualizing vectors in 2 dimensions}%
\label{sec:visualizing_vectors_in_2_dimensions}

In classical mechanics, you deal with up to 3 dimensions. How do vectors add?

\begin{figure}[ht]
    \centering
    \incfig{single-key-test}
    \caption{single key test}
    \label{fig:single-key-test}
\end{figure}

%  %  %  %  %  %  %  %  %  %  %  %  %  %  %  %  %  %  %  %  %  %  %  %  %  %  %  %
\newpage
%%%
\end{document}

%LEAVE EMPTY ROW ABOVE THIS ONE
