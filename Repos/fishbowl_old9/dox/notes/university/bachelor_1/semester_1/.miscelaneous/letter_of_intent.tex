%%%%%%%%%%%%%%%%%%%%%%%%%%%%%%%%%%%%%%%%%
% Plain Cover Letter
% LaTeX Template
% Version 1.0 (28/5/13)
%
% This template has been downloaded from:
% http://www.LaTeXTemplates.com
%
% Original author:
% Rensselaer Polytechnic Institute
% http://www.rpi.edu/dept/arc/training/latex/resumes/
%
% License:
% CC BY-NC-SA 3.0 (http://creativecommons.org/licenses/by-nc-sa/3.0/)
%
%%%%%%%%%%%%%%%%%%%%%%%%%%%%%%%%%%%%%%%%%

%----------------------------------------------------------------------------------------
%	PACKAGES AND OTHER DOCUMENT CONFIGURATIONS
%----------------------------------------------------------------------------------------

\documentclass[11pt]{letter} % Default font size of the document, change to 10pt to fit more text

%\usepackage{newcent} % Default font is the New Century Schoolbook PostScript font
\usepackage{helvet} % Uncomment this (while commenting the above line) to use the Helvetica font

% Margins
\topmargin=-1.6in % Moves the top of the document 1 inch above the default
\textheight=11.5in % Total height of the text on the page before text goes on to the next page, this can be increased in a longer letter
\oddsidemargin=-10pt % Position of the left margin, can be negative or positive if you want more or less room
\textwidth=6.5in % Total width of the text, increase this if the left margin was decreased and vice-versa

\let\raggedleft\raggedright % Pushes the date (at the top) to the left, comment this line to have the date on the right

\begin{document}

%----------------------------------------------------------------------------------------
%	ADDRESSEE SECTION
%----------------------------------------------------------------------------------------
\begin{letter}{Karen C. Gerardi\\
Senior Assistant Director for Transfer Students \& Operations \\
Washington \& Jefferson College\\
60 S Lincoln St.\\
Washington, PA 15301}

%----------------------------------------------------------------------------------------
%	YOUR NAME & ADDRESS SECTION
%----------------------------------------------------------------------------------------

\begin{center}
\large\bf Jasper Runco \\ % Your name
%\vspace{20pt} \hrule height 1pt % If you would like a horizontal line separating the name from the address, uncomment the line to the left of this text
1204 Loraine St. \\ Pittsburgh, PA 15212 \\ (412) 588-5134 % Your address and phone number
\end{center}
%\vfill

\signature{Jasper Runco} % Your name for the signature at the bottom

%----------------------------------------------------------------------------------------
%	LETTER CONTENT SECTION
%----------------------------------------------------------------------------------------

\opening{Dear Mrs. Gerardi:}

I am interested in completing a baccalaureate at Washington \& Jefferson College. I hope to declare a physics major in the engineering dual-degree program. I first learned of the W\&J dual-degree program from Professor Michael McCracken and his students while attending a physics outreach conference in January. After considerable debate over when and where to transfer credits obtained at community college, I am convinced that this fall term at W\&J is the best option. I had not planned to obtain an associate degree at CCAC. I sought an academic holding pattern from which to develop an informed plan of action -- a plan that would best serve my research interests and goal for a career in the emerging nuclear industry.

My first time attending college, I lacked a plan. I enrolled at Pitt after attending a flashy internship in high school and never being challenged. I adopted the mentality that I could focus on anything but classes, refuse help, and dash away with a degree. Being academically dismissed was my first major failure and for some time I denied responsibility. This resulted in a period of life marked by disillusionment. However, I would not wish to change the past. I have learned of its power to benefit others. As my master chief once said, this was a defining moment, but I refuse to let it define me.

In 2017 I was eager for a change and made two successive discoveries. First, I learned that a handful of North American companies as well as Asian and European governments were pursuing alternative nuclear energy platforms that eliminate radioactive stockpiles rather than produce them, the premise for all previous generations of reactors. This is just one of several design features that motivate my interest in seeing this technology realized. Second, I discovered that the Navy would pay me to learn about the principals of reactor operation. After a clearance investigation and basic training I entered the Navy's nuclear training pipeline. It was there I found a joy for the underlying nature of what we observe, the particle interactions that propel our carriers and submarines through the sea.

This too was a defining moment. It determined how I should order my priorities. First and foremost, I wish to orient and conduct myself as a person of science. Secondly, I want to develop an engineer's intuition and toolkit. Only after that can I be useful in the energy sector.

I have spent this time of social isolation preparing for my studies. I have configured an assortment of open-source software to make lectures, research, writing, and student life efficient and fun. I have been using this software to review material and tutor a future naval officer in math. Acquiring financial support has allowed me to mostly forgo working while taking classes. I am hoping that by improving my GPA this year, scholarships will be more available.
\closing{Sincerely yours,}



%----------------------------------------------------------------------------------------

\end{letter}

\end{document}
