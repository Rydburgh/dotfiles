%%%\documentclass[a4paper,11pt,twoside]{report}

%%%%| |_) | |_) |  _|   / _ \ | |\/| |  _ \| |   |  _|
%|  __/|  _ <| |___ / ___ \| |  | | |_) | |___| |___
%|_|   |_| \_\_____/_/   \_\_|  |_|____/|_____|_____|
% Jasper Runco
% last updated: 2020-05-20
%%%%%%%%%%%%%%%%%%%%%%%%%%%%%%%%%%%%%%%%%%%%%%%%%%%%%%%%%%%%%%%%%%%%%%%%%%%%%%%%%%%%%%%%%%
%%% PACKAGES
%%%%%%%%%%%%%%%%%%%%%%%%%%%%%%%%%%%%%%%%%%%%%%%%%%%%%%%%%%%%%%%%%%%%%%%%%%%%%%%%%%%%%%%%%%
\pdfminorversion=7						% to prevent errors when building pdf
% some basic packages
\usepackage{amsthm, amsmath}
\usepackage{url}						% to format hyperlink text
\usepackage{float}						% for custom figure/table environment
\usepackage{xifthen}					% to handle tests
\usepackage{booktabs}					% for table commands and optimisation
\usepackage{enumitem}					% to format enumerate, itemize, and description
\usepackage{textcomp}					% to support different glyphs
\usepackage{graphicx}					% to support \includegraphics
\usepackage[T1]{fontenc}				% for unicode encoding
\usepackage[utf8]{inputenc}				% for unicode input
\setlength{\headheight}{13.6pt}
\usepackage[top=1.5in,bottom=1in,right=1in,left=1in,headheight=45pt]{geometry}
\usepackage{fancyhdr}					% for adding different headers
\pagestyle{fancy}
% for list of equations
\usepackage{tocloft}					% for custom lists
\usepackage{ragged2e} 					% to undo \centering
\usepackage{hyperref} 					% to make references hyperlinks
\usepackage{glossaries}
% for figures
\usepackage{import}						% for file control
\usepackage{pdfpages}					% for pdf, graphics, and hypertext
\usepackage{transparent}				% for color stack transparency
\usepackage{xcolor}						% for arbitrary color mixing
%%%%%%%%%%%%%%%%%%%%%%%%%%%%%%%%%%%%%%%%%%%%%%%%%%%%%%%%%%%%%%%%%%%%%%%%%%%%%%%%%%%%
% Commands
%%%%%%%%%%%%%%%%%%%%%%%%%%%%%%%%%%%%%%%%%%%%%%%%%%%%%%%%%%%%%%%%%%%%%%%%%%%%%%%%%%%%
% to make a new figure
\newcommand{\incfig}[2][1]{%
	\def\svgwidth{#1\columnwidth}
	\import{./figures/}{#2.pdf_tex}
}
\pdfsuppresswarningpagegroup=1
% define list of equations
\newcommand{\listequationsname}{\Large{List of Equations}}
\newlistof{myequations}{equ}{\listequationsname}
\newcommand{\myequations}[1]{
	\addcontentsline{equ}{myequations}{\protect\numberline{\theequation}#1}
}
\setlength{\cftmyequationsnumwidth}{2.3em}
\setlength{\cftmyequationsindent}{1.5em}
% command to box, label, refference, and include
% noteworthy equations in list of equations
\newcommand{\noteworthy}[2]{
\begin{align} \label{#2} \ensuremath{\boxed{#1}} \end{align}
\myequations{#2} \centering \small \textit{#2} \normalsize \justify }
%%%%%%%%%%%%%%%%%%%%%%%%%%%%%%%%%%%%%%%%%%%%%%%%%%%%%%%%%%%%%%%%%%%%%%%%%%%%%%%%%%%%
% Theorems
%%%%%%%%%%%%%%%%%%%%%%%%%%%%%%%%%%%%%%%%%%%%%%%%%%%%%%%%%%%%%%%%%%%%%%%%%%%%%%%%%%%%
\newtheorem{definition}{Definition}
\newtheorem{theorem}{Theorem}
\newtheorem{lemma}{Lemma}
\newtheorem{corollary}{Corollary}


%%%\begin{document}

\chapter{Atomic Structure}%
\label{cha:atomic_structure}


\LARGE\textsc{Date: 2020-06-02} \\ \LARGE\textsc{Announcements:}



\paragraph \hrule \paragraph \\ \fancyhead[R]{Lesson 1} \fancyhead[L]{Week 2}
%  %  %  %  %  %  %  %  %  %  %  %  %  %  %  %  %  %  %  %  %  %  %  %  %  %  %  %
\section{History: discovery of the electron}%
\label{sec:history_discovery_of_the_electron}

J. J. Thompson was interested in cathonde rays. He had a hydrogen gas cylinder and put a charge to it,
and got these rays. He was curious to see what would happen to the rays as they passed a charged
deflection plate, and saw the cathode ray being deflected towards the positive plate, which told him
the cathode rays were made of \textbf{negatively} charged particles.

$\Delta x_{(-)} \propto \frac{e_{(-)}}{m_{(-)}} \frac{charge}{mass} $.

This deflection was directly
proportional to charge and inversely to mass. When he applied a greater charge, he saw a small
deflection towards the negatively charged particle, which told him there were also \textbf{positively}
charged particles. \[
	\Delta x_{(+)} \propto \frac{e_{(-)}}{m_{(-)}} \frac{charge}{mass}
.\]
The relative deflection told him something about the masses of these particles \[
	\frac{\left| \Delta x_{(-)} \right| }{\left| \Delta x_{(+)} \right| } =
	\frac{\left| \frac{e_{(-)}}{m_{(-)}} \right| }{\left| \frac{e_{(+)}}{m_{(+)}} \right| } =
	\frac{m_{(+)}}{m_{(-)}}
.\]
This negatively charged particle was later named the electron and its mass was very small
$m = 9.11 \times 10^{-31} kg$.
\section{Rutheford Discovery of the Nucleus (1911)}%
\label{sec:rutheford_discovery_of_the_nucleus_1911_}

Rutheford and his students did an experiment with a sample Curie sent them. They had a detector
that counted the number of $\alpha$ particles comming off the radioactive sample. They counted
and then put thin gold foil in the path. When they counted, they seemed to be the same.
They moved the detector behind the foil to detect a small number of backscatter.\[
P = \frac{\:\text{count rate backscattered}\:}{\:\text{count rate of incident particles}\:} =
\frac{20}{132,000} = 2 \times 10^{-4}
.\]
\subsection{Interpretation:}%
\label{sub:interpretation_}

The Au atoms are mostly \textbf{empty.}

The majority of each atom's mass is concntrated in a very small volume, we call now the \textbf{nucleus}

\noteworthy{
Z = \:\text{atomic number}\:
}{Z}
\noteworthy{
e = \:\text{the absolute value of an electron's charge}\:
}{e}
\begin{definition}[]
	Charge of the electron in an atom $= -Ze$
\end{definition}
\begin{definition}[]
	Charge of the nucleus in an atom $= + Ze$
\end{definition}

Rutheford used backscattering to measure the diameter of the nucleus
\begin{definition}[Diameter of the nucleus]
	$10^{-14} m$
\end{definition}

\section{(Failure of) the classical description of the atom}%
\label{sec:failure_of_the_classical_description_of_the_atom}

\begin{definition}[Columb's Force law]
	Columb Force $F(r) = \frac{Q_{1}Q_{2}}{4\pi \epsilon_{0}r^2}$
\end{definition}
\noteworthy{
	Q_{x} = \:\text{charge on particle x}\:
}{}
\noteworthy{
r  = \:\text{distance between two charges}\:
}{}
\begin{definition}[Permittivity constant of a vacuum]
	$\epsilon_{0}= 8.854 \times 10^{-12}C^2J^{-1}m^{-1}$
\end{definition}

The electron should plummet into the nucleus, the equations don't work at small scales.



%  %  %  %  %  %  %  %  %  %  %  %  %  %  %  %  %  %  %  %  %  %  %  %  %  %  %  %
\newpage
%%%\end{document}


%LEAVE EMPTY ROW ABOVE THIS ONE
