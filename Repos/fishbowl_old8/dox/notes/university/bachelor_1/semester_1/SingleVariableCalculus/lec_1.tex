%%%
\documentclass[a4paper,11pt,twoside]{report}
%%%
% ____  ____  _____    _    __  __ ____  _     _____
%|  _ \|  _ \| ____|  / \  |  \/  | __ )| |   | ____|
%| |_) | |_) |  _|   / _ \ | |\/| |  _ \| |   |  _|
%|  __/|  _ <| |___ / ___ \| |  | | |_) | |___| |___
%|_|   |_| \_\_____/_/   \_\_|  |_|____/|_____|_____|
%
% last updated: 2020-05-21
%%%%%%%%%%%%%%%%%%%%%%%%%%%%%%%%%%%%%%%%%%%%%%%%%%%%%%%%%%%%%%%%%%%%%%%%%%%%%%%%%%%%%%%%%%
%%% PACKAGES
%%%%%%%%%%%%%%%%%%%%%%%%%%%%%%%%%%%%%%%%%%%%%%%%%%%%%%%%%%%%%%%%%%%%%%%%%%%%%%%%%%%%%%%%%%


\pdfminorversion=7						% to prevent errors when building pdf

% some basic packages
\usepackage{amsmath, amsthm, amssymb}
\usepackage{mhchem}						% for chemical symbols
\usepackage{url}						% to format hyperlink text
\usepackage{float}						% for custom figure/table environment
\usepackage{xifthen}					% to handle tests
\usepackage{booktabs}					% for table commands and optimisation
\usepackage{enumitem}					% to format enumerate, itemize, and description
\usepackage{textcomp}					% to support different glyphs
\usepackage{graphicx}					% to support \includegraphics
\usepackage[T1]{fontenc}				% for unicode encoding
\usepackage[utf8]{inputenc}				% for unicode input
\setlength{\headheight}{13.6pt}
\usepackage[top=1.5in,bottom=1in,right=1in,left=1in,headheight=45pt]{geometry}
\usepackage{fancyhdr}					% for adding different headers
\pagestyle{fancy}
% for list of equations
\usepackage{tocloft}					% for custom lists
\usepackage{ragged2e} 					% to undo \centering
\usepackage{hyperref} 					% to make references hyperlinks
\usepackage{glossaries}

% for figures
\usepackage{import}						% for file control
\usepackage{pdfpages}					% for pdf, graphics, and hypertext
\usepackage{transparent}				% for color stack transparency
\usepackage{xcolor}						% for arbitrary color mixing



\author{Jasper Runco}
\date{2020 // Fall}

%%%%%%%%%%%%%%%%%%%%%%%%%%%%%%%%%%%%%%%%%%%%%%%%%%%%%%%%%%%%%%%%%%%%%%%%%%%%%%%%%%%%
% Commands
%%%%%%%%%%%%%%%%%%%%%%%%%%%%%%%%%%%%%%%%%%%%%%%%%%%%%%%%%%%%%%%%%%%%%%%%%%%%%%%%%%%%

% to make a new figure
\newcommand{\incfig}[2][scale=1]{%
	% \def\svgwidth{#1\columnwidth}
	\import{./figures/}{#2.pdf_tex}
}
\pdfsuppresswarningpagegroup=1

% define list of equations
\newcommand{\listequationsname}{\Large{List of Equations}}
\newlistof{myequations}{equ}{\listequationsname}
\newcommand{\myequations}[1]{
	\phantomsection
	\addcontentsline{equ}{myequations}{\protect\numberline{\theequation}#1}
}

\setlength{\cftmyequationsnumwidth}{2.3em}
\setlength{\cftmyequationsindent}{1.5em}

% command to box, label, reference, and include
% noteworthy equations in list of equations
\newcommand{\noteworthy}[2]{
\begin{align} \label{#2} \ensuremath{\boxed{#1}} \end{align}
\myequations{#2} \centering \textit{#2} \justify}

\newtheorem{definition}{Definition}
\newtheorem{theorem}{Theorem}
\newtheorem{lemma}{Lemma}
\newtheorem{corollary}{Corollary}
\newtheorem{example}{Example}
\newtheorem{solution}{Solution}
\newtheorem{constant}{Constant}
\newtheorem{note}{Note}

%%%
\begin{document}
\chapter{Unit 1: Differentiation}%
\label{cha:unit_1_differentiation}


\LARGE\textsc{Date: 2020-06-01} \\ \LARGE\textsc{Announcements:}



\paragraph \hrule \paragraph \\ \fancyhead[R]{Lesson 1} \fancyhead[L]{Week 1}
%  %  %  %  %  %  %  %  %  %  %  %  %  %  %  %  %  %  %  %  %  %  %  %  %  %  %  %
\section{A. What is a derivative?}%
\label{sec:a_what_is_a_derivative_}

\begin{itemize}
	\item Geometric interpretation
	\item Physical interpretation
	\item Importance of derivatives to all measurements
		\begin{itemize}
			\item Science
			\item Enineering
			\item Econ
			\item PolSci
		\end{itemize}

\end{itemize}
\section{B. How to differentiate anything}%
\label{sec:b_how_to_differentiate_anything}
Example: $e^{x \arctan(x)}$

\subsection{Geometric interpretation:}%
\label{sub:geometric_interpretation_}

Find the tangent line to $y = f(x)$ at $P = (x_0,y_0)$

The tangent line is defined by the equation $y-y_0=m(x-x_0)$

Point: $y_0=f(x_0)$

Slope: $m=f'(x_0)$

\begin{definition}[Derivative]
	$f'(x_0)$, the derivative of $f$ at $x_0$, is the slope
	of the tangent line to $y=f(x)$ at the point $P$
\end{definition}

Taken for granted the geometric interpretation, we know the line through a point,
but we want to annalytically describe the tangent in a way that a machine could reproduce.

First grasp this with language.

$y-y_0 = m(x-x_0$

point \[
	y_0 = f(x_0)
.\]

slope: \[
	m = f'(x_0)
.\]

\begin{definition}[Derivativ]
	$f'(x_0)$, the derivative of $f$ at $x_0$, si the slope of the tangent line to
	$y=f(x)$ at P.
\end{definition}

\begin{definition}[Tangent Line]
	Limit of the secant lines $PQ$ as $Q\to P$ (P fixed)
\end{definition}

\subsection{Finding the Slope}%
\label{sub:finding_the_slope}

To find the slope between P and Q, we denote the horizontal distance as $\Delta x$,
and the height as $\Delta f$

The slope of the secand line is $\frac{\Delta  f}{\Delta x}$
 The slope of the tangent line is

 \noteworthy{
m = \lim_{\Delta x \to 0 } \frac{\Delta f}{\Delta x}
}{Slope of the tangent line}

So we redefine the points P and Q to make this formula more usable, \[
	P = (x_0,f(x_0)
.\] \[
Q = (x_0 + \Delta x, f(x_0+\Delta x)
.\]
\noteworthy{
f'(x_0) = \lim_{\Delta x \to 0}\frac{f(x_0+\Delta x) - f(x_0)}{\Delta x}
}{Limit definition of derivative}

\begin{example}[Example 1]
	$f(x) = \frac{1}{x}$
\end{example}
\begin{solution}[Example 1]
	\begin{align*}
		\frac{\Delta f}{\Delta x} &= \frac{\frac{1}{x_0+\Delta x}- \frac{1}{x_0}}{\Delta x} \\
								  &= \frac{1}{\Delta x}\left( \frac{x_0-(x_0+\Delta x}{(x_0+\Delta x)x_0} \right) \\
								  &= \frac{-1}{(x_0+\Delta x)x_{0}} \to (\Delta x \to 0) \to \\
								  &= \frac{-1}{x_0^2}
\end{align*}
\end{solution}

Note: The calculus part of calculus is easy, but it is made hard by the putting it in the
context of everything learned up to that point.

\begin{example}[Geometry problem]
	Find areas of triangles enclosed by the axes and tangent to $y = \frac{1}{x}$
\end{example}
\begin{solution}[Geometry problem]
	$y = \frac{1}{x}$

	Pick a point $(x_0,y_0$ on this curve, and we must find the base and the height. Find the
	formula for the tangent line:\[
		y - y_0 = \frac{-1}{x_0^2}(x-x_0)
	.\]
	Find the x-intercept:
	\begin{align*}
		(y&= 0) \\
		0- \frac{1}{x_0}&= \frac{-1}{x_0^2}(x-x_0) \\
		&= \frac{-x}{x_0^2} + \frac{1}{x_0} \\
		\frac{x}{x_0^2} &= \frac{2}{x_0} \\
		&= 2x_0
	\end{align*}
\end{solution}


%  %  %  %  %  %  %  %  %  %  %  %  %  %  %  %  %  %  %  %  %  %  %  %  %  %  %  %
\newpage
%%%
\end{document}

%LEAVE EMPTY ROW ABOVE THIS ONE
