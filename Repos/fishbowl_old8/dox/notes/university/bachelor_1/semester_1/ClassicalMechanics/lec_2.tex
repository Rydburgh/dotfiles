%%%\documentclass[a4paper,11pt,twoside]{report}o

%%%%| |_) | |_) |  _|   / _ \ | |\/| |  _ \| |   |  _|
%|  __/|  _ <| |___ / ___ \| |  | | |_) | |___| |___
%|_|   |_| \_\_____/_/   \_\_|  |_|____/|_____|_____|
% Jasper Runco
% last updated: 2020-05-20
%%%%%%%%%%%%%%%%%%%%%%%%%%%%%%%%%%%%%%%%%%%%%%%%%%%%%%%%%%%%%%%%%%%%%%%%%%%%%%%%%%%%%%%%%%
%%% PACKAGES
%%%%%%%%%%%%%%%%%%%%%%%%%%%%%%%%%%%%%%%%%%%%%%%%%%%%%%%%%%%%%%%%%%%%%%%%%%%%%%%%%%%%%%%%%%
\pdfminorversion=7						% to prevent errors when building pdf
% some basic packages
\usepackage{amsthm, amsmath}
\usepackage{url}						% to format hyperlink text
\usepackage{float}						% for custom figure/table environment
\usepackage{xifthen}					% to handle tests
\usepackage{booktabs}					% for table commands and optimisation
\usepackage{enumitem}					% to format enumerate, itemize, and description
\usepackage{textcomp}					% to support different glyphs
\usepackage{graphicx}					% to support \includegraphics
\usepackage[T1]{fontenc}				% for unicode encoding
\usepackage[utf8]{inputenc}				% for unicode input
\setlength{\headheight}{13.6pt}
\usepackage[top=1.5in,bottom=1in,right=1in,left=1in,headheight=45pt]{geometry}
\usepackage{fancyhdr}					% for adding different headers
\pagestyle{fancy}
% for list of equations
\usepackage{tocloft}					% for custom lists
\usepackage{ragged2e} 					% to undo \centering
\usepackage{hyperref} 					% to make references hyperlinks
\usepackage{glossaries}
% for figures
\usepackage{import}						% for file control
\usepackage{pdfpages}					% for pdf, graphics, and hypertext
\usepackage{transparent}				% for color stack transparency
\usepackage{xcolor}						% for arbitrary color mixing
%%%%%%%%%%%%%%%%%%%%%%%%%%%%%%%%%%%%%%%%%%%%%%%%%%%%%%%%%%%%%%%%%%%%%%%%%%%%%%%%%%%%
% Commands
%%%%%%%%%%%%%%%%%%%%%%%%%%%%%%%%%%%%%%%%%%%%%%%%%%%%%%%%%%%%%%%%%%%%%%%%%%%%%%%%%%%%
% to make a new figure
\newcommand{\incfig}[2][1]{%
	\def\svgwidth{#1\columnwidth}
	\import{./figures/}{#2.pdf_tex}
}
\pdfsuppresswarningpagegroup=1
% define list of equations
\newcommand{\listequationsname}{\Large{List of Equations}}
\newlistof{myequations}{equ}{\listequationsname}
\newcommand{\myequations}[1]{
	\addcontentsline{equ}{myequations}{\protect\numberline{\theequation}#1}
}
\setlength{\cftmyequationsnumwidth}{2.3em}
\setlength{\cftmyequationsindent}{1.5em}
% command to box, label, refference, and include
% noteworthy equations in list of equations
\newcommand{\noteworthy}[2]{
\begin{align} \label{#2} \ensuremath{\boxed{#1}} \end{align}
\myequations{#2} \centering \small \textit{#2} \normalsize \justify }
%%%%%%%%%%%%%%%%%%%%%%%%%%%%%%%%%%%%%%%%%%%%%%%%%%%%%%%%%%%%%%%%%%%%%%%%%%%%%%%%%%%%
% Theorems
%%%%%%%%%%%%%%%%%%%%%%%%%%%%%%%%%%%%%%%%%%%%%%%%%%%%%%%%%%%%%%%%%%%%%%%%%%%%%%%%%%%%
\newtheorem{definition}{Definition}
\newtheorem{theorem}{Theorem}
\newtheorem{lemma}{Lemma}
\newtheorem{corollary}{Corollary}
o

%%%\begin{document}o

\chapter{Kinematic formulas}%
\label{cha:kinematic_formulas}

\LARGE\textsc{Date: 2020-05-21} \\ \LARGE\textsc{Announcements:}

\paragraph \hrule \paragraph \\ \fancyhead[R]{Lesson 2} \fancyhead[L]{Week 1}
%  %  %  %  %  %  %  %  %  %  %  %  %  %  %  %  %  %  %  %  %  %  %  %  %  %  %  %
\section{What are the kinematic formulas?}%
\label{sec:what_are_the_kinematic_formulas_}
The kinematic formulas are how we relate the five kinematic (constant acceleration)
variables below.
\begin{table}[htpb]
	\centering
	\caption{Kinematic variables}
	\label{tab:label}
	\begin{tabular}{l l}
		$\Delta x$ & Displacement \\
		$t$ & Time interval \\
		$v_0$ & Initial velocity \\
		$v$ & Final velocity \\
		$a$ & Constant acceleration
	\end{tabular}
\end{table}

If acceleration is constant, we can use a kinematic formula to solve for the unknown. The
kinematic formulas are written as the following four equations:
\begin{align*}
	v &= v_0 + at \\
	\Delta x &= \left( \frac{v + v_0}{2} t \right) \\
	\Delta x &= v_0 t + \frac{1}{2} a t^2 \\
	v^2 &= v_0^2 + 2a \Delta x
.\end{align*}

\begin{definition}[Free flying object]
	All free flying objects -- also called projectiles -- on Earth, regardless of mass,
	have a constant downward acceleration due to gravity
\end{definition}

\begin{constant}[Acceleration due to gravity]
	$g = 9.81 \frac{m}{s^2}$
\end{constant}

\section{Selecting kinematic formula}%
\label{sec:selecting_kinematic_formula}

Each formula is missing one variable, so we choose the one that has three variables we
already know and the one we are looking for.
\subsection{Problem solving tip:}%
\label{sub:problem_solving_tip_}

Note that each of the formulas is missing one of the variables:
\begin{table}[htpb]
	\centering
	\caption{}
	\label{tab:label}
	\begin{tabular}{l c c}
		1. & $v = v_0 + at$ & (This formula is missing $\Delta x$.)  \\
		2. & $\Delta x = \left( \frac{v + v_0}{2} \right) t $ & (This formula is missing $a$.)   \\
		3. & $\Delta x = v_0 t + \frac{1}{2} a t^2$   & (This formula is missing $v$.)  \\
		4. & $v^2 = v_0^2 + 2a \Delta x$  &  (This formula is missing $t$.)\\
	\end{tabular}
\end{table}
\section{Deriving the first kinematic formula, $v = v_0 + at$}%
\label{sec:deriving_the_first_kinematic_formula_v_v_0_at_}

This one is just a rearranged version of the definition of acceleration, \[
	a = \frac{\Delta v}{\Delta t}
	.\] Now we replace $\Delta v$ with the definition of the change in velocity. \[
	a = \frac{v - v_0}{\Delta t}
	.\] Finally solve for $v$ to get \[
	v = v_0 + a \Delta t
.\] and $\Delta t$ just becomes $t$ to arrive at the \textbf{first kinematic formula}
\noteworthy{
	v = v_0 +at
}{first kinematic formula}

\section{Deriving the second kinematic formula, $\Delta x = \left( \frac{v + v_0}{2} \right) t$}%
\label{sec:deriving_the_second_kinematic_formula_delta_x_left_v + v_0_2_right_t_}

Visualize this kinematic formula by looking at a velocity graph for an object
with constant acceleration (constant slope) with some initial velocity. See figure \ref{fig:velocity-graph}

\begin{figure}[ht]
	\centering
	\incfig{velocity-graph}
	\caption{Velocity graph}
	\label{fig:velocity-graph}
\end{figure}

The area under the graph gives displacement $\Delta x$. We can break this down into
a rectangle and a triangle. The height of the blue rectangle is $v_0$, so its
area is $v_0t$. The base of the red triangle is $t$ and its height
is $v-v_0$, so the are of the triangle is $\frac{1}{2}t \left( v - v_0 \right) $.
The total area will be \[
	\Delta x = v_0t + \frac{1}{2} v t - \frac{1}{2} v_0 t
	.\] We can simplify to get \[
	\Delta x = \frac{1}{2} v t + \frac{1}{2} v_0 t
.\] And finally rewrite the right side to get the \textbf{second kinematic formula}
\noteworthy{
	\Delta x = \left( \frac{v + v_0}{2} \right) t
}{Second kinematic formula}

\section{Deriving the third kinematic formula, $\Delta x = v_0t + \frac{1}{2} a t^2$}%
\label{sec:deriving_the_third_kinematic_formula_delta_x_v_0t_1_2_a_t_2_}

Reconsider the figure \ref{fig:velocity-graph}, since the area under the graph is
the displacement, we can again use geometry to derive the following:
\noteworthy{
	\Delta x = v_0 t + a t^2
}{Third kinematic formula}

\section{Deriving the fourth kinematic formula, $v^2 = v_0^2 + 2a \Delta x$}%
\label{sec:deriving_the_fourth_kinematic_formula_v_2_v_0_2_2a_delta_x_}

To derive this formula, we start with the second kinematic formula: \[
	\Delta x = \left( \frac{v + v_0}{2} \right) t
.\] Then we eliminate $t$ from this formula by solving the first kinematic
equation for $t$ to get $t = \frac{v -v_0}{a}$, then substituting this
into the second formula: \[
	\Delta x = \left( \frac{v + v_0}{2} \right) \left( \frac{v - v_0}{a} \right)
	.\] Multiplying the fractions gives \[
	\Delta x = \left( \frac{v^2-v_0^2}{2a} \right)
.\] And now solve for $v^2$ to get the fourth kinematic equation.
\noteworthy{
	v^2 = v_0^2 +2a\Delta x
}{Fourth kinematic formula}

\section{Common confusion}%
\label{sec:common_confusion}

\paragraph{Don't forget}%
\label{par:don_t_forget}
\begin{itemize}
	\item  These equations are only true when acceleration is constant.
	\item Sometimes the known variables will not be explicitly given, but rather
		implied by \textbf{codewords} like "starts from rest" or "comes to a stop"
	\item All the variables but $t$ can be negative.
	\item  The third kinematic formula, equation \ref{Third kinematic formula}, might
		require the use of the \textbf{quadratic formula}.
\end{itemize}

\section{Examples}%
\label{sec:examples}

\begin{example}[First kinematic formula, $v = v_0 + at$]
	A water balloon filled with Kool-Aid is dropped from the
	top of a building. What is the velocity of the watter balloon
	after falling for $t = 2.35$ s?
\end{example}

\begin{solution}[]
	Assuming upward is the positive direction, our known variables are
	\begin{table}[htpb]
		\centering
		\caption{}
		\label{tab:label}
		\begin{tabular}{l r}
			$v_0 = 0$ & (Since the water balloon was dropped from rest.) \\
			$t = 2.35 s$ & (This is the time interval after which we want the velocity.) \\
			$a_g = -9.81 \frac{m}{s^2}$ & (This is implied by freely falling.)
		\end{tabular}
	\end{table}
\end{solution}

Use $y$ as the position variable because the motion is vertical. Since we don't know the
displacement and weren't asked to find displacement, use the first kinematic formula, which
is missing $\Delta y$.
\begin{align*}
	v &= v_0 +at \\
	v &= 0 \frac{m}{s} + (-9.81 \frac{m}{s^2}(2.35 s) \\
	v &=  -23.1 \frac{m}{s}
\end{align*}

\begin{note}[]
	The final velocity was negative since the water balloon was heading downward.
\end{note}

\begin{example}[Second kinematic formula, $\Delta x = \left( \frac{v + v_0}{2} \right) t$]
	A leopard is running at $6.20 \frac{m}{s}$ and then speeds up to $23.1 \frac{m}{s}$
	in a time of $3.3 s$. How much ground did it cover?
\end{example}
\begin{solution}[]
	Assume the initial direction is positive, our known variables are
	\begin{align*}
		v_0 &=  6.20 \frac{m}{s} \\
		v &= 23.1 \frac{m}{s} \\
		t &= 3.30 s
	.\end{align*}
	Since we don't know and aren't asked for acceleration, we use the second kinematic
	formula for the horizontal direction, which is missing $a$.
	\begin{align*}
		\Delta x &=  \left( \frac{v + v_0}{2} \right) t \\
		\Delta x &=  \left( \frac{23.1 \frac{m}{s} + 6.20 \frac{m}{s}}{2} \right) (3.30 s) \\
		\Delta x &= 48.3 m
	\end{align*}
\end{solution}

\begin{example}[third kinematic formula, $\Delta x = v_0t + \frac{1}{2} a t^2$]
	A student throws her pencil straight up at $18.3 \frac{m}{s}$. How long
	does it take the pencil to reach a point $12.2m$ higher than where
	it was thrown?
\end{example}
\begin{solution}[]
	Assuming upward is the positive direction, our known variables are
	\begin{align*}
		v_0 &=  18.3 \frac{m}{s} \\
		\Delta y &=  12.2 m \\
		a &= -9.81 \frac{m}{s}
	.\end{align*}

	Since we don't know and aren't asked for the final velocity, use the third
	kinematic formula for the vertical direction:
	\begin{align*}
		\Delta y &=  v_{0_y} t + \frac{1}{2}a_y t^2 \\
		\Delta y &= v_{0_y}t + \frac{1}{2}a_yt^2 \\
	.\end{align*}
\end{solution}

\begin{note}[]
	We cannot solve this algebraically if none of the terms are zero because it is a
	quadratic equation, so we substitute the known values and solve with the quadratic equation:
\end{note}
\begin{align*}
	0 &= \frac{1}{2}(-9.81 \frac{m}{s^2} + (18.3 \frac{m}{s}t - 12.2 m \\
	t &= \frac{-b \pm \sqrt{b^2 - 4ac}}{2a} \\
	t &= \frac{-18.3 \frac{m}{s} \pm \sqrt{(18.3 \frac{m}{s})^2 - 4 \left[ \frac{1}{2}(-9.81 \frac{m}{s^2})(-12.2m)\right]  }}{2\left[ \frac{1}{2}(-9.81 \frac{m}{s^2} \right] } \\
	t &= 0.869s \:\text{and}\:t = 2.86 s
.\end{align*}

There are two answers, which tells us that the pencil reaches this height twice in it's trajectory. We will chose
the smaller time $t = 0.869 s$

\begin{example}[fourth kinematic formula, $v^2 = v_0^2 + 2a \Delta x$]
A motorcyclist starts with a speed of $23.4 \frac{m}{s}$ and slows down over a length of $50.2 m$
with a constant deceleration of magnitude $3.20 \frac{m}{s^2}$. Assume the it is moving forward
the entire trip. What is the new velocity of the motorcycle after slowing down through the $50.2 m$?
\end{example}
\begin{solution}[]
	Assume the initial direction of travel is the positive direction, our known variables are
	\begin{align*}
		v_0 &=  23.4 \frac{m}{s} \\
		a &=  -3.20 \frac{m}{s^2} \\
		\Delta x &= 50.2 m
	.\end{align*}
	Since we don't know and aren't asked to find the time, we use the fourth kinematic formula
	for the horizontal direction
	\begin{align*}
		v_x^2 &=  v_{0x}^2 +2a_x\Delta x \\
		v_{x} &=  \pm \sqrt{v_{0x}^2 + 2a_{x}\Delta x}  \\
			  &\:\text{We will assume the positive answer}\: \\
		v_{x} &=  \sqrt{v_{0x}^2 + 2a_{x}\Delta x}  \\
		v_{x} &= \sqrt{(23.4 \frac{m}{s})^2 + 2(-3.20 \frac{m}{s^2})(50.2m)} \\
		v_{x} &= 15.0 \frac{m}{s}
	\end{align*}
\end{solution}

%  %  %  %  %  %  %  %  %  %  %  %  %  %  %  %  %  %  %  %  %  %  %  %  %  %  %  %
\newpage
%%%\end{document}o


%LEAVE EMPTY ROW ABOVE THIS ONE
