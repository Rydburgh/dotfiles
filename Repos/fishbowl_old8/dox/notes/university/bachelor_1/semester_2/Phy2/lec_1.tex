%%%\documentclass[a4paper,11pt,twoside]{report}

%%%%| |_) | |_) |  _|   / _ \ | |\/| |  _ \| |   |  _|
%|  __/|  _ <| |___ / ___ \| |  | | |_) | |___| |___
%|_|   |_| \_\_____/_/   \_\_|  |_|____/|_____|_____|
% Jasper Runco
% last updated: 2020-05-20
%%%%%%%%%%%%%%%%%%%%%%%%%%%%%%%%%%%%%%%%%%%%%%%%%%%%%%%%%%%%%%%%%%%%%%%%%%%%%%%%%%%%%%%%%%
%%% PACKAGES
%%%%%%%%%%%%%%%%%%%%%%%%%%%%%%%%%%%%%%%%%%%%%%%%%%%%%%%%%%%%%%%%%%%%%%%%%%%%%%%%%%%%%%%%%%
\pdfminorversion=7						% to prevent errors when building pdf
% some basic packages
\usepackage{amsthm, amsmath}
\usepackage{url}						% to format hyperlink text
\usepackage{float}						% for custom figure/table environment
\usepackage{xifthen}					% to handle tests
\usepackage{booktabs}					% for table commands and optimisation
\usepackage{enumitem}					% to format enumerate, itemize, and description
\usepackage{textcomp}					% to support different glyphs
\usepackage{graphicx}					% to support \includegraphics
\usepackage[T1]{fontenc}				% for unicode encoding
\usepackage[utf8]{inputenc}				% for unicode input
\setlength{\headheight}{13.6pt}
\usepackage[top=1.5in,bottom=1in,right=1in,left=1in,headheight=45pt]{geometry}
\usepackage{fancyhdr}					% for adding different headers
\pagestyle{fancy}
% for list of equations
\usepackage{tocloft}					% for custom lists
\usepackage{ragged2e} 					% to undo \centering
\usepackage{hyperref} 					% to make references hyperlinks
\usepackage{glossaries}
% for figures
\usepackage{import}						% for file control
\usepackage{pdfpages}					% for pdf, graphics, and hypertext
\usepackage{transparent}				% for color stack transparency
\usepackage{xcolor}						% for arbitrary color mixing
%%%%%%%%%%%%%%%%%%%%%%%%%%%%%%%%%%%%%%%%%%%%%%%%%%%%%%%%%%%%%%%%%%%%%%%%%%%%%%%%%%%%
% Commands
%%%%%%%%%%%%%%%%%%%%%%%%%%%%%%%%%%%%%%%%%%%%%%%%%%%%%%%%%%%%%%%%%%%%%%%%%%%%%%%%%%%%
% to make a new figure
\newcommand{\incfig}[2][1]{%
	\def\svgwidth{#1\columnwidth}
	\import{./figures/}{#2.pdf_tex}
}
\pdfsuppresswarningpagegroup=1
% define list of equations
\newcommand{\listequationsname}{\Large{List of Equations}}
\newlistof{myequations}{equ}{\listequationsname}
\newcommand{\myequations}[1]{
	\addcontentsline{equ}{myequations}{\protect\numberline{\theequation}#1}
}
\setlength{\cftmyequationsnumwidth}{2.3em}
\setlength{\cftmyequationsindent}{1.5em}
% command to box, label, refference, and include
% noteworthy equations in list of equations
\newcommand{\noteworthy}[2]{
\begin{align} \label{#2} \ensuremath{\boxed{#1}} \end{align}
\myequations{#2} \centering \small \textit{#2} \normalsize \justify }
%%%%%%%%%%%%%%%%%%%%%%%%%%%%%%%%%%%%%%%%%%%%%%%%%%%%%%%%%%%%%%%%%%%%%%%%%%%%%%%%%%%%
% Theorems
%%%%%%%%%%%%%%%%%%%%%%%%%%%%%%%%%%%%%%%%%%%%%%%%%%%%%%%%%%%%%%%%%%%%%%%%%%%%%%%%%%%%
\newtheorem{definition}{Definition}
\newtheorem{theorem}{Theorem}
\newtheorem{lemma}{Lemma}
\newtheorem{corollary}{Corollary}


%%%\begin{document}

\chapter{Electric Charge and Electric Field}%
\label{cha:sylabus}


\LARGE\textsc{Date: 2020-08-17} \\ \\ \LARGE\textsc{Announcements:} \\
\small

\begin{itemize}
	\item[Instructor -] Dr. Emily Marshman
	\item[Office Hours -] M 12:00pm - 1:00pm, T 10:00am - 1:00pm, W 12:00pm - 1:00am
	\item[Email -] emarshman@ccac.edu
	\item[Book -] University Physics with Modern Physics by Young and Freedman, 15th edition
\end{itemize}

\textbf{Assignment (Sept 20 11:59): Chapter 21 Homework} \\

\textbf{Assignment (Aug 31): Lab 1} \\

\paragraph \hrule \paragraph \\ \fancyhead[R]{Lesson 1} \fancyhead[L]{Week 1}
%  %  %  %  %  %  %  %  %  %  %  %  %  %  %  %  %  %  %  %  %  %  %  %  %  %  %  %
\section{Introduction}%
\label{sec:chapter_21_electric_charge_and_electric_field}

\subsection{Learning outcomes}%
\label{sub:learning_outcomes}

\begin{itemize}
	\item How objects become charged, and how we know it's conserved.
	\item How to use Coulombs law.
	\item Distinction between electric force and field.
	\item How to use idea of electric field lines.
	\item Calculate properties of electric charge distributions.
\end{itemize}

\section{Electric Charge and Electric Fields}%
\label{sec:electric_charge_and_electric_fields}


\subsection{Electric charge}%
\label{sub:electric_charge}

When we rub glass rods with silk, the rods become charged and repel each other.

A charged plastic rod \textbf{attracts} a charged glass rod, and both attract the cloth.

This shows there are two kinds of charge.

\subsubsection{Electric charge and the structure of matter}%
\label{ssub:electric_charge_and_the_structure_of_matter}


The particles of the atoms are:

\begin{itemize}
	\item the negative electrons
	\item the positive protons
	\item the uncharged neutrons
\end{itemize}

\subsubsection{Atoms and Ions}%
\label{ssub:atoms_and_ions}

A neutral atom has the same number of protons as electrons

A \textbf{positive ion} has one or more electrons removed.

A \textbf{negative ion} has an excess of electrons.

\subsection{Conservation of Charge}%
\label{sub:conservation_of_charge}

\begin{itemize}
	\item The proton and electron have the same magnitude of charge.
	\item This magnitude is \textbf{quantized} unit of charge.
	\item \textbf{principle of charge conservation} states the sum of all charges in a closed system is constant.
\end{itemize}

\noteworthy{
	1 \:\text{Coulomb}\: = \:\text{charge on}\: 6.241 \times 10^{18} protons
}{Coulomb}

\noteworthy{
	\:\text{Charge on 1 proton, +e}\: = 1.6 \times 10^{-19} C
}{Proton charge}

\noteworthy{
	\:\text{Charge on 1 electron, -e}\: = -1.6 \times 10^{-19} C
}{Electron charge}
\begin{example}[]
	Common static electricity involves charges ranging from nanocoulombs to mircrocoulombs.

	(a) How many electrons are needed to form a charge of $-2.00nC$?

	(b) How many electrons must be removed from a neutral object to leave a net charge of $0.500 \mu C$
\end{example}

\begin{solution}[a]
	\begin{align*}
		\:\text{Charge of }\:e&= -1.6 \times  10^{-19} C\\
		-2.00 \mu C &= -2.00 \times  10^{-9} C \implies\\
		\frac{-2.00\times 10^{-9}C}{-1.6 \times 10^{-19}C} &= 1.25 \times 10^{10}\:\text{electrons}\:
	\end{align*}
\end{solution}

\begin{solution}[b]
	\begin{align*}
		0.500 \mu C &= 0.500 \times 10^{-19}C  \implies\\
		\frac{0.500\times 10^{-9}C}{-1.6\times 10^{-19}C} &=  3.13 \times 10^{11} \:\text{electrons removed}\:\\
	\end{align*}

\end{solution}

\begin{definition}[Conductor]
	A material that allows charge to flow through it easily (most metals).
\end{definition}
\begin{definition}[Insulator]
	a material that does not allow charge to flow through it easily (e.g. plastic,
	paper, nylon, wood).
\end{definition}

\section{Conductors, Insulators, and Induced Charges}%
\label{sec:conductors_insulators_and_induced_charges}


\subsection{Charge by contact}%
\label{sub:charge_by_contact}

Electrons are transfered by rubbing the negatively charged rod on the metal sphere.

When the rod is removed, the electron distribute themselves over the surface.

\subsection{Without contact (Induction)}%
\label{sub:without_contact_induction_}

\begin{enumerate}
	\item An uncharged metal ball stands on an insulator
	\item free electrons in the metal ball are repelled by the excess in the rod, and shift
		away from the rod.
	\item While the rod is near, connect the ball to the ground with a conducting wire.
	\item Free electrons in the metal ball are repelled by the excess in the rod, and shift
		away from the rod.
	\item disconnect the wire and a net positive charge is left on the ball.
		The earth acquires an equal negative charge.
\end{enumerate}

\subsection{Electric Forces on Uncharged Objects}%
\label{sub:electric_forces_on_uncharged_objects}

\begin{itemize}
	\item negative plastic come causes shifting of charges within the neutral insulator, called \textbf{polarization}
	\item a charged object of \textbf{either} sign exerts an \textbf{attractive} force on an uncharged insulator.
\end{itemize}

\section{Coulomb's Law}%
\label{sec:coulomb_s_law}

The magnitude of the electric force between two point charges is directly proportional to the product of their
charges and inversely proportional to the square of the distance between them.

\noteworthy{
	F_{e} = k \frac{\left| q_{1}q_{2} \right| }{r^2}
}{Coulomb's Law}

The direction of the force depends on the relative sign of the charge.

\noteworthy{
	k = \frac{1}{4\pi \epsilon_{0}} = 9.0\times 10^{9} \frac{Nm^2}{C^2}
}{Proportionality Constant}

\noteworthy{
	\epsilon_{0} = 8.85 \times 10^{-12} \frac{C^2}{Nm^2}
}{Electric constant}

\begin{example}[Applying Coulomb's law]

\end{example}


\section{Electric Fields}%
\label{sec:electric_fields}

A charged object modifies the properties of the space around it. E.g. in the vicinity of a positive charge, other positives repell and negatives attract

Gravitational field:

\begin{align*}
	F_{g} &= \frac{G Mm}{r^2}\\
	\frac{F_{g}}{m} &= \frac{GM}{r^2} \\
					&	\:\text{near earths surface}\:\\
	\frac{\overline{F}_{g}}{m} &=  \overline{g} = 9.8 \frac{m}{s^2} \\
	\overline{F}_{g}&= m\overline{g} \\
\end{align*}

Electric field:
\begin{align*}
	F_{e} &= \frac{k\left| Qq_{0} \right| }{r^2} \\
	\frac{F_{e}}{q_{0}} &= \frac{kQ}{r^2} = \overline{E} \\
	\:\text{force / charge}\: & \:\text{: electric field of point charge}\:
\end{align*}

\noteworthy{
	\overline{E} = \frac{1}{4\pi \epsilon_{0}}\frac{q}{r^2}\hat{r}
}{Electric field of a point charge}

\begin{description}
	\item[$\epsilon_{0}$] Electric constant
	\item[q -] Value of point charge
	\item[r -] Distance from point charge to where field is measured
	\item[$\hat{r}$ -] unit vector from point charge towards where the field is measured
\end{description}

Direction: electric field points \textbf{radially inward} towards negative charge and
\textbf{radially outward} for positive charge.

%  %  %  %  %  %  %  %  %  %  %  %  %  %  %  %  %  %  %  %  %  %  %  %  %  %  %  %
\newpage
%%%\end{document}


%LEAVE EMPTY ROW ABOVE THIS ONE
