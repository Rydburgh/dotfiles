%%%\documentclass[a4paper,11pt,twoside]{report}

%%%% ____  ____  _____    _    __  __ ____  _     _____
%|  _ \|  _ \| ____|  / \  |  \/  | __ )| |   | ____|
%| |_) | |_) |  _|   / _ \ | |\/| |  _ \| |   |  _|
%|  __/|  _ <| |___ / ___ \| |  | | |_) | |___| |___
%|_|   |_| \_\_____/_/   \_\_|  |_|____/|_____|_____|
%
% last updated: 2020-05-21
%%%%%%%%%%%%%%%%%%%%%%%%%%%%%%%%%%%%%%%%%%%%%%%%%%%%%%%%%%%%%%%%%%%%%%%%%%%%%%%%%%%%%%%%%%
%%% PACKAGES
%%%%%%%%%%%%%%%%%%%%%%%%%%%%%%%%%%%%%%%%%%%%%%%%%%%%%%%%%%%%%%%%%%%%%%%%%%%%%%%%%%%%%%%%%%


\pdfminorversion=7						% to prevent errors when building pdf

% some basic packages
\usepackage{amsmath, amsthm, amssymb}
\usepackage{mhchem}						% for chemical symbols
\usepackage{url}						% to format hyperlink text
\usepackage{float}						% for custom figure/table environment
\usepackage{xifthen}					% to handle tests
\usepackage{booktabs}					% for table commands and optimisation
\usepackage{enumitem}					% to format enumerate, itemize, and description
\usepackage{textcomp}					% to support different glyphs
\usepackage{graphicx}					% to support \includegraphics
\usepackage[T1]{fontenc}				% for unicode encoding
\usepackage[utf8]{inputenc}				% for unicode input
\setlength{\headheight}{13.6pt}
\usepackage[top=1.5in,bottom=1in,right=1in,left=1in,headheight=45pt]{geometry}
\usepackage{fancyhdr}					% for adding different headers
\pagestyle{fancy}
% for list of equations
\usepackage{tocloft}					% for custom lists
\usepackage{ragged2e} 					% to undo \centering
\usepackage{hyperref} 					% to make references hyperlinks
\usepackage{glossaries}

% for figures
\usepackage{import}						% for file control
\usepackage{pdfpages}					% for pdf, graphics, and hypertext
\usepackage{transparent}				% for color stack transparency
\usepackage{xcolor}						% for arbitrary color mixing



\author{Jasper Runco}
\date{2020 // Fall}

%%%%%%%%%%%%%%%%%%%%%%%%%%%%%%%%%%%%%%%%%%%%%%%%%%%%%%%%%%%%%%%%%%%%%%%%%%%%%%%%%%%%
% Commands
%%%%%%%%%%%%%%%%%%%%%%%%%%%%%%%%%%%%%%%%%%%%%%%%%%%%%%%%%%%%%%%%%%%%%%%%%%%%%%%%%%%%

% to make a new figure
\newcommand{\incfig}[2][scale=1]{%
	% \def\svgwidth{#1\columnwidth}
	\import{./figures/}{#2.pdf_tex}
}
\pdfsuppresswarningpagegroup=1

% define list of equations
\newcommand{\listequationsname}{\Large{List of Equations}}
\newlistof{myequations}{equ}{\listequationsname}
\newcommand{\myequations}[1]{
	\phantomsection
	\addcontentsline{equ}{myequations}{\protect\numberline{\theequation}#1}
}

\setlength{\cftmyequationsnumwidth}{2.3em}
\setlength{\cftmyequationsindent}{1.5em}

% command to box, label, reference, and include
% noteworthy equations in list of equations
\newcommand{\noteworthy}[2]{
\begin{align} \label{#2} \ensuremath{\boxed{#1}} \end{align}
\myequations{#2} \centering \textit{#2} \justify}

\newtheorem{definition}{Definition}
\newtheorem{theorem}{Theorem}
\newtheorem{lemma}{Lemma}
\newtheorem{corollary}{Corollary}
\newtheorem{example}{Example}
\newtheorem{solution}{Solution}
\newtheorem{constant}{Constant}
\newtheorem{note}{Note}


%%%\begin{document}

% cha
\LARGE\textsc{Date: 2020-09-02} \\ \\ \LARGE\textsc{Announcements:} \\
\small

As an exercise for the reader, theorems 1.4.7 - 1.4.9 should be examined.

\textbf{Assignment: 1.4 (1-22, 25-28, 29-31, 39, 40, 45, 49, 50)}


\paragraph \hrule \paragraph \\ \fancyhead[R]{Lesson 7} \fancyhead[L]{Week 3}
%  %  %  %  %  %  %  %  %  %  %  %  %  %  %  %  %  %  %  %  %  %  %  %  %  %  %  %

\section{Inverse}%
\label{sec:inverse}

\subsection{$4\times 4$ Example}%
\label{sub:_4times_4_example}


The matrix $A = \begin{bmatrix} a & b \\ c & d \end{bmatrix} $ is invertible
if and only if $ad - bc$ $/= 0 $, in which case the inverse is given by the formula \[
	A^{-1} \neq \frac{1}{ad-bc}\begin{bmatrix} d & -b \\ -c & a \end{bmatrix} \]

	\begin{itemize}
		\item $ad - bc$ is called the determinant of A, $det(A)$.
	\end{itemize}

\subsubsection{Identity Proof}%
\label{ssub:identity_proof}

\[ A A^{-1} = I\] \[A^{-1}A = I\]

\[ \begin{bmatrix} a &b \\ c & d \end{bmatrix} \frac{1}{ad-bc}\begin{bmatrix} d & -b \\ -c & a \end{bmatrix} \]

\[ \frac{1}{ad-bc}\begin{bmatrix} a &b \\ c & d \end{bmatrix} \begin{bmatrix} d & -b \\ -c & a \end{bmatrix} \]

\[ \frac{1}{ad-bc}\begin{bmatrix} ad -bc & 0 \\ 0 & -bc + ad \end{bmatrix} \]

\[ \frac{1}{ad-bc}\begin{bmatrix} ad -bc & 0 \\ 0 & -bc + ad \end{bmatrix} \]

\[= \begin{bmatrix} 1 & 0 \\ 0 & 1 \end{bmatrix} \]

\begin{example}[]
	Solve
	\begin{align*}
		2x + 3y &= 5 \\
		-x + 7y &=  12 \\
	\end{align*}

	\[\begin{bmatrix} 2 & 3 \\ -1 & 7 \end{bmatrix}_{2\times 2} \begin{bmatrix} x \\ y \end{bmatrix}_{2\times 1} = \begin{bmatrix} 5 \\ 12 \end{bmatrix}_{2\times 1}  \]
	\[ [A] [x] = b\]

	If A is invertible:

	\[A^{-1}Ax = A^{-1}b\]

	The inverse must be a left side multiplication on both sides of equations.

	\[x = A^{-1}b\]

	\[\begin{bmatrix} x \\ y  \end{bmatrix} = \begin{bmatrix} \\ \end{bmatrix} \]

	\begin{align*}
	\begin{bmatrix} 2 & 3 \\ -1 & 7 \end{bmatrix}_{2\times 2} \begin{bmatrix} x \\ y \end{bmatrix}_{2\times 1} &= \begin{bmatrix} 5 \\ 12 \end{bmatrix}_{2\times 1}  \\
	A^{-1} &=  \frac{1}{17}\begin{bmatrix} 7 & -3 \\ 1 & 2 \end{bmatrix} \\
	\frac{1}{17}\begin{bmatrix} 7 & -3 \\ 1 & 2 \end{bmatrix} \begin{bmatrix} 2 & 3 \\ -1 & 7 \end{bmatrix} \begin{bmatrix} x \\ y \end{bmatrix} &= \frac{1}{17}\begin{bmatrix} 7 & -3 \\ 1 & 2  \end{bmatrix} \begin{bmatrix} 5 \\ 12 \end{bmatrix}  \\
	\frac{1}{17}\begin{bmatrix} 17 & 0 \\ 0 & 17 \end{bmatrix} \begin{bmatrix} x \\ 7 \end{bmatrix} &= \frac{1}{17}\begin{bmatrix} -1 \\29 \end{bmatrix}  \\
	\begin{bmatrix} x \\ y \end{bmatrix} &=  \begin{bmatrix} -\frac{1}{17} \\ \frac{29}{17} \end{bmatrix}
	\end{align*}

\end{example}

\begin{theorem}[]
	If B and C are both inverses of A, then $B = C$
\end{theorem}

\paragraph{Proof}
	If B is an inverse of A, $AB = BA = I$

	If C is an inverse of A, $AC = CA = I$

	\[(BA)C = IC = C\]
	\[B(AC) = BI = B\] \qed

\begin{theorem}[]
	If A and B are invertible matrices with the same size, then AB is invertible and
	\[(AB)^{-1} = B^{-1}A^{-1}\]
\end{theorem}

\paragraph{proof}
\[(AB)(AB)^{-1} = (AB) B^{-1}A^{-1} = (AI)A^{-1} = A A^{-1} = I\]
\[(AB)^{-1}(AB) = B^{-1}A^{-1}AB = B^{-1}IB = I\]

\begin{corollary}[]
\[A_1,A_2,\ldots,A_{n}\] are all invertible matrices of the same size, then
\[(A_1A_2 \ldots A_{n})\] is invertible and
\[(A_1A_2 \ldots A_{n})^{-1}= A_{n}^{-1}A_{n-1}^{-1}\ldots A_{2}^{-1}A_1^{-1}\]
\end{corollary}

\begin{definition}[]
	If A is a square matrix, then \[A^{n} = A A A \ldots A \{n factors\]

		\[A^{0} = I\]

		\[A^{-n} = (A^{-1})^{n} = A^{-1}A^{-1}\ldots A^{-1}\]
\end{definition}

\begin{theorem}[]
	If A is invertible and n is a nonnegative integer, then:

	$A^{-1}$ is invertible and $(A^{-1})^{-1} = A$.

	$A^{n}$ is invertible and $(A^{n})^{-1} = A^{-n} = (A^{-1})^{n}$

	$kA$ is invertible for any nonzero scalar $k$, and $(kA)^{-1} = k^{-1}A^{-1}$
\end{theorem}

\begin{theorem}[]
	If the sizes of the matrices are such that the stated operations can be performed, then:
	\begin{table}[htpb]
		\centering
		\caption{caption}
		\label{tab:label}
		\begin{tabular}{l l}
			(a)	 &  $(A^{T})^{T} = A$ \\
			(b)	 &  $(A+B)^{T} = A^{T} + B^{T}$ \\
			(c)	 &  $(A-B)^{T} = A^{T} - B^{T}$ \\
			(d)	 &  $(kA)^{T} =kA^{T}$ \\
			(e)	 &  $(AB)^{T} = A^{T}B^{T}$ \\
		\end{tabular}
	\end{table}
\end{theorem}

\begin{theorem}[]
	If A is an invertible matrix, then $A^{T}$ is also invertible and
	\[(A^{T})^{-1} = (A^{-1})^{T}\]
\end{theorem}



%  %  %  %  %  %  %  %  %  %  %  %  %  %  %  %  %  %  %  %  %  %  %  %  %  %  %  %
\newpage
%%%\end{document}


%LEAVE EMPTY ROW ABOVE THIS ONE
