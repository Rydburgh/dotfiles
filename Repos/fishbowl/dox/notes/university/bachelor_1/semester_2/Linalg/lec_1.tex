%%%\documentclass[a4paper,11pt,twoside]{report}

%%%%| |_) | |_) |  _|   / _ \ | |\/| |  _ \| |   |  _|
%|  __/|  _ <| |___ / ___ \| |  | | |_) | |___| |___
%|_|   |_| \_\_____/_/   \_\_|  |_|____/|_____|_____|
% Jasper Runco
% last updated: 2020-05-20
%%%%%%%%%%%%%%%%%%%%%%%%%%%%%%%%%%%%%%%%%%%%%%%%%%%%%%%%%%%%%%%%%%%%%%%%%%%%%%%%%%%%%%%%%%
%%% PACKAGES
%%%%%%%%%%%%%%%%%%%%%%%%%%%%%%%%%%%%%%%%%%%%%%%%%%%%%%%%%%%%%%%%%%%%%%%%%%%%%%%%%%%%%%%%%%
\pdfminorversion=7						% to prevent errors when building pdf
% some basic packages
\usepackage{amsthm, amsmath}
\usepackage{url}						% to format hyperlink text
\usepackage{float}						% for custom figure/table environment
\usepackage{xifthen}					% to handle tests
\usepackage{booktabs}					% for table commands and optimisation
\usepackage{enumitem}					% to format enumerate, itemize, and description
\usepackage{textcomp}					% to support different glyphs
\usepackage{graphicx}					% to support \includegraphics
\usepackage[T1]{fontenc}				% for unicode encoding
\usepackage[utf8]{inputenc}				% for unicode input
\setlength{\headheight}{13.6pt}
\usepackage[top=1.5in,bottom=1in,right=1in,left=1in,headheight=45pt]{geometry}
\usepackage{fancyhdr}					% for adding different headers
\pagestyle{fancy}
% for list of equations
\usepackage{tocloft}					% for custom lists
\usepackage{ragged2e} 					% to undo \centering
\usepackage{hyperref} 					% to make references hyperlinks
\usepackage{glossaries}
% for figures
\usepackage{import}						% for file control
\usepackage{pdfpages}					% for pdf, graphics, and hypertext
\usepackage{transparent}				% for color stack transparency
\usepackage{xcolor}						% for arbitrary color mixing
%%%%%%%%%%%%%%%%%%%%%%%%%%%%%%%%%%%%%%%%%%%%%%%%%%%%%%%%%%%%%%%%%%%%%%%%%%%%%%%%%%%%
% Commands
%%%%%%%%%%%%%%%%%%%%%%%%%%%%%%%%%%%%%%%%%%%%%%%%%%%%%%%%%%%%%%%%%%%%%%%%%%%%%%%%%%%%
% to make a new figure
\newcommand{\incfig}[2][1]{%
	\def\svgwidth{#1\columnwidth}
	\import{./figures/}{#2.pdf_tex}
}
\pdfsuppresswarningpagegroup=1
% define list of equations
\newcommand{\listequationsname}{\Large{List of Equations}}
\newlistof{myequations}{equ}{\listequationsname}
\newcommand{\myequations}[1]{
	\addcontentsline{equ}{myequations}{\protect\numberline{\theequation}#1}
}
\setlength{\cftmyequationsnumwidth}{2.3em}
\setlength{\cftmyequationsindent}{1.5em}
% command to box, label, refference, and include
% noteworthy equations in list of equations
\newcommand{\noteworthy}[2]{
\begin{align} \label{#2} \ensuremath{\boxed{#1}} \end{align}
\myequations{#2} \centering \small \textit{#2} \normalsize \justify }
%%%%%%%%%%%%%%%%%%%%%%%%%%%%%%%%%%%%%%%%%%%%%%%%%%%%%%%%%%%%%%%%%%%%%%%%%%%%%%%%%%%%
% Theorems
%%%%%%%%%%%%%%%%%%%%%%%%%%%%%%%%%%%%%%%%%%%%%%%%%%%%%%%%%%%%%%%%%%%%%%%%%%%%%%%%%%%%
\newtheorem{definition}{Definition}
\newtheorem{theorem}{Theorem}
\newtheorem{lemma}{Lemma}
\newtheorem{corollary}{Corollary}


%%%\begin{document}

\chapter{Systems of Linear Equations and Matrices}%
\label{cha:introduction}


\LARGE\textsc{Date: 2020-08-17} \\  \\ \LARGE\textsc{Announcements:} \\
\small
\begin{itemize}
	\item[Instructor -] Kathleen Kane
	\item[Office Hours -] MWF 11:30am - 12:30pm TR 8:00am - 9:00 am
		\item[Email -] kkane@ccac.edu
		\item[Phone -] (412) 237-4511
			\item[Book -] Elementary Linear Algebra: Applications Version by Howard Anton and Chris Rorries, 11th edition 9781118434413

\end{itemize}
\textbf{Assignment(Aug 19 08:50): practice uploading 3 scanned pages in a single pdf}


\paragraph \hrule \paragraph \\ \fancyhead[R]{Lesson 1} \fancyhead[L]{Week 1}
%  %  %  %  %  %  %  %  %  %  %  %  %  %  %  %  %  %  %  %  %  %  %  %  %  %  %  %
\section{Policies and Procedures}%
\label{sec:policies_and_procedures}


\subsection{Learning Outcomes}%
\label{ssec:learning_outcomes}

\begin{enumerate}
	\item Perform basic operations with vectors in n-dimensional space.
		\item Perform basic operations with matrices.
			\item Solve a system of m linear equations in n unknowns.
\item Prove basic theorems in a vector space.
	\item Perform basic operations with vectors in the standard matrix spaces and function space.
		\item Find the matrix repersentation of a linear transformation between two vector spaces.
		\item Find eigenvalues and eigenvectors for a given matrix.
			\item Perform basic operations in an inner product space
				\item Prove basic theorems in an inner product space.
\end{enumerate}
\subsection{Evaluation}%
\label{ssec:evaluation}

\begin{enumerate}
	\item Assignments (10\%)
	\item Testes (70\%)
	\item Final (weighted) (20\%)
\end{enumerate}

\subsection{Testing}%
\label{sub:testing}

\begin{enumerate}
	\item Required to scan test and submit via pdf
	\item 50 minutes each test and 10 minutes to submit test
	\item No make up tests
	\item One test may be substituted with final exam grade
	\item Missing final is automatic F.
\end{enumerate}

\section{Introduction to Systems of Linear Equations and Matrices}%
\label{sec:systems_of_linear_equations_and_matric}

\begin{example}[One solution]

Solve:
\begin{align*}
	3x + y &= 6 \\
	5x -3y &= 10
\end{align*}
\end{example}

\begin{solution}[]
	\begin{align*}
		3x + y &= 6 \implies \\
		9x + 3y &= 18 \\
		+[5x -3y &= 10] \\
		\cline{1-2}
		14x &= 28 \implies \\
			&\boxed{x = 2}
	\end{align*}
\end{solution}
\begin{figure}[ht]
    \centering
    \incfig{one-solution}
    \caption{one solution}
    \label{fig:one-solution}
\end{figure}

\begin{example}[Infinite solutions]
	Solve:
\begin{align*}
	2x - y &= 7 \\
4x - 2y &= 14
\end{align*}

\end{example}

\begin{solution}[]
	\begin{align*}
		4x - 2y &= 14 \implies \\
	2x - y &= 7 \\
	-[2x - y &= 7] \\
	\cline{1-2}
			 &\boxed{0=0}\:\text{(no solution)}\:
	\end{align*}
\end{solution}
\begin{figure}[ht]
    \centering
    \incfig{infinite-solutions}
    \caption{infinite solutions}
    \label{fig:infinite-solutions}
\end{figure}

\begin{example}[No solutions]
	\begin{align*}
		2x - y &= 6 \\
		4x - 2y &= 6
	\end{align*}
\end{example}

\begin{solution}[]
	\begin{align*}
		[4x - 2y &= 6] \implies \\
		2x - y &= 3 \\
		-[2x - y &= 6] \\
		\cline{1-2}
				 &\boxed{0=-3}\:\text{(false equation)}\:
	\end{align*}
\end{solution}



%  %  %  %  %  %  %  %  %  %  %  %  %  %  %  %  %  %  %  %  %  %  %  %  %  %  %  %
\newpage
%%%\end{document}


%LEAVE EMPTY ROW ABOVE THIS ONE
