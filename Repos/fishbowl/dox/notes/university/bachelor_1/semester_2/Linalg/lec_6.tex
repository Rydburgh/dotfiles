%%%\documentclass[a4paper,11pt,twoside]{report}

%%%% ____  ____  _____    _    __  __ ____  _     _____
%|  _ \|  _ \| ____|  / \  |  \/  | __ )| |   | ____|
%| |_) | |_) |  _|   / _ \ | |\/| |  _ \| |   |  _|
%|  __/|  _ <| |___ / ___ \| |  | | |_) | |___| |___
%|_|   |_| \_\_____/_/   \_\_|  |_|____/|_____|_____|
%
% last updated: 2020-05-21
%%%%%%%%%%%%%%%%%%%%%%%%%%%%%%%%%%%%%%%%%%%%%%%%%%%%%%%%%%%%%%%%%%%%%%%%%%%%%%%%%%%%%%%%%%
%%% PACKAGES
%%%%%%%%%%%%%%%%%%%%%%%%%%%%%%%%%%%%%%%%%%%%%%%%%%%%%%%%%%%%%%%%%%%%%%%%%%%%%%%%%%%%%%%%%%


\pdfminorversion=7						% to prevent errors when building pdf

% some basic packages
\usepackage{amsmath, amsthm, amssymb}
\usepackage{mhchem}						% for chemical symbols
\usepackage{url}						% to format hyperlink text
\usepackage{float}						% for custom figure/table environment
\usepackage{xifthen}					% to handle tests
\usepackage{booktabs}					% for table commands and optimisation
\usepackage{enumitem}					% to format enumerate, itemize, and description
\usepackage{textcomp}					% to support different glyphs
\usepackage{graphicx}					% to support \includegraphics
\usepackage[T1]{fontenc}				% for unicode encoding
\usepackage[utf8]{inputenc}				% for unicode input
\setlength{\headheight}{13.6pt}
\usepackage[top=1.5in,bottom=1in,right=1in,left=1in,headheight=45pt]{geometry}
\usepackage{fancyhdr}					% for adding different headers
\pagestyle{fancy}
% for list of equations
\usepackage{tocloft}					% for custom lists
\usepackage{ragged2e} 					% to undo \centering
\usepackage{hyperref} 					% to make references hyperlinks
\usepackage{glossaries}

% for figures
\usepackage{import}						% for file control
\usepackage{pdfpages}					% for pdf, graphics, and hypertext
\usepackage{transparent}				% for color stack transparency
\usepackage{xcolor}						% for arbitrary color mixing



\author{Jasper Runco}
\date{2020 // Fall}

%%%%%%%%%%%%%%%%%%%%%%%%%%%%%%%%%%%%%%%%%%%%%%%%%%%%%%%%%%%%%%%%%%%%%%%%%%%%%%%%%%%%
% Commands
%%%%%%%%%%%%%%%%%%%%%%%%%%%%%%%%%%%%%%%%%%%%%%%%%%%%%%%%%%%%%%%%%%%%%%%%%%%%%%%%%%%%

% to make a new figure
\newcommand{\incfig}[2][scale=1]{%
	% \def\svgwidth{#1\columnwidth}
	\import{./figures/}{#2.pdf_tex}
}
\pdfsuppresswarningpagegroup=1

% define list of equations
\newcommand{\listequationsname}{\Large{List of Equations}}
\newlistof{myequations}{equ}{\listequationsname}
\newcommand{\myequations}[1]{
	\phantomsection
	\addcontentsline{equ}{myequations}{\protect\numberline{\theequation}#1}
}

\setlength{\cftmyequationsnumwidth}{2.3em}
\setlength{\cftmyequationsindent}{1.5em}

% command to box, label, reference, and include
% noteworthy equations in list of equations
\newcommand{\noteworthy}[2]{
\begin{align} \label{#2} \ensuremath{\boxed{#1}} \end{align}
\myequations{#2} \centering \textit{#2} \justify}

\newtheorem{definition}{Definition}
\newtheorem{theorem}{Theorem}
\newtheorem{lemma}{Lemma}
\newtheorem{corollary}{Corollary}
\newtheorem{example}{Example}
\newtheorem{solution}{Solution}
\newtheorem{constant}{Constant}
\newtheorem{note}{Note}


%%%\begin{document}

% cha
\LARGE\textsc{Date: 2020-08-31} \\ \\ \LARGE\textsc{Announcements:} \\
\small
\textbf{Test: Friday, September 11th, or moved to Monday 14th if still catching up.}

\textbf{Assignment: Sec 1.4 (1-4, 51-58)}


\paragraph \hrule \paragraph \\ \fancyhead[R]{Lesson 6} \fancyhead[L]{Week 3}
%  %  %  %  %  %  %  %  %  %  %  %  %  %  %  %  %  %  %  %  %  %  %  %  %  %  %  %

\subsection{Matrix Arithmetic}%
\label{sub:matrix_arithmetic}




\begin{theorem}[1.4.1 Properties of matrix arithmetic]
	Assuming that the sizes of the matrices are such that the indicated operations can be performed,
	the following rules of matrix arithmetic are valid.
	\begin{itemize}
		\item $A + B = B + A$
		\item $A + ( B + C ) = ( A + B ) + C$
		\item $A(BC) = (AB)C$ \[ A_{m\times n}B_{n\times r}C_{r\times w}\] "Operations must be valid"
		\item $A ( B + C) = AB + AC$
		\item $(B+C)A = BA + CA$
		\item $A(B-C) = AB -AC$
		\item $(B-C)A = BA - CA$
		\item $a(A+B) = aA + aB$
		\item $a(A-B) = aA - aB$
		\item $(a+b)A = aA + bA$
		\item $(a-b)A = aA - bA$
		\item $a(bC) = (ab)C$
		\item $a ( BC) = (aB)C = B(aC)$
	\end{itemize}
\end{theorem}

\subsubsection{How to Verify $a(A+B) = aA + aB$:}%
\label{ssub:verify_}
\begin{enumerate}
	\item Must show that each side produces a matrix of the same size:

let A be an $m\times n$ matrix.

let B be an $m\times n$ matrix

A+B is defined as an $m\times n$ matrix

a(A+B) is defined and is an $m\times n$ matrix

Let A be an $m\times n$ matrix. Let B be an $m\times n$ matrix

$aA$ is an $m\times n$ matrix.

 $aB$ is an $m\times n$ matrix.

 $\boxed{aA + aB \:\text{is defined and is an }\:m\times n \:\text{matrix.}\:}$
\item Show that the corresponding entries of each side are equal:

	$a(A+B)$

	$(a(A+B))_{ij} = a ( a_{ij}+ b_{ij})$ \{ALL SCALARS $\implies$

		$= aa_{ij} + ab_{ij}$

		$= a(A)_{ij} + a (B)_{ij}$

		$= aA + aB$
\end{enumerate}


\begin{example}[Theorem 1.4.1]
	Example to be done outside class. Prove 1.4.1 Theorems. 1.4.1c is the hardest.

	Same thing with 1.4.2
\end{example}

\subsubsection{Inverse of a Matrix}%
\label{ssub:inverse_of_a_matrix}

\begin{definition}[Inverse Matrix]
	If A is a square matrix, and if a matrix B of the same size can be found
	such that $AB = BA = I$,
	then A is said to be invertible (or nonsingular) and B is called the inverse
	of A. If no such B can be found, then A is said to be not invertible or
	singular.
\end{definition}













%  %  %  %  %  %  %  %  %  %  %  %  %  %  %  %  %  %  %  %  %  %  %  %  %  %  %  %
\newpage
%%%\end{document}


%LEAVE EMPTY ROW ABOVE THIS ONE
