%%%
\documentclass[a4paper,11pt,twoside]{report}
%%%
% ____  ____  _____    _    __  __ ____  _     _____
%|  _ \|  _ \| ____|  / \  |  \/  | __ )| |   | ____|
%| |_) | |_) |  _|   / _ \ | |\/| |  _ \| |   |  _|
%|  __/|  _ <| |___ / ___ \| |  | | |_) | |___| |___
%|_|   |_| \_\_____/_/   \_\_|  |_|____/|_____|_____|
%
% last updated: 2020-05-21
%%%%%%%%%%%%%%%%%%%%%%%%%%%%%%%%%%%%%%%%%%%%%%%%%%%%%%%%%%%%%%%%%%%%%%%%%%%%%%%%%%%%%%%%%%
%%% PACKAGES
%%%%%%%%%%%%%%%%%%%%%%%%%%%%%%%%%%%%%%%%%%%%%%%%%%%%%%%%%%%%%%%%%%%%%%%%%%%%%%%%%%%%%%%%%%


\pdfminorversion=7						% to prevent errors when building pdf

% some basic packages
\usepackage{amsmath, amsthm, amssymb}
\usepackage{mhchem}						% for chemical symbols
\usepackage{url}						% to format hyperlink text
\usepackage{float}						% for custom figure/table environment
\usepackage{xifthen}					% to handle tests
\usepackage{booktabs}					% for table commands and optimisation
\usepackage{enumitem}					% to format enumerate, itemize, and description
\usepackage{textcomp}					% to support different glyphs
\usepackage{graphicx}					% to support \includegraphics
\usepackage[T1]{fontenc}				% for unicode encoding
\usepackage[utf8]{inputenc}				% for unicode input
\setlength{\headheight}{13.6pt}
\usepackage[top=1.5in,bottom=1in,right=1in,left=1in,headheight=45pt]{geometry}
\usepackage{fancyhdr}					% for adding different headers
\pagestyle{fancy}
% for list of equations
\usepackage{tocloft}					% for custom lists
\usepackage{ragged2e} 					% to undo \centering
\usepackage{hyperref} 					% to make references hyperlinks
\usepackage{glossaries}

% for figures
\usepackage{import}						% for file control
\usepackage{pdfpages}					% for pdf, graphics, and hypertext
\usepackage{transparent}				% for color stack transparency
\usepackage{xcolor}						% for arbitrary color mixing



\author{Jasper Runco}
\date{2020 // Fall}

%%%%%%%%%%%%%%%%%%%%%%%%%%%%%%%%%%%%%%%%%%%%%%%%%%%%%%%%%%%%%%%%%%%%%%%%%%%%%%%%%%%%
% Commands
%%%%%%%%%%%%%%%%%%%%%%%%%%%%%%%%%%%%%%%%%%%%%%%%%%%%%%%%%%%%%%%%%%%%%%%%%%%%%%%%%%%%

% to make a new figure
\newcommand{\incfig}[2][scale=1]{%
	% \def\svgwidth{#1\columnwidth}
	\import{./figures/}{#2.pdf_tex}
}
\pdfsuppresswarningpagegroup=1

% define list of equations
\newcommand{\listequationsname}{\Large{List of Equations}}
\newlistof{myequations}{equ}{\listequationsname}
\newcommand{\myequations}[1]{
	\phantomsection
	\addcontentsline{equ}{myequations}{\protect\numberline{\theequation}#1}
}

\setlength{\cftmyequationsnumwidth}{2.3em}
\setlength{\cftmyequationsindent}{1.5em}

% command to box, label, reference, and include
% noteworthy equations in list of equations
\newcommand{\noteworthy}[2]{
\begin{align} \label{#2} \ensuremath{\boxed{#1}} \end{align}
\myequations{#2} \centering \textit{#2} \justify}

\newtheorem{definition}{Definition}
\newtheorem{theorem}{Theorem}
\newtheorem{lemma}{Lemma}
\newtheorem{corollary}{Corollary}
\newtheorem{example}{Example}
\newtheorem{solution}{Solution}
\newtheorem{constant}{Constant}
\newtheorem{note}{Note}

%%%
\begin{document}
\fancyhead[R]{2020-09-14} \fancyhead[C]{MAT-253}\fancyhead[L]{Jasper Runco}
\begin{table}[htpb]
	\centering
	\large
	\begin{tabular}{| p{6.6cm} | p{3.9cm} | p{4cm} |}
		\hline
	Assignment: Homework 2 &	Assigned: 12SEP20   & Due: 14SEP20:1600 \\
	\hline
	\end{tabular}
\end{table}
\small
%  %  %  %  %  %  %  %  %  %  %  %  %  %  %  %  %  %  %  %  %  %  %  %  %  %  %  %
\noindent Question 1: Determine conditions on $b_1,b_2,b_3$ such that system is consistent. \\ \hrule

.

.

\[
\begin{bmatrix} 1 & 4 & 7 & b_1 \\2 & 5 & 8 & b_2 \\ -3 & -9 & -15 & b_3 \end{bmatrix}
\]

$-2R_1 + R_2 \to R_2$

$3R_1 + R_3 \to R_3$

\[\begin{bmatrix} 1 & 4 & 7 & b_1 \\ 0 & -3 & -6 & b_2 -2b_1 \\ 0 & 3 & 6 & b_3 + 3b_1 \end{bmatrix}\]

$R_2 + R_3 \to R_3$

\[\begin{bmatrix} 1 & 4 & 7 & b_1 \\ 0 & -3 & -6 & b_2 -2b_1 \\ 0 & 0 & 0 & b_1 + b_2 + b_3   \end{bmatrix}\]


\begin{align*}
\implies& b_1 + b_2 + b_3 = 0 \\
\implies& b_3 = -b_2 - b_3 \\
\end{align*}

\[\implies b=\boxed{\begin{bmatrix} b_1 \\ b_2 \\ -b_2 -b_3 \end{bmatrix} }\]


\noindent Question 2: Find $A^{-1}$ if it exists or explain why it doesn't. \\ \hrule

.

\[\begin{bmatrix} 1 & -2 & 3 & 4 \\ 0 & 3 & 0 & 0 \\ 0 & 5 & 1 & 2 \\ 0 & -1 & 3 & 6 \end{bmatrix} \]

Inverse Algorithm

\[\begin{bmatrix} 1 & -2 & 3 & 4 & | & 1 & 0 & 0 & 0 \\ 0 & 3 & 0 & 0 & | & 0 & 1 & 0 & 0
\\ 0 & 5 & 1 & 2 & | & 0 & 0 & 1 & 0\\ 0 & -1 & 3 & 6 & | & 0 & 0 & 0 & 1\end{bmatrix} \]

$3R_4 + R_2 \to R_2$

\[\begin{bmatrix} 1 & -2 & 3 & 4 & | & 1 & 0 & 0 & 0 \\ 0 & 0 & 9 & 18 & | & 0 & 1 & 0 & 3
\\ 0 & 5 & 1 & 2 & | & 0 & 0 & 1 & 0\\ 0 & -1 & 3 & 6 & | & 0 & 0 & 0 & 1\end{bmatrix} \]

$\frac{1}{9}R_2 \to R_2$

$R_2 \leftrightarrow R_3$

\[\begin{bmatrix} 1 & -2 & 3 & 4 & | & 1 & 0 & 0 & 0 \\ 0 & 5 & 1 & 2 & | & 0 & 0 & 1 & 0\\
0 & 0 & 1 & 2 & | & 0 & \frac{1}{9} & 0 & \frac{1}{3} \\0 & -1 & 3 & 6 & | & 0 & 0 & 0 & 1\end{bmatrix} \]

$4R_4 + R_2 \to R_2$

\[\begin{bmatrix} 1 & -2 & 3 & 4 & | & 1 & 0 & 0 & 0 \\ 0 & 1 & 13 & 26 & | & 0 & 0 & 1 & 4\\
0 & 0 & 1 & 2 & | & 0 & \frac{1}{9} & 0 & \frac{1}{3} \\0 & -1 & 3 & 6 & | & 0 & 0 & 0 & 1\end{bmatrix} \]

$R_2 + R_4 \to R_4$

\[\begin{bmatrix} 1 & -2 & 3 & 4 & | & 1 & 0 & 0 & 0 \\ 0 & 1 & 13 & 26 & | & 0 & 0 & 1 & 4\\
0 & 0 & 1 & 2 & | & 0 & \frac{1}{9} & 0 & \frac{1}{3} \\0 & 0 & 16 & 32 & | & 0 & 0 & 1 & 5\end{bmatrix} \]

$-16R_3 + R_4 \to R_4$

\[\begin{bmatrix} 1 & -2 & 3 & 4 & | & 1 & 0 & 0 & 0 \\ 0 & 1 & 13 & 26 & | & 0 & 0 & 1 & 4\\
0 & 0 & 1 & 2 & | & 0 & \frac{1}{9} & 0 & \frac{1}{3} \\0 & 0 & 0 & 0 & | & 0 & 0 & -16 & -\frac{1}{3}\end{bmatrix} \]

Main diagonal contains constant of zero $\implies$ non-trivial solution.

Given $A$ is a $4\times 4$ matrix, Equivalent Statements Theorem gives $\boxed{\:\text{A is not invertible}\:}$

$\qed$


\noindent Question 3: Find $A^{-1}$ if possible, otherwise solve another way. \\ \hrule
\begin{align*}
	x + 2z &= 6 \\
	-x + 2y + 3z &=  -5 \\
	x - y &= 6 \\
\end{align*}

\[\begin{bmatrix} 1 & 0 & 2 \\ -1 & 2 & 3 \\ 1 & -1 & 0 \end{bmatrix}
\begin{bmatrix} x \\ y \\ z \end{bmatrix}
\begin{bmatrix} 6 \\ -5 \\ 6 \end{bmatrix} \]

If $A^{-1}$ exists, $A^{-1}Ax = A^{-1}b$
\[\begin{bmatrix}  1 & 0 & 2 & | & 1 & 0 & 0 \\ -1 & 2 & 3 & | & 0 & 1 & 0 \\ 1 & -1 & 0 & | & 0 & 0 & 1  \end{bmatrix} \]

$R_1 + R_2 \to R_2$

$-R_1 + R_3 \to R_3$

\[\begin{bmatrix}  1 & 0 & 2 & | & 1 & 0 & 0 \\ 0 & 2 & 5 & | & 1 & 1 & 0 \\ 0 & -1 & -2 & | & -1 & 0 & 1  \end{bmatrix} \]

$R_2 \leftrightarrow R_3$

\[\begin{bmatrix}  1 & 0 & 2 & | & 1 & 0 & 0  \\ 0 & -1 & -2 & | & -1 & 0 & 1 \\ 0 & 2 & 5 & | & 1 & 1 & 0 \end{bmatrix} \]

$-R_2 \to R_2$

\[\begin{bmatrix}  1 & 0 & 2 & | & 1 & 0 & 0  \\ 0 & 1 & 2 & | & 1 & 0 & -1 \\ 0 & 2 & 5 & | & 1 & 1 & 0 \end{bmatrix} \]

$-2R_2 + R_3 \to R_3$

\[\begin{bmatrix}  1 & 0 & 2 & | & 1 & 0 & 0  \\ 0 & 1 & 2 & | & 1 & 0 & -1 \\ 0 & 0 & 1 & | & -1 & 1 & 2 \end{bmatrix} \]

$-2R_3 + R_1 \to R_1$

$-2R_3 + R_2 \to R_2$

\[\begin{bmatrix}  1 & 0 & 0 & | & 3 & -2 & -4  \\ 0 & 1 & 0 & | & 3 & -2 & -5 \\ 0 & 0 & 1 & | & -1 & 1 & 2 \end{bmatrix} \]

\[A^{-1} = \boxed{\begin{bmatrix}   3 & -2 & -4  \\  3 & -2 & -5 \\  -1 & 1 & 2 \end{bmatrix}} \]

Check:

\[ \begin{bmatrix}   3 & -2 & -4  \\  3 & -2 & -5 \\  -1 & 1 & 2 \end{bmatrix}
\begin{bmatrix} 1 & 0 & 2 \\ -1 & 2 & 3 \\ 1 & -1 & 0 \end{bmatrix}
\begin{bmatrix} x \\ y \\ z \end{bmatrix} =
 \begin{bmatrix}   3 & -2 & -4  \\  3 & -2 & -5 \\  -1 & 1 & 2 \end{bmatrix}
\begin{bmatrix} 6 \\ -5 \\ 6 \end{bmatrix} \]

\[\begin{bmatrix} 1 & 0 & 0 \\ 0 & 1 & 0 \\ 0 & 0 & 1 \end{bmatrix}
\begin{bmatrix} x \\ y \\ z \end{bmatrix} =
\begin{bmatrix} 4 \\ -2 \\ 1 \end{bmatrix} \]

\[\boxed{\left( 4, -2, 1 \right) }\]



\noindent Question 4: Determine X. \\ \hrule

\[\begin{bmatrix} 2 & -2 \\ 1 & 0 \end{bmatrix} X = \begin{bmatrix} 5 & -3 & 6 \\ 1 & 0 & 9 \end{bmatrix} \]

Matrix Product:

$A_{m\times r} B_{r\times n} = C_{m\times n}$

$A_{2\times 2} B_{2\times 3} = C_{2\times 3}$

$A^{-1}A = I$

\[\begin{bmatrix} 2 & -2 \\ 1 & 0 \end{bmatrix}^{-1} \begin{bmatrix} 2 & -2 \\ 1 & 0 \end{bmatrix} X = \begin{bmatrix} 2 & -2 \\ 1 & 0 \end{bmatrix}^{-1}\begin{bmatrix} 5 & -3 & 6 \\ 1 & 0 & 9 \end{bmatrix} \]
\[	\begin{bmatrix} 2 & -2 & | & 1 & 0 \\ 1 & 0 & | & 0 & 1\end{bmatrix}	\]

$R_1 \leftrightarrow R_2$

$-2R_1 + R_2 \to R_2$

$-\frac{1}{2}R_2 \to R_2$

\[	\begin{bmatrix}  1 & 0 & | & 0 & 1 \\ 0 & 1 & | & -\frac{1}{2} & 1 \end{bmatrix}	\]

\[X = \begin{bmatrix} 0 & 1 \\ -\frac{1}{2} & 1 \end{bmatrix} \begin{bmatrix} 5 & -3 & 6 \\ 1 & 0 & 9 \end{bmatrix}  \]

\[\boxed{X =\begin{bmatrix} 1 & 0 & 9 \\ -\frac{3}{2} & \frac{3}{2} & 6 \end{bmatrix} }\]

Check:

\[\begin{bmatrix} 2 & -2 \\ 1 & 0 \end{bmatrix} \begin{bmatrix} 1 & 0 & 9 \\ -\frac{3}{2} & \frac{3}{2} & 6 \end{bmatrix} = \begin{bmatrix} 5 & -3 & 6 \\ 1 & 0 & 9 \end{bmatrix} \]

\noindent Question 5a: If symmetric, find a, b, c, and d . \\ \hrule

\[\begin{bmatrix} a & 2a-b+c & 2a -b -4c \\ 4 & b & a-b+2c \\ -16 & -6 & c \end{bmatrix} \]

\begin{align*}
	4 &= 2a - b + c \\
	-16 &=  2a - b -4c \\
	-6 &=  a - b + 2c \\
\end{align*}

\[\begin{bmatrix} 2 & -1 & 1 & 4 \\ 2 & -1 & -4 & -16 \\ 1 & -1 & 2 & -6 \end{bmatrix} \]

$R_3 \leftrightarrow R_1$

\[\begin{bmatrix} 1 & -1 & 2 & -6 \\ 2 & -1 & -4 & -16 \\ 2 & -1 & 1 & 4 \end{bmatrix} \]

$-2R_1 + R_2 \to R_2$

$-2R_1 + R_3 \to R_3$

\[\begin{bmatrix} 1 & -1 & 2 & -6 \\ 0 & 1 & -8 & -4 \\ 0 & 1 & -3 & 16 \end{bmatrix} \]

$-R_2 + R_3 \to R_3$

$\frac{1}{5}R_3 \to R_3$

\[\begin{bmatrix} 1 & -1 & 2 & -6 \\ 0 & 1 & -8 & -4 \\ 0 & 0 & 1 & 4 \end{bmatrix} \]

$8R_3 + R_2 \to R_2$

$-2R_3 + R_1 \to R_1$

\[\begin{bmatrix} 1 & -1 & 0 & -14 \\ 0 & 1 & 0 & 28 \\ 0 & 0 & 1 & 4 \end{bmatrix} \]

$R_2 + R_1 \to R_1$

\[\begin{bmatrix} 1 & 0 & 0 & 14 \\ 0 & 1 & 0 & 28 \\ 0 & 0 & 1 & 4 \end{bmatrix} \]

\[\boxed{(a,b,c) = (14,28,4)}\]


\noindent Question 5b: If lower triangle, find a, b, c, and d . \\ \hrule

\[\begin{bmatrix} a & 2a-b+c & 2a -b -4c \\ 4 & b & a-b+2c \\ -16 & -6 & c \end{bmatrix} \]

\[0 = 2a - b + c = 2a - b -4c = a - b + 2c\]

\begin{align*}
	a &=  b - 2c\\
	2a &=  b + 4c \\
	2a &= b -c \\
\end{align*}

\[b-c = b + 4c\]

\[\implies \boxed{a = b = c = 0}\]





%  %  %  %  %  %  %  %  %  %  %  %  %  %  %  %  %  %  %  %  %  %  %  %  %  %  %  %
\newpage
%%%
\end{document}

%LEAVE EMPTY ROW ABOVE THIS ONE
