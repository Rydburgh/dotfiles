%%%\documentclass[a4paper,11pt,twoside]{report}

%%%% ____  ____  _____    _    __  __ ____  _     _____
%|  _ \|  _ \| ____|  / \  |  \/  | __ )| |   | ____|
%| |_) | |_) |  _|   / _ \ | |\/| |  _ \| |   |  _|
%|  __/|  _ <| |___ / ___ \| |  | | |_) | |___| |___
%|_|   |_| \_\_____/_/   \_\_|  |_|____/|_____|_____|
%
% last updated: 2020-05-21
%%%%%%%%%%%%%%%%%%%%%%%%%%%%%%%%%%%%%%%%%%%%%%%%%%%%%%%%%%%%%%%%%%%%%%%%%%%%%%%%%%%%%%%%%%
%%% PACKAGES
%%%%%%%%%%%%%%%%%%%%%%%%%%%%%%%%%%%%%%%%%%%%%%%%%%%%%%%%%%%%%%%%%%%%%%%%%%%%%%%%%%%%%%%%%%


\pdfminorversion=7						% to prevent errors when building pdf

% some basic packages
\usepackage{amsmath, amsthm, amssymb}
\usepackage{mhchem}						% for chemical symbols
\usepackage{url}						% to format hyperlink text
\usepackage{float}						% for custom figure/table environment
\usepackage{xifthen}					% to handle tests
\usepackage{booktabs}					% for table commands and optimisation
\usepackage{enumitem}					% to format enumerate, itemize, and description
\usepackage{textcomp}					% to support different glyphs
\usepackage{graphicx}					% to support \includegraphics
\usepackage[T1]{fontenc}				% for unicode encoding
\usepackage[utf8]{inputenc}				% for unicode input
\setlength{\headheight}{13.6pt}
\usepackage[top=1.5in,bottom=1in,right=1in,left=1in,headheight=45pt]{geometry}
\usepackage{fancyhdr}					% for adding different headers
\pagestyle{fancy}
% for list of equations
\usepackage{tocloft}					% for custom lists
\usepackage{ragged2e} 					% to undo \centering
\usepackage{hyperref} 					% to make references hyperlinks
\usepackage{glossaries}

% for figures
\usepackage{import}						% for file control
\usepackage{pdfpages}					% for pdf, graphics, and hypertext
\usepackage{transparent}				% for color stack transparency
\usepackage{xcolor}						% for arbitrary color mixing



\author{Jasper Runco}
\date{2020 // Fall}

%%%%%%%%%%%%%%%%%%%%%%%%%%%%%%%%%%%%%%%%%%%%%%%%%%%%%%%%%%%%%%%%%%%%%%%%%%%%%%%%%%%%
% Commands
%%%%%%%%%%%%%%%%%%%%%%%%%%%%%%%%%%%%%%%%%%%%%%%%%%%%%%%%%%%%%%%%%%%%%%%%%%%%%%%%%%%%

% to make a new figure
\newcommand{\incfig}[2][scale=1]{%
	% \def\svgwidth{#1\columnwidth}
	\import{./figures/}{#2.pdf_tex}
}
\pdfsuppresswarningpagegroup=1

% define list of equations
\newcommand{\listequationsname}{\Large{List of Equations}}
\newlistof{myequations}{equ}{\listequationsname}
\newcommand{\myequations}[1]{
	\phantomsection
	\addcontentsline{equ}{myequations}{\protect\numberline{\theequation}#1}
}

\setlength{\cftmyequationsnumwidth}{2.3em}
\setlength{\cftmyequationsindent}{1.5em}

% command to box, label, reference, and include
% noteworthy equations in list of equations
\newcommand{\noteworthy}[2]{
\begin{align} \label{#2} \ensuremath{\boxed{#1}} \end{align}
\myequations{#2} \centering \textit{#2} \justify}

\newtheorem{definition}{Definition}
\newtheorem{theorem}{Theorem}
\newtheorem{lemma}{Lemma}
\newtheorem{corollary}{Corollary}
\newtheorem{example}{Example}
\newtheorem{solution}{Solution}
\newtheorem{constant}{Constant}
\newtheorem{note}{Note}


%%%\begin{document}


\LARGE\textsc{Date: 2020-09-23} \\ \\ \LARGE\textsc{Announcements:} \\
\small

\textbf{Oct2 , Friday, Test CH2: 2.1, 2.2, 2.3}

\textbf{HW: 2.2 1-21 Odd}


\paragraph \hrule \paragraph \\ \fancyhead[R]{Lesson 15} \fancyhead[L]{Week 6}
%  %  %  %  %  %  %  %  %  %  %  %  %  %  %  %  %  %  %  %  %  %  %  %  %  %  %  %

\begin{example}[Determinant of 4x4]
\[det(A) = |A|\]
	\[A = \begin{bmatrix} 1 & 2 & 0 & 1 \\
	2 & 0 & -1 & 5 \\ 3 & 0 & 1 & 4 \\
 	0 & 1 & 2 & -1\end{bmatrix} \]
	\[\begin{bmatrix} + & - & & & \\ & + & & \\ & - & & \\ & + & & \end{bmatrix} \]

	\[det(a) = -2\begin{bmatrix} 2 & -1 & 5 \\ 3 & 1 & 4 \\ 0 & 2 & -1 \end{bmatrix} + 0 = 0 +
	1 \begin{bmatrix} 1 & 0 & 1 \\ 2 & -1 & 5 \\ 3 & 1 & 4 \end{bmatrix} \]
	\[ = -2\left[ 0 -2(8 - 15) -1(2+3)  \right] 1\left[   1(-4-5) -0 +1(2 +3) \right] \]
	\[= -2(14-5) + (-9+5) = -22\]
\smallskip\hfill$\bullet$\end{example}

\begin{theorem}[Determinant of triangular matrix]
	If $A$ is an $n\times n$ triangular matrix (upper, lower, diagonal), then the $det(A)$ is the
	product of the entries on the main diagonal.
	\[detA = a_{11}a_{22}a_{33}\ldots a_{nn}\]
\smallskip\hfill$\bullet$\end{theorem}

\section{Evaluating Determinants by Row Reduction}%
\label{sec:evaluating_determinants_by_row_reduction}

\begin{theorem}[]
	Let $A$ be an $n\times n$ matrix. If $A$ has a row of zeros, then $detA = 0$
\smallskip\hfill$\bullet$\end{theorem}

\begin{theorem}[]
	Let $A$ be an $n\times n$ matrix. Then $detA = det(A^{T})$
\smallskip\hfill$\bullet$\end{theorem}

\begin{theorem}[]
	Let $A$ be an $n\times n$ matrix.
	\begin{enumerate}
		\item If $B$ is the matrix that results when a single row or single column of
			$A$ is multiplied by a scalar $k$, then
			\[ detB = k detA.\]
			\item If $B$ is the matrix that results when two rows or two columns
				of $A$ are interchanged, then
			\[ detB = - detA.\]
			\item If $B$ is the matrix obtained when a multiple of one row or
				one column is added to another, then
				\[det B = det A\]
	\end{enumerate}
\smallskip\hfill$\bullet$\end{theorem}

\begin{example}[]
	\[A = \begin{bmatrix} 1 & 2 & 3 & 4  \\ 3 & 6 & 7 & 1 \\
	4 & 8 & 9 & 2 \\ 5 & 3 & 1 & 7\end{bmatrix} \]
	find $detA$
\smallskip\hfill$\bullet$\end{example}







%  %  %  %  %  %  %  %  %  %  %  %  %  %  %  %  %  %  %  %  %  %  %  %  %  %  %  %
\newpage
%%%\end{document}


%LEAVE EMPTY ROW ABOVE THIS ONE
