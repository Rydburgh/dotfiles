%%%\documentclass[a4paper,11pt,twoside]{report}

%%%%| |_) | |_) |  _|   / _ \ | |\/| |  _ \| |   |  _|
%|  __/|  _ <| |___ / ___ \| |  | | |_) | |___| |___
%|_|   |_| \_\_____/_/   \_\_|  |_|____/|_____|_____|
% Jasper Runco
% last updated: 2020-05-20
%%%%%%%%%%%%%%%%%%%%%%%%%%%%%%%%%%%%%%%%%%%%%%%%%%%%%%%%%%%%%%%%%%%%%%%%%%%%%%%%%%%%%%%%%%
%%% PACKAGES
%%%%%%%%%%%%%%%%%%%%%%%%%%%%%%%%%%%%%%%%%%%%%%%%%%%%%%%%%%%%%%%%%%%%%%%%%%%%%%%%%%%%%%%%%%
\pdfminorversion=7						% to prevent errors when building pdf
% some basic packages
\usepackage{amsthm, amsmath}
\usepackage{url}						% to format hyperlink text
\usepackage{float}						% for custom figure/table environment
\usepackage{xifthen}					% to handle tests
\usepackage{booktabs}					% for table commands and optimisation
\usepackage{enumitem}					% to format enumerate, itemize, and description
\usepackage{textcomp}					% to support different glyphs
\usepackage{graphicx}					% to support \includegraphics
\usepackage[T1]{fontenc}				% for unicode encoding
\usepackage[utf8]{inputenc}				% for unicode input
\setlength{\headheight}{13.6pt}
\usepackage[top=1.5in,bottom=1in,right=1in,left=1in,headheight=45pt]{geometry}
\usepackage{fancyhdr}					% for adding different headers
\pagestyle{fancy}
% for list of equations
\usepackage{tocloft}					% for custom lists
\usepackage{ragged2e} 					% to undo \centering
\usepackage{hyperref} 					% to make references hyperlinks
\usepackage{glossaries}
% for figures
\usepackage{import}						% for file control
\usepackage{pdfpages}					% for pdf, graphics, and hypertext
\usepackage{transparent}				% for color stack transparency
\usepackage{xcolor}						% for arbitrary color mixing
%%%%%%%%%%%%%%%%%%%%%%%%%%%%%%%%%%%%%%%%%%%%%%%%%%%%%%%%%%%%%%%%%%%%%%%%%%%%%%%%%%%%
% Commands
%%%%%%%%%%%%%%%%%%%%%%%%%%%%%%%%%%%%%%%%%%%%%%%%%%%%%%%%%%%%%%%%%%%%%%%%%%%%%%%%%%%%
% to make a new figure
\newcommand{\incfig}[2][1]{%
	\def\svgwidth{#1\columnwidth}
	\import{./figures/}{#2.pdf_tex}
}
\pdfsuppresswarningpagegroup=1
% define list of equations
\newcommand{\listequationsname}{\Large{List of Equations}}
\newlistof{myequations}{equ}{\listequationsname}
\newcommand{\myequations}[1]{
	\addcontentsline{equ}{myequations}{\protect\numberline{\theequation}#1}
}
\setlength{\cftmyequationsnumwidth}{2.3em}
\setlength{\cftmyequationsindent}{1.5em}
% command to box, label, refference, and include
% noteworthy equations in list of equations
\newcommand{\noteworthy}[2]{
\begin{align} \label{#2} \ensuremath{\boxed{#1}} \end{align}
\myequations{#2} \centering \small \textit{#2} \normalsize \justify }
%%%%%%%%%%%%%%%%%%%%%%%%%%%%%%%%%%%%%%%%%%%%%%%%%%%%%%%%%%%%%%%%%%%%%%%%%%%%%%%%%%%%
% Theorems
%%%%%%%%%%%%%%%%%%%%%%%%%%%%%%%%%%%%%%%%%%%%%%%%%%%%%%%%%%%%%%%%%%%%%%%%%%%%%%%%%%%%
\newtheorem{definition}{Definition}
\newtheorem{theorem}{Theorem}
\newtheorem{lemma}{Lemma}
\newtheorem{corollary}{Corollary}


%%%\begin{document}

\chapter{Determinants}%
\label{cha:determinants}


\LARGE\textsc{Date: 2020-09-18} \\ \\ \LARGE\textsc{Announcements:} \\
\small



\paragraph \hrule \paragraph \\ \fancyhead[R]{Lesson 13} \fancyhead[L]{Week 5}
%  %  %  %  %  %  %  %  %  %  %  %  %  %  %  %  %  %  %  %  %  %  %  %  %  %  %  %

Consider the set $\{  j_1,j_2,\ldots jn \} $ where each $j_{i}$ is a positive integer.

Each possible ordering of the set (standard order) is called a permutation set

	A set with n elements has $n!$ permutation sets.

	In each permutation set, define a number $\alpha_{i}$ for each number.

$ \alpha_{i} = $ the entries following $j_{i}$ that are less than $j_{i}$.

\begin{example}[set $\{1,2,3\} $]
	permutations:
	\[\begin{matrix}
	1, & 2, &3 \\
	1, & 3, &2 \\
	2, & 1, &3 \\
	2, & 3, &1 \\
	3, & 1, &2 \\
	3, & 2, &1 \\
\end{matrix} \]
	\[\begin{matrix}
	\alpha_1 = 0, & \alpha_2 = 0, & \alpha_3 = 0 \\
	\alpha_1 = 0, & \alpha_2 = 1, & \alpha_3 = 0 \\
	\alpha_1 = 1, & \alpha_2 = 0, & \alpha_3 = 0 \\
	\alpha_1 = 1, & \alpha_2 = 1, & \alpha_3 = 0 \\
	\alpha_1 = 2, & \alpha_2 = 0, & \alpha_3 = 0 \\
	\alpha_1 = 2, & \alpha_2 = 1, & \alpha_3 = 0 \\
\end{matrix} \]
\smallskip\hfill$\bullet$\end{example}

$\alpha_1 + \alpha_2 + \ldots + \alpha_{n}$ = total number of inversions in permutation set $=$ parity

Parity is either even or odd.

Define a function

\[\delta(j_1,j_2,\ldots j_{n}) = \left\{ 1 , -1 \right\}  \]

Form all possible products of the form $a_{1j_{1}} a_{2j_{2}} a_{2j_{3}} \ldots a_{nj_{n}}$.

Determinant of $A = det(A)$
\[= \sum_{\:\text{all possible permutation sets}\:} \delta(j_1,j_2,\ldots,jn)a_{1j_{1}} a_{2j_{2}} a_{2j_{3}} \ldots a_{nj_{n}}\]







%  %  %  %  %  %  %  %  %  %  %  %  %  %  %  %  %  %  %  %  %  %  %  %  %  %  %  %
\newpage
%%%\end{document}


%LEAVE EMPTY ROW ABOVE THIS ONE
