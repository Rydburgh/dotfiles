%%%\documentclass[a4paper,11pt,twoside]{report}

%%%%| |_) | |_) |  _|   / _ \ | |\/| |  _ \| |   |  _|
%|  __/|  _ <| |___ / ___ \| |  | | |_) | |___| |___
%|_|   |_| \_\_____/_/   \_\_|  |_|____/|_____|_____|
% Jasper Runco
% last updated: 2020-05-20
%%%%%%%%%%%%%%%%%%%%%%%%%%%%%%%%%%%%%%%%%%%%%%%%%%%%%%%%%%%%%%%%%%%%%%%%%%%%%%%%%%%%%%%%%%
%%% PACKAGES
%%%%%%%%%%%%%%%%%%%%%%%%%%%%%%%%%%%%%%%%%%%%%%%%%%%%%%%%%%%%%%%%%%%%%%%%%%%%%%%%%%%%%%%%%%
\pdfminorversion=7						% to prevent errors when building pdf
% some basic packages
\usepackage{amsthm, amsmath}
\usepackage{url}						% to format hyperlink text
\usepackage{float}						% for custom figure/table environment
\usepackage{xifthen}					% to handle tests
\usepackage{booktabs}					% for table commands and optimisation
\usepackage{enumitem}					% to format enumerate, itemize, and description
\usepackage{textcomp}					% to support different glyphs
\usepackage{graphicx}					% to support \includegraphics
\usepackage[T1]{fontenc}				% for unicode encoding
\usepackage[utf8]{inputenc}				% for unicode input
\setlength{\headheight}{13.6pt}
\usepackage[top=1.5in,bottom=1in,right=1in,left=1in,headheight=45pt]{geometry}
\usepackage{fancyhdr}					% for adding different headers
\pagestyle{fancy}
% for list of equations
\usepackage{tocloft}					% for custom lists
\usepackage{ragged2e} 					% to undo \centering
\usepackage{hyperref} 					% to make references hyperlinks
\usepackage{glossaries}
% for figures
\usepackage{import}						% for file control
\usepackage{pdfpages}					% for pdf, graphics, and hypertext
\usepackage{transparent}				% for color stack transparency
\usepackage{xcolor}						% for arbitrary color mixing
%%%%%%%%%%%%%%%%%%%%%%%%%%%%%%%%%%%%%%%%%%%%%%%%%%%%%%%%%%%%%%%%%%%%%%%%%%%%%%%%%%%%
% Commands
%%%%%%%%%%%%%%%%%%%%%%%%%%%%%%%%%%%%%%%%%%%%%%%%%%%%%%%%%%%%%%%%%%%%%%%%%%%%%%%%%%%%
% to make a new figure
\newcommand{\incfig}[2][1]{%
	\def\svgwidth{#1\columnwidth}
	\import{./figures/}{#2.pdf_tex}
}
\pdfsuppresswarningpagegroup=1
% define list of equations
\newcommand{\listequationsname}{\Large{List of Equations}}
\newlistof{myequations}{equ}{\listequationsname}
\newcommand{\myequations}[1]{
	\addcontentsline{equ}{myequations}{\protect\numberline{\theequation}#1}
}
\setlength{\cftmyequationsnumwidth}{2.3em}
\setlength{\cftmyequationsindent}{1.5em}
% command to box, label, refference, and include
% noteworthy equations in list of equations
\newcommand{\noteworthy}[2]{
\begin{align} \label{#2} \ensuremath{\boxed{#1}} \end{align}
\myequations{#2} \centering \small \textit{#2} \normalsize \justify }
%%%%%%%%%%%%%%%%%%%%%%%%%%%%%%%%%%%%%%%%%%%%%%%%%%%%%%%%%%%%%%%%%%%%%%%%%%%%%%%%%%%%
% Theorems
%%%%%%%%%%%%%%%%%%%%%%%%%%%%%%%%%%%%%%%%%%%%%%%%%%%%%%%%%%%%%%%%%%%%%%%%%%%%%%%%%%%%
\newtheorem{definition}{Definition}
\newtheorem{theorem}{Theorem}
\newtheorem{lemma}{Lemma}
\newtheorem{corollary}{Corollary}


%%%\begin{document}

% cha
\LARGE\textsc{Date: 2020-09-04} \\ \\ \LARGE\textsc{Announcements:} \\
\small
\textbf{Assignment: section 1.5 (1-20)}

Watch weekend video.



\paragraph \hrule \paragraph \\ \fancyhead[R]{Lesson 8} \fancyhead[L]{Week 3}
%  %  %  %  %  %  %  %  %  %  %  %  %  %  %  %  %  %  %  %  %  %  %  %  %  %  %  %

\section{Elementary Matrices and Methods for Finding $A^{-1}$}%
\label{sec:elementary_matrices_and_methods_for_finding_a_1_}



\begin{definition}[Row Equivalent]
	Matrices A and B are called Row Equivalent if either is obtained
	from the other by a sequence of row operations.
\end{definition}

\begin{definition}[Elementary Matrix]
	A matrix E is called an elementary matrix if it can be obtained
	from the identity matrix by performing one row operation.
\end{definition}

\[\begin{bmatrix} 1 & 0 & 0 \\ 0 & 1 & 0 \\ 0 & 0 & 1 \end{bmatrix} \]

\[E = \begin{bmatrix} 1 & 0 & 0 \\ 0 & 2 & 0 \\ 0 & 0 & 1 \end{bmatrix} \]

The product $EA$, the result is the same as performing the elementary
row operation on a yourself.
\[A = \begin{bmatrix}  1 & 2 & 3 \\ 4 & 5 & 6 \\ 7 & 8 &  9 \end{bmatrix} \]
\[2R_2\to R_2\]
\[A = \begin{bmatrix}  1 & 2 & 3 \\ 8 & 10 & 12 \\ 7 & 8 &  9 \end{bmatrix} \]
Is the same as:
\[\begin{bmatrix} 1 & 0 & 0 \\ 0 & 2 & 0 \\ 0 & 0 & 1 \end{bmatrix} \begin{bmatrix}  1 & 2 & 3 \\ 4 & 5 & 6 \\ 7 & 8 &  9 \end{bmatrix} \]

\begin{theorem}[]
	Every elementary matrix is invertible and its inverse is also an
	elementary matrix.
\end{theorem}

\subsubsection{Equivalent Statements Theorem}%
\label{ssub:equivalent_statements}

\begin{theorem}[]
If A is an $n\times n$ matrix, then the following are equivalent, that is all
are true or all are false,
\begin{enumerate}[label=\Alph*]
	\item A is invertible.
	\item $Ax = 0$ has only the solution $c\begin{bmatrix} 0 \\ 0 \\ \ldots \\ 0  \end{bmatrix} $ (trivial solution).
		\item The reduced row echelon form of A is $I_{n}$.
			\item A is expressible as a product of elementary matrices.
\end{enumerate}
\end{theorem}

\begin{proof}[Proof: equivalent statements]

	\begin{itemize}
		\item Assume A true $\to$ B true $\to$ C true $\to$ D true $\to$ A true
	\item Assume A is invertible $\to$ $A^{-1}$ exists and $A A^{-1} = A^{-1}A = I_{n}$
	\item $Ax = 0$
	\item $(A^{-1}Ax = A^{-1}0 \to Ix = \begin{bmatrix}  0 \\ 0 \\ .. \\ 0 \end{bmatrix} $
	\item $\begin{bmatrix} a_{11} & a_{12} & \ldots & a_{1n} & | & 0 \\a_{21} & a_{22} & \ldots & a_{2n} & | & 0 \\a_{n1} & a_{n2} & \ldots & a_{nn} & | & 0 \\ \end{bmatrix} $
	$\begin{bmatrix} x_1 &  &  &  & | & 0 \\ & x_2 &  &  & | & 0 \\ &  & x_3 &  & | & 0 \\ \end{bmatrix} $

	\item $[A|0]$, perform row operations and get $[I | 0]$
		\[E_{n} \ldots E_2E_1 A = I\]
		\[E_1^{-1}E_2^{-1}\ldots E_{n-1}^{-1}E_{n}^{-1}E_{n}\ldots E_2E_1A = E_1^{-1}E_2^{-1}\ldots E_{n-1}^{-1} E_{n}^{-1} I\]

	\item (theorem) If A and B are invertible then $(AB)$ is invertible $\to$ \[(AB)^{-1} = B^{-1}A^{-1}\]
		\item $\to$ A is true.
	\end{itemize}




\end{proof}




%  %  %  %  %  %  %  %  %  %  %  %  %  %  %  %  %  %  %  %  %  %  %  %  %  %  %  %
\newpage
%%%\end{document}


%LEAVE EMPTY ROW ABOVE THIS ONE
