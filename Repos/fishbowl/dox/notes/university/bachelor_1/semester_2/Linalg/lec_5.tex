%%%\documentclass[a4paper,11pt,twoside]{report}

%%%%| |_) | |_) |  _|   / _ \ | |\/| |  _ \| |   |  _|
%|  __/|  _ <| |___ / ___ \| |  | | |_) | |___| |___
%|_|   |_| \_\_____/_/   \_\_|  |_|____/|_____|_____|
% Jasper Runco
% last updated: 2020-05-20
%%%%%%%%%%%%%%%%%%%%%%%%%%%%%%%%%%%%%%%%%%%%%%%%%%%%%%%%%%%%%%%%%%%%%%%%%%%%%%%%%%%%%%%%%%
%%% PACKAGES
%%%%%%%%%%%%%%%%%%%%%%%%%%%%%%%%%%%%%%%%%%%%%%%%%%%%%%%%%%%%%%%%%%%%%%%%%%%%%%%%%%%%%%%%%%
\pdfminorversion=7						% to prevent errors when building pdf
% some basic packages
\usepackage{amsthm, amsmath}
\usepackage{url}						% to format hyperlink text
\usepackage{float}						% for custom figure/table environment
\usepackage{xifthen}					% to handle tests
\usepackage{booktabs}					% for table commands and optimisation
\usepackage{enumitem}					% to format enumerate, itemize, and description
\usepackage{textcomp}					% to support different glyphs
\usepackage{graphicx}					% to support \includegraphics
\usepackage[T1]{fontenc}				% for unicode encoding
\usepackage[utf8]{inputenc}				% for unicode input
\setlength{\headheight}{13.6pt}
\usepackage[top=1.5in,bottom=1in,right=1in,left=1in,headheight=45pt]{geometry}
\usepackage{fancyhdr}					% for adding different headers
\pagestyle{fancy}
% for list of equations
\usepackage{tocloft}					% for custom lists
\usepackage{ragged2e} 					% to undo \centering
\usepackage{hyperref} 					% to make references hyperlinks
\usepackage{glossaries}
% for figures
\usepackage{import}						% for file control
\usepackage{pdfpages}					% for pdf, graphics, and hypertext
\usepackage{transparent}				% for color stack transparency
\usepackage{xcolor}						% for arbitrary color mixing
%%%%%%%%%%%%%%%%%%%%%%%%%%%%%%%%%%%%%%%%%%%%%%%%%%%%%%%%%%%%%%%%%%%%%%%%%%%%%%%%%%%%
% Commands
%%%%%%%%%%%%%%%%%%%%%%%%%%%%%%%%%%%%%%%%%%%%%%%%%%%%%%%%%%%%%%%%%%%%%%%%%%%%%%%%%%%%
% to make a new figure
\newcommand{\incfig}[2][1]{%
	\def\svgwidth{#1\columnwidth}
	\import{./figures/}{#2.pdf_tex}
}
\pdfsuppresswarningpagegroup=1
% define list of equations
\newcommand{\listequationsname}{\Large{List of Equations}}
\newlistof{myequations}{equ}{\listequationsname}
\newcommand{\myequations}[1]{
	\addcontentsline{equ}{myequations}{\protect\numberline{\theequation}#1}
}
\setlength{\cftmyequationsnumwidth}{2.3em}
\setlength{\cftmyequationsindent}{1.5em}
% command to box, label, refference, and include
% noteworthy equations in list of equations
\newcommand{\noteworthy}[2]{
\begin{align} \label{#2} \ensuremath{\boxed{#1}} \end{align}
\myequations{#2} \centering \small \textit{#2} \normalsize \justify }
%%%%%%%%%%%%%%%%%%%%%%%%%%%%%%%%%%%%%%%%%%%%%%%%%%%%%%%%%%%%%%%%%%%%%%%%%%%%%%%%%%%%
% Theorems
%%%%%%%%%%%%%%%%%%%%%%%%%%%%%%%%%%%%%%%%%%%%%%%%%%%%%%%%%%%%%%%%%%%%%%%%%%%%%%%%%%%%
\newtheorem{definition}{Definition}
\newtheorem{theorem}{Theorem}
\newtheorem{lemma}{Lemma}
\newtheorem{corollary}{Corollary}


%%%\begin{document}



\LARGE\textsc{Date: 2020-08-28} \\ \\ \LARGE\textsc{Announcements:} \\
\small

Exam date will be updeated on Blackboard today.
\textbf{Assignmtent: first homework due Wednesday}

\paragraph \hrule \paragraph \\ \fancyhead[R]{Lesson 5} \fancyhead[L]{Week 2}
%  %  %  %  %  %  %  %  %  %  %  %  %  %  %  %  %  %  %  %  %  %  %  %  %  %  %  %



\subsection{Trace of a Matrix}%
\label{sub:trace_of_a_matrix}



\begin{definition}[Trace]
	If A is a square matrix, then the trace of the matrix, dentoed by $tr(A)$ is the
	sum of the entries on the main diagonal.
\end{definition}
\[A = \begin{bmatrix}  1 & 2 & 3 \\ 4 & 5 & 6 \\ 7 & 8 & 9 \end{bmatrix} \]
\[tr(A) = 15\]
\[A = \begin{bmatrix}  2 & 3 & 4 \\ 1 & 5 & 9 \end{bmatrix} \]
\[tr(A) \:\text{not defined}\:\]

\subsection{Linear Combinations}%
\label{sub:linear_combinations}



\begin{definition}[Linear combination]
	If $A_1, A_2, \ldots, A_{r}$ are matrices of the same size and if $C_1, C_2, \ldots, C_{r}$ are
	scalars, then an expression of the form \[ C_1A_1 + C_2A_2 + \ldots + C_rA_{r}\] is called
	a linear combination of $A_1, A_2,\ldots,A_{r}$ with coefficients $C_1, C_2, \ldots, C_{r}$.
\end{definition}


\begin{theorem}[1.3.1]
If A is an $m\times n$ matrix and if $x$ is an $n\times 1$ column vector, then
the product $Ax$ can be expressed as a linear combination of the
column vectors in $A$ in which the coefficients are the entries
of $x$.
\end{theorem}

\[\begin{bmatrix} 1& 2 & 3 \\ 4 & 5 & 6 \end{bmatrix}  \begin{bmatrix} 3 & 1 \\ -1 & 2 \\ 5 & 9 \end{bmatrix} \]
\[= \begin{bmatrix} 16 & 32 \\ 37 & 68 \end{bmatrix} \]

We begin thinking about this in different ways.

\[= \begin{bmatrix} 16 & 32 \\ 37 & 68 \end{bmatrix} \implies \begin{bmatrix} A_{b_1} & A_{b_2} \end{bmatrix}  \]

In general: \[AB = A[b_1 b_2 \ldots b_{n}] = [Ab_1 Ab_2 \ldots Ab_{n}]\]

Nothing says you can't partition differently.

\[\begin{bmatrix} 1& 2 & 3 \\ 4 & 5 & 6 \end{bmatrix}  \begin{bmatrix} 3 & 1 \\ -1 & 2 \\ 5 & 9 \end{bmatrix} \]
\[\implies \begin{bmatrix} a_1 \\ a_2 \end{bmatrix} B \]
\[= \begin{bmatrix} a_1B \\ a_2B \end{bmatrix} \]

\section{Inverse and Algebraic Properties of Matrices}%
\label{sec:inverse_and_algebraic_properties_of_matrices}



The idea behind this section is we want to see these matrices like numbers in arithmetic.

\begin{definition}[zero matrix]
	a matrix of all zero's.
\end{definition}

\begin{definition}[identity matrix]
	An $n\times n$ identity matrix is a square matrix with 1's on the main diagonal and
	0's everywhere else.
\end{definition}

\begin{definition}[Additive inverse]
	\[A + (-A) = 0\]
\end{definition}

\begin{definition}[multiplication inverse]
	$A^{-1}$
\end{definition}



%  %  %  %  %  %  %  %  %  %  %  %  %  %  %  %  %  %  %  %  %  %  %  %  %  %  %  %
\newpage
%%%\end{document}


%LEAVE EMPTY ROW ABOVE THIS ONE
