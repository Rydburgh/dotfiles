%%%\documentclass[a4paper,11pt,twoside]{report}

%%%% ____  ____  _____    _    __  __ ____  _     _____
%|  _ \|  _ \| ____|  / \  |  \/  | __ )| |   | ____|
%| |_) | |_) |  _|   / _ \ | |\/| |  _ \| |   |  _|
%|  __/|  _ <| |___ / ___ \| |  | | |_) | |___| |___
%|_|   |_| \_\_____/_/   \_\_|  |_|____/|_____|_____|
%
% last updated: 2020-05-21
%%%%%%%%%%%%%%%%%%%%%%%%%%%%%%%%%%%%%%%%%%%%%%%%%%%%%%%%%%%%%%%%%%%%%%%%%%%%%%%%%%%%%%%%%%
%%% PACKAGES
%%%%%%%%%%%%%%%%%%%%%%%%%%%%%%%%%%%%%%%%%%%%%%%%%%%%%%%%%%%%%%%%%%%%%%%%%%%%%%%%%%%%%%%%%%


\pdfminorversion=7						% to prevent errors when building pdf

% some basic packages
\usepackage{amsmath, amsthm, amssymb}
\usepackage{mhchem}						% for chemical symbols
\usepackage{url}						% to format hyperlink text
\usepackage{float}						% for custom figure/table environment
\usepackage{xifthen}					% to handle tests
\usepackage{booktabs}					% for table commands and optimisation
\usepackage{enumitem}					% to format enumerate, itemize, and description
\usepackage{textcomp}					% to support different glyphs
\usepackage{graphicx}					% to support \includegraphics
\usepackage[T1]{fontenc}				% for unicode encoding
\usepackage[utf8]{inputenc}				% for unicode input
\setlength{\headheight}{13.6pt}
\usepackage[top=1.5in,bottom=1in,right=1in,left=1in,headheight=45pt]{geometry}
\usepackage{fancyhdr}					% for adding different headers
\pagestyle{fancy}
% for list of equations
\usepackage{tocloft}					% for custom lists
\usepackage{ragged2e} 					% to undo \centering
\usepackage{hyperref} 					% to make references hyperlinks
\usepackage{glossaries}

% for figures
\usepackage{import}						% for file control
\usepackage{pdfpages}					% for pdf, graphics, and hypertext
\usepackage{transparent}				% for color stack transparency
\usepackage{xcolor}						% for arbitrary color mixing



\author{Jasper Runco}
\date{2020 // Fall}

%%%%%%%%%%%%%%%%%%%%%%%%%%%%%%%%%%%%%%%%%%%%%%%%%%%%%%%%%%%%%%%%%%%%%%%%%%%%%%%%%%%%
% Commands
%%%%%%%%%%%%%%%%%%%%%%%%%%%%%%%%%%%%%%%%%%%%%%%%%%%%%%%%%%%%%%%%%%%%%%%%%%%%%%%%%%%%

% to make a new figure
\newcommand{\incfig}[2][scale=1]{%
	% \def\svgwidth{#1\columnwidth}
	\import{./figures/}{#2.pdf_tex}
}
\pdfsuppresswarningpagegroup=1

% define list of equations
\newcommand{\listequationsname}{\Large{List of Equations}}
\newlistof{myequations}{equ}{\listequationsname}
\newcommand{\myequations}[1]{
	\phantomsection
	\addcontentsline{equ}{myequations}{\protect\numberline{\theequation}#1}
}

\setlength{\cftmyequationsnumwidth}{2.3em}
\setlength{\cftmyequationsindent}{1.5em}

% command to box, label, reference, and include
% noteworthy equations in list of equations
\newcommand{\noteworthy}[2]{
\begin{align} \label{#2} \ensuremath{\boxed{#1}} \end{align}
\myequations{#2} \centering \textit{#2} \justify}

\newtheorem{definition}{Definition}
\newtheorem{theorem}{Theorem}
\newtheorem{lemma}{Lemma}
\newtheorem{corollary}{Corollary}
\newtheorem{example}{Example}
\newtheorem{solution}{Solution}
\newtheorem{constant}{Constant}
\newtheorem{note}{Note}


%%%\begin{document}

\LARGE\textsc{Date: 2020-09-09} \\ \\ \LARGE\textsc{Announcements:} \\
\small



\paragraph \hrule \paragraph \\ \fancyhead[R]{Lesson 9} \fancyhead[L]{Week 4}
%  %  %  %  %  %  %  %  %  %  %  %  %  %  %  %  %  %  %  %  %  %  %  %  %  %  %  %
\subsection{Inversion Algorithm}%
\label{sub:inversion_algorithm}

To find the inverse of an invertible matrix A, find a sequence of Elementary row operations
that reduce A to the Identity and then perform that same sequence of operations on
$I_{n}$ to obtain $A^{-1}$

\begin{example}[Inversion Algorithm]
	\[A = \begin{bmatrix} 1 & 1 & 0 \\ -1 & 3 & 4 \\ 0 & 4 & 3 \end{bmatrix}  \]
	\[R_1+R_2 \to R_1\]
	\[A = \begin{bmatrix} 1 & 1 & 0 & | & 1 & 0 & 0\\ -1 & 3 & 4 & | & 0 & 1 & 1 \\ 0 & 4 & 3 & | & 0 & 0 & 1 \end{bmatrix}  \]
	\[\to \begin{bmatrix} 1 & 1 & 0 & | & 1 & 0 & 0\\ 0 & 4 & 4 & | & 1 & 1 & 0 \\ 0 & 4 & 3 & | & 0 & 0 & 1 \end{bmatrix}  \]
	\[\frac{1}{4} R_2 \to R_2\]
	\[\to \begin{bmatrix} 1 & 1 & 0 & | & 1 & 0 & 0\\ 0 & 1 & 1 & | & \frac{1}{4} & \frac{1}{4} & 0 \\ 0 & 4 & 3 & | & 0 & 0 & 1 \end{bmatrix}  \]
	\[-4 R_2 + R_3 \to R_3\]
	\[\to \begin{bmatrix} 1 & 1 & 0 & | & 1 & 0 & 0\\ 0 & 1 & 1 & | & \frac{1}{4} & \frac{1}{4} & 0 \\ 0 & 0 & -1 & | & -1 & -1 & 1 \end{bmatrix}  \]
	\[-R_3 \to R_3\]
	\[\to \begin{bmatrix} 1 & 1 & 0 & | & 1 & 0 & 0\\ 0 & 1 & 1 & | & \frac{1}{4} & \frac{1}{4} & 0 \\ 0 & 0 & 1 & | & 1 & 1 & -1 \end{bmatrix}  \]
	If a zero results in the main diagonal of the matrix you are inverting, it is not invertible.
	\[-R_3 + R_2 \to R_2\]
	\[\to \begin{bmatrix} 1 & 1 & 0 & | & 1 & 0 & 0\\ 0 & 1 & 0 & | & \frac{-3}{4} & \frac{-3}{4} & 1 \\ 0 & 0 & 1 & | & 1 & 1 & -1 \end{bmatrix}  \]
	\[-R_2 + R_1 \to R_1\]
	\[\to \begin{bmatrix} 1 & 0 & 0 & | & \frac{7}{4} & \frac{3}{4} & 1\\ 0 & 1 & 0 & | & \frac{-3}{4} & \frac{-3}{4} & 1 \\ 0 & 0 & 1 & | & 1 & 1 & -1 \end{bmatrix}  \]
	\[\boxed{A^{-1} = \begin{bmatrix} \frac{7}{4} & \frac{3}{4} & 1\\ \frac{-3}{4} & \frac{-3}{4} & 1 \\ 1 & 1 & -1 \end{bmatrix} }\]
	To prove it, multiply the origional by the inverse to get the identity:
	\[A A^{-1} = I\]
	\[A^{-1} A = I\]
	\[A A^{-1} = \begin{bmatrix} 1 & 1 & 0 \\ -1 & 3 & 4 \\ 0 & 4 & 3 \end{bmatrix}\begin{bmatrix} \frac{7}{4} & \frac{3}{4} & 1\\ \frac{-3}{4} & \frac{-3}{4} & 1 \\ 1 & 1 & -1 \end{bmatrix} = \begin{bmatrix} 1& 0 & 0 \\ 0 1 & 0 \\ 0 & 0 & 1 \end{bmatrix} \]
	\[ A^{-1} A= \begin{bmatrix} \frac{7}{4} & \frac{3}{4} & 1\\ \frac{-3}{4} & \frac{-3}{4} & 1 \\ 1 & 1 & -1 \end{bmatrix}\begin{bmatrix} 1 & 1 & 0 \\ -1 & 3 & 4 \\ 0 & 4 & 3 \end{bmatrix} = \begin{bmatrix} 1& 0 & 0 \\ 0 1 & 0 \\ 0 & 0 & 1 \end{bmatrix} \]
\end{example}









%  %  %  %  %  %  %  %  %  %  %  %  %  %  %  %  %  %  %  %  %  %  %  %  %  %  %  %
\newpage
%%%\end{document}


%LEAVE EMPTY ROW ABOVE THIS ONE
