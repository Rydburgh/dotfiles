%%%\documentclass[a4paper,11pt,twoside]{report}

%%%%| |_) | |_) |  _|   / _ \ | |\/| |  _ \| |   |  _|
%|  __/|  _ <| |___ / ___ \| |  | | |_) | |___| |___
%|_|   |_| \_\_____/_/   \_\_|  |_|____/|_____|_____|
% Jasper Runco
% last updated: 2020-05-20
%%%%%%%%%%%%%%%%%%%%%%%%%%%%%%%%%%%%%%%%%%%%%%%%%%%%%%%%%%%%%%%%%%%%%%%%%%%%%%%%%%%%%%%%%%
%%% PACKAGES
%%%%%%%%%%%%%%%%%%%%%%%%%%%%%%%%%%%%%%%%%%%%%%%%%%%%%%%%%%%%%%%%%%%%%%%%%%%%%%%%%%%%%%%%%%
\pdfminorversion=7						% to prevent errors when building pdf
% some basic packages
\usepackage{amsthm, amsmath}
\usepackage{url}						% to format hyperlink text
\usepackage{float}						% for custom figure/table environment
\usepackage{xifthen}					% to handle tests
\usepackage{booktabs}					% for table commands and optimisation
\usepackage{enumitem}					% to format enumerate, itemize, and description
\usepackage{textcomp}					% to support different glyphs
\usepackage{graphicx}					% to support \includegraphics
\usepackage[T1]{fontenc}				% for unicode encoding
\usepackage[utf8]{inputenc}				% for unicode input
\setlength{\headheight}{13.6pt}
\usepackage[top=1.5in,bottom=1in,right=1in,left=1in,headheight=45pt]{geometry}
\usepackage{fancyhdr}					% for adding different headers
\pagestyle{fancy}
% for list of equations
\usepackage{tocloft}					% for custom lists
\usepackage{ragged2e} 					% to undo \centering
\usepackage{hyperref} 					% to make references hyperlinks
\usepackage{glossaries}
% for figures
\usepackage{import}						% for file control
\usepackage{pdfpages}					% for pdf, graphics, and hypertext
\usepackage{transparent}				% for color stack transparency
\usepackage{xcolor}						% for arbitrary color mixing
%%%%%%%%%%%%%%%%%%%%%%%%%%%%%%%%%%%%%%%%%%%%%%%%%%%%%%%%%%%%%%%%%%%%%%%%%%%%%%%%%%%%
% Commands
%%%%%%%%%%%%%%%%%%%%%%%%%%%%%%%%%%%%%%%%%%%%%%%%%%%%%%%%%%%%%%%%%%%%%%%%%%%%%%%%%%%%
% to make a new figure
\newcommand{\incfig}[2][1]{%
	\def\svgwidth{#1\columnwidth}
	\import{./figures/}{#2.pdf_tex}
}
\pdfsuppresswarningpagegroup=1
% define list of equations
\newcommand{\listequationsname}{\Large{List of Equations}}
\newlistof{myequations}{equ}{\listequationsname}
\newcommand{\myequations}[1]{
	\addcontentsline{equ}{myequations}{\protect\numberline{\theequation}#1}
}
\setlength{\cftmyequationsnumwidth}{2.3em}
\setlength{\cftmyequationsindent}{1.5em}
% command to box, label, refference, and include
% noteworthy equations in list of equations
\newcommand{\noteworthy}[2]{
\begin{align} \label{#2} \ensuremath{\boxed{#1}} \end{align}
\myequations{#2} \centering \small \textit{#2} \normalsize \justify }
%%%%%%%%%%%%%%%%%%%%%%%%%%%%%%%%%%%%%%%%%%%%%%%%%%%%%%%%%%%%%%%%%%%%%%%%%%%%%%%%%%%%
% Theorems
%%%%%%%%%%%%%%%%%%%%%%%%%%%%%%%%%%%%%%%%%%%%%%%%%%%%%%%%%%%%%%%%%%%%%%%%%%%%%%%%%%%%
\newtheorem{definition}{Definition}
\newtheorem{theorem}{Theorem}
\newtheorem{lemma}{Lemma}
\newtheorem{corollary}{Corollary}


%%%\begin{document}

% cha
\LARGE\textsc{Date: 2020-09-14} \\ \\ \LARGE\textsc{Announcements:} \\
\small

\textbf{Assignment: 1.8 1,3,5,7,9,11,15,17,19}


\paragraph \hrule \paragraph \\ \fancyhead[R]{Lesson 12} \fancyhead[L]{Week 4}
%  %  %  %  %  %  %  %  %  %  %  %  %  %  %  %  %  %  %  %  %  %  %  %  %  %  %  %
\section{(1.8) Matrix Transformations}%
\label{sec:_1_8_matrix_transformations}

\begin{definition}[Transformations]

If $f$ is a function with Domain $\mathbb{R}^{n}$ and codomain $\mathbb{R}^{m}$, then we say that
$f$ is a transformation from $\mathbb{R}^{n} \to \mathbb{R}^{m}$ or that $f$ maps from $\mathbb{R}^{n}$ to
$\mathbb{R}^{m}$ which we denote by writing \[f: \mathbb{R}^{n} \to \mathbb{R}^{m}\]

If $m=n$, then the transformation is sometimes called an operator on $\mathbb{R}^{n}$.
\smallskip\hfill$\bullet$\end{definition}

\begin{example}[transformations]
	\begin{align*}
		3x_1 + 4x_2 - 5x_3 + x_4 &= w_1 \\
		7x_1 - 9x_2 + 3x_3 - x_4 &= w_2 \\
		\implies \begin{bmatrix} w_1 \\ w_2 \end{bmatrix} &=
		\begin{bmatrix} 3 & 4 & -5 & 1 \\ 7 & -9 & 3 & -1 \end{bmatrix}
		\begin{bmatrix} x_1 \\ x_2 \\ x_3 \\ x_4 \end{bmatrix} \\
	\end{align*}

	4 dimensional space would be the domain, and this transformation maps it to an
	output with a 2 dimensional codomain.
	\[f: \mathbb{R}^2 \to \mathbb{R}^3\]
\smallskip\hfill$\bullet$\end{example}

\subsection{Matrix Transformation Notation}%
\label{sub:matrix_transformation}

When you can accomplish what you want to do with just one matrix multiplication:
	\[w_{(m\times 1)} = A_{(m\times n)}x_{(n\times 1}\]

Then this tranformation is called Matrix Transformation, can use Matrix Transformation Notation

\[T_{A}: R^{n} \to R^{m}\]

\[w = T_{A}(x)\]

\begin{example}[Matrix transformations].

	A)
	\begin{align*}
		w_1 &= x_1^2 + 3x_2 + x_3 \\
		w_2 &= x_1^2 + 3x_2 x_3 \\
	\end{align*}

	B)
	\begin{align*}
		w_1 &=  4x_1 -2x_2 + x_3 \\
		w_2 &= 5x_1 -7x_2 + 7x_3 \\
	\end{align*}

	A and B are both $\mathbb{R}^3 \to \mathbb{R}^2$

	A) Very different situation
	\[\begin{bmatrix} w_1 \\ w_2 \end{bmatrix} = \begin{bmatrix} x_1 & 3 & 1 \\ x_1 & 3 & -1 \end{bmatrix}
	\begin{bmatrix} x_1 \\ x_2 \\ x_3 \end{bmatrix} \]
	This is a transformation, but we cannot use matrix multiplication so
	all we can say is $f: \mathbb{R}^3 \to \mathbb{R}^2$


	B)

	\[\begin{bmatrix} w_1 \\ w_2 \end{bmatrix} = \begin{bmatrix} 4 & -2 & 1 \\ 5 & -7 & 7 \end{bmatrix}
	\begin{bmatrix} x_1 \\ x_2 \\ x_3 \end{bmatrix} \]

	\[T_{A}: \mathbb{R}^3 \to \mathbb{R}^2\]
	\[A = \begin{bmatrix} 4 & -2 & 1 \\ 5 & -7 & 7 \end{bmatrix} \]
	\[w = Ax \]
	\[w = T_{A}(x)\]

\smallskip\hfill$\bullet$\end{example}


\begin{example}[matrix transformations]
	\begin{align*}
		w_1 &=  6x_1 - 5x_2 + 7 \\
		w_2 &=  x_2 - 3x_2 + 4 \\
		w_3 &= x_1 - 7x_2 + 5 \\
		\mathbb{R}^2 &\to \mathbb{R}^3 \\
		\begin{bmatrix} w_1 \\w_2 \\w_3 \end{bmatrix} &=
		\begin{bmatrix} 6 & -5 \\ 1 & -3 \\ 1 & -7 \end{bmatrix}
		\begin{bmatrix} x_1 \\ x_2 \end{bmatrix} \\
	\end{align*}
\smallskip\hfill$\bullet$\end{example}

\begin{theorem}[matrix transformation proporties]

For every matrix A, the matrix transformation $T_{A}: \mathbb{R}^{n}\to\mathbb{R}^{m}$ has
the following proporties:

\begin{enumerate}
	\item $T_{A}(0) = 0$
	\item $T_{A}(kx) = k T_{A}(x)$
	\item $T_{A}(x_1+x_2) = T_{A}(x_1) + T_{A}(x_2)$
	\item $T_{A}(x_1 - x_2) = T_{A}(x_1) = T_{A}(x_2)$
\end{enumerate}

\smallskip\hfill$\bullet$\end{theorem}

You should be able to identify the Domain and Codomain.
Find matrix of transformation.
Tell whether or not someting is a matrix transformation.


%  %  %  %  %  %  %  %  %  %  %  %  %  %  %  %  %  %  %  %  %  %  %  %  %  %  %  %
\newpage
%%%\end{document}


%LEAVE EMPTY ROW ABOVE THIS ONE
