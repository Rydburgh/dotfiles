%%%\documentclass[a4paper,11pt,twoside]{report}

%%%%| |_) | |_) |  _|   / _ \ | |\/| |  _ \| |   |  _|
%|  __/|  _ <| |___ / ___ \| |  | | |_) | |___| |___
%|_|   |_| \_\_____/_/   \_\_|  |_|____/|_____|_____|
% Jasper Runco
% last updated: 2020-05-20
%%%%%%%%%%%%%%%%%%%%%%%%%%%%%%%%%%%%%%%%%%%%%%%%%%%%%%%%%%%%%%%%%%%%%%%%%%%%%%%%%%%%%%%%%%
%%% PACKAGES
%%%%%%%%%%%%%%%%%%%%%%%%%%%%%%%%%%%%%%%%%%%%%%%%%%%%%%%%%%%%%%%%%%%%%%%%%%%%%%%%%%%%%%%%%%
\pdfminorversion=7						% to prevent errors when building pdf
% some basic packages
\usepackage{amsthm, amsmath}
\usepackage{url}						% to format hyperlink text
\usepackage{float}						% for custom figure/table environment
\usepackage{xifthen}					% to handle tests
\usepackage{booktabs}					% for table commands and optimisation
\usepackage{enumitem}					% to format enumerate, itemize, and description
\usepackage{textcomp}					% to support different glyphs
\usepackage{graphicx}					% to support \includegraphics
\usepackage[T1]{fontenc}				% for unicode encoding
\usepackage[utf8]{inputenc}				% for unicode input
\setlength{\headheight}{13.6pt}
\usepackage[top=1.5in,bottom=1in,right=1in,left=1in,headheight=45pt]{geometry}
\usepackage{fancyhdr}					% for adding different headers
\pagestyle{fancy}
% for list of equations
\usepackage{tocloft}					% for custom lists
\usepackage{ragged2e} 					% to undo \centering
\usepackage{hyperref} 					% to make references hyperlinks
\usepackage{glossaries}
% for figures
\usepackage{import}						% for file control
\usepackage{pdfpages}					% for pdf, graphics, and hypertext
\usepackage{transparent}				% for color stack transparency
\usepackage{xcolor}						% for arbitrary color mixing
%%%%%%%%%%%%%%%%%%%%%%%%%%%%%%%%%%%%%%%%%%%%%%%%%%%%%%%%%%%%%%%%%%%%%%%%%%%%%%%%%%%%
% Commands
%%%%%%%%%%%%%%%%%%%%%%%%%%%%%%%%%%%%%%%%%%%%%%%%%%%%%%%%%%%%%%%%%%%%%%%%%%%%%%%%%%%%
% to make a new figure
\newcommand{\incfig}[2][1]{%
	\def\svgwidth{#1\columnwidth}
	\import{./figures/}{#2.pdf_tex}
}
\pdfsuppresswarningpagegroup=1
% define list of equations
\newcommand{\listequationsname}{\Large{List of Equations}}
\newlistof{myequations}{equ}{\listequationsname}
\newcommand{\myequations}[1]{
	\addcontentsline{equ}{myequations}{\protect\numberline{\theequation}#1}
}
\setlength{\cftmyequationsnumwidth}{2.3em}
\setlength{\cftmyequationsindent}{1.5em}
% command to box, label, refference, and include
% noteworthy equations in list of equations
\newcommand{\noteworthy}[2]{
\begin{align} \label{#2} \ensuremath{\boxed{#1}} \end{align}
\myequations{#2} \centering \small \textit{#2} \normalsize \justify }
%%%%%%%%%%%%%%%%%%%%%%%%%%%%%%%%%%%%%%%%%%%%%%%%%%%%%%%%%%%%%%%%%%%%%%%%%%%%%%%%%%%%
% Theorems
%%%%%%%%%%%%%%%%%%%%%%%%%%%%%%%%%%%%%%%%%%%%%%%%%%%%%%%%%%%%%%%%%%%%%%%%%%%%%%%%%%%%
\newtheorem{definition}{Definition}
\newtheorem{theorem}{Theorem}
\newtheorem{lemma}{Lemma}
\newtheorem{corollary}{Corollary}


%%%\begin{document}

\LARGE\textsc{Date: 2020-09-11} \\ \\ \LARGE\textsc{Announcements:} \\
\small

Test on Wednesday, 8:50

\textbf{Assignment: due Monday 4:00 PM}


\paragraph \hrule \paragraph \\ \fancyhead[R]{Lesson 11} \fancyhead[L]{Week 4}
%  %  %  %  %  %  %  %  %  %  %  %  %  %  %  %  %  %  %  %  %  %  %  %  %  %  %  %
\section{(1.7) Definitions of Matrices}%
\label{sec:_1_7_}

\begin{definition}[Diagonal Matrix]
	A square matrix where all the entries off of the main diagonal are zero.
	\[\begin{bmatrix} 1 & 0 & 0 \\ 0 & 1 & 0 \\ 0 & 0 & 1 \end{bmatrix} \]
	\[\begin{bmatrix} 1 & 0 & 0 \\ 0 & 3 & 0 \\ 0 & 0 & 5 \end{bmatrix} \]
	\[\begin{bmatrix} 1 & 0 & 0 & 0 \\ 0 & 0 & 0 & 0 \\0 & 0 & 5 & 0 \\ 0 & 0 & 0 & 6 \end{bmatrix} \]
\end{definition}

\subsubsection{Properties of diagonal matrices}%
\label{ssub:properties_of_diagonal_matrices}

\begin{itemize}
	\item Multiplication is easy. Multiplying a matrix by a diagonal from the left gives the entry
		in the diagonal times the row:
		\[\begin{bmatrix} 1 & 0 & 0 \\ 0 & 2 & 0 \\ 0 & 0 & 3 \end{bmatrix}
		\begin{bmatrix} 4 & 0 & 0 \\ 0 & 5 & 0 \\ 0 & 0 & 6 \end{bmatrix} =
		\begin{bmatrix} 4 & 0 & 0 \\ 0 & 10 & 0 \\ 0 & 0 & 18 \end{bmatrix} \]

		\[ \begin{bmatrix} -1 & 0 & 0 & \\ 0 & 2 & 0 \\ 0 & 0 & 3 \end{bmatrix}
		\begin{bmatrix} 1 & 2 \\ 3 & 4 \\ 5 & 6 \end{bmatrix} =
		\begin{bmatrix} -1 & -2 \\ 6 & 8 \\15 & 18 \end{bmatrix} \]

	\item Multiplying from the right gives the diagonal entry times the column.
		\[\begin{bmatrix} 1 & 2 & 3 \\ 4 & 5 & 6 \end{bmatrix}
		\begin{bmatrix} -1 & 0 & 0 \\ 0 & 2 & 0 \\ 0 & 0 & 3 \end{bmatrix} =
		\begin{bmatrix}  -1 & 4 & 9 \\ -4 & 10 & 18 \end{bmatrix} \]
	\item Square a diagonal squares the entries.

		\[D = \begin{bmatrix} 2 & 0 & 0 \\ 0 & 3 & 0 \\ 0 & 0 & 4 \end{bmatrix} \]
		\[D^2 = \begin{bmatrix} 2 & 0 & 0 \\ 0 & 3 & 0 \\ 0 & 0 & 4 \end{bmatrix}
		\begin{bmatrix} 2 & 0 & 0 \\ 0 & 3 & 0 \\ 0 & 0 & 4 \end{bmatrix}=
		\begin{bmatrix} 4 & 0 & 0 \\ 0 & 9 & 0 \\ 0 & 0 & 16 \end{bmatrix} \]

	\item Inverting the diagonal gives the reciprical of the entries

		\[D = \begin{bmatrix} 2 & 0 & 0 \\ 0 & 3 & 0 \\ 0 & 0 & 4 \end{bmatrix} \]

		\[\begin{bmatrix} 2 & 0 & 0 \\ 0 & 3 & 0 \\ 0 & 0 & 4 \end{bmatrix} D^{-1}	= \begin{bmatrix} 1 & 0 & 0 \\ 0 & 1 & 0 \\ 0 & 0 & 1 \end{bmatrix} \]
		\[\begin{bmatrix} 2 & 0 & 0 & | & 1 & 0 & 0  \\ 0 & 3 & 0 & | & 0 & 1 & 0 \\ 0 & 0 & 4 & | & 0 & 0 & 1 \end{bmatrix} =\]
		\[\begin{bmatrix} 1 & 0 & 0 & | & \frac{1}{2} & 0 & 0  \\ 0 & 1 & 0 & | & 0 & \frac{1}{3} & 0 \\ 0 & 0 & 1 & | & 0 & 0 & \frac{1}{4} \end{bmatrix} \]
	\item If it is invertible
		\[D = \begin{bmatrix}  3 & 0 & 0 \\ 0 & 5 & 0 \\ 0 & 0 & 0 \end{bmatrix} \]
		$D^{-1}$ (D is not invertible.)
\end{itemize}

\begin{definition}[Triangular Matrices]
	\begin{description}
		\item[Upper Triangular] All non-zero entries are on the main diagonal or above.
		\item[Lower Triangular] All non-zero entries are on the main diagonal or below.
	\end{description}
\end{definition}

\begin{example}[Upper triangular matrix]
	\[\begin{bmatrix}  1 & 3 & 4 \\ 0 & 4 & 5 \\ 0 & 0 & 6 \end{bmatrix} \]
\end{example}

\begin{example}[Lower triangular matrix]
	\[\begin{bmatrix}  1 & 0 & 0 \\ 2 & 3 & 0 \\ 4 & 5 & 6 \end{bmatrix} \]
\end{example}


\begin{theorem}[1.7.1] .

	\begin{enumerate}
		\item The \textbf{transpose} of a \textbf{lower} triangular is \textbf{upper} triangular and the transpose of
			an upper triangular matrix is lower triangular.
		\item The \textbf{product} of  \textbf{two lower} triangular matrices is \textbf{lower} triangular and
			the product of two upper triangular matrices is upper triangular.
		\item A triangular matrix is \textbf{invertible if }and only if it's \textbf{diagonal} entries
			are all \textbf{non-zero}.
		\item The \textbf{inverse} of an invertible \textbf{lower} triangular matrix \textbf{is lower}
			triangular and the inverse of an invertible \textbf{upper} triangular matrix
			\textbf{is upper} triangular.
\end{enumerate}
\end{theorem}

\subsection{Symmetric Matrix}%
\label{sub:symmetric_matrix}

\begin{definition}[Symmetric Matrix]
	A square matrix where $A = A^{T}$.

	\[(A)_{ij} = (A)_{ji}\]
\end{definition}

\begin{example}[Symmetric matrix]
	\[A = \begin{bmatrix} 1 & 2 & 3 \\ 2 & 5 & 4 \\ 3 & 4 & 6 \end{bmatrix} \]
	\[A^{T} = \begin{bmatrix} 1 & 2 & 3 \\ 2 & 5 & 4 \\ 3 & 4 & 6 \end{bmatrix} \]
$\bullet$\end{example}

\begin{theorem}[]
	If A and B are symmetric matrices with the same size, and if k is any scalar,
	then
	\begin{enumerate}
		\item $A^{T}$ is symmetric.
			\item $A+B$ and $A-B$ are symmetric.
				\item kA is symmetric.
	\end{enumerate}
$\bullet$\end{theorem}

\begin{theorem}[]
	The product of tow symmetric matrices is symmetric if and only if the matrices commute.
 \smallskip\hfill$\bullet$\end{theorem}


%  %  %  %  %  %  %  %  %  %  %  %  %  %  %  %  %  %  %  %  %  %  %  %  %  %  %  %
\newpage
%%%		\end{document}


		%LEAVE EMPTY ROW ABOVE THIS ONE
