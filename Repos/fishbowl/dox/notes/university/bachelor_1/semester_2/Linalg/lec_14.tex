%%%\documentclass[a4paper,11pt,twoside]{report}

%%%%| |_) | |_) |  _|   / _ \ | |\/| |  _ \| |   |  _|
%|  __/|  _ <| |___ / ___ \| |  | | |_) | |___| |___
%|_|   |_| \_\_____/_/   \_\_|  |_|____/|_____|_____|
% Jasper Runco
% last updated: 2020-05-20
%%%%%%%%%%%%%%%%%%%%%%%%%%%%%%%%%%%%%%%%%%%%%%%%%%%%%%%%%%%%%%%%%%%%%%%%%%%%%%%%%%%%%%%%%%
%%% PACKAGES
%%%%%%%%%%%%%%%%%%%%%%%%%%%%%%%%%%%%%%%%%%%%%%%%%%%%%%%%%%%%%%%%%%%%%%%%%%%%%%%%%%%%%%%%%%
\pdfminorversion=7						% to prevent errors when building pdf
% some basic packages
\usepackage{amsthm, amsmath}
\usepackage{url}						% to format hyperlink text
\usepackage{float}						% for custom figure/table environment
\usepackage{xifthen}					% to handle tests
\usepackage{booktabs}					% for table commands and optimisation
\usepackage{enumitem}					% to format enumerate, itemize, and description
\usepackage{textcomp}					% to support different glyphs
\usepackage{graphicx}					% to support \includegraphics
\usepackage[T1]{fontenc}				% for unicode encoding
\usepackage[utf8]{inputenc}				% for unicode input
\setlength{\headheight}{13.6pt}
\usepackage[top=1.5in,bottom=1in,right=1in,left=1in,headheight=45pt]{geometry}
\usepackage{fancyhdr}					% for adding different headers
\pagestyle{fancy}
% for list of equations
\usepackage{tocloft}					% for custom lists
\usepackage{ragged2e} 					% to undo \centering
\usepackage{hyperref} 					% to make references hyperlinks
\usepackage{glossaries}
% for figures
\usepackage{import}						% for file control
\usepackage{pdfpages}					% for pdf, graphics, and hypertext
\usepackage{transparent}				% for color stack transparency
\usepackage{xcolor}						% for arbitrary color mixing
%%%%%%%%%%%%%%%%%%%%%%%%%%%%%%%%%%%%%%%%%%%%%%%%%%%%%%%%%%%%%%%%%%%%%%%%%%%%%%%%%%%%
% Commands
%%%%%%%%%%%%%%%%%%%%%%%%%%%%%%%%%%%%%%%%%%%%%%%%%%%%%%%%%%%%%%%%%%%%%%%%%%%%%%%%%%%%
% to make a new figure
\newcommand{\incfig}[2][1]{%
	\def\svgwidth{#1\columnwidth}
	\import{./figures/}{#2.pdf_tex}
}
\pdfsuppresswarningpagegroup=1
% define list of equations
\newcommand{\listequationsname}{\Large{List of Equations}}
\newlistof{myequations}{equ}{\listequationsname}
\newcommand{\myequations}[1]{
	\addcontentsline{equ}{myequations}{\protect\numberline{\theequation}#1}
}
\setlength{\cftmyequationsnumwidth}{2.3em}
\setlength{\cftmyequationsindent}{1.5em}
% command to box, label, refference, and include
% noteworthy equations in list of equations
\newcommand{\noteworthy}[2]{
\begin{align} \label{#2} \ensuremath{\boxed{#1}} \end{align}
\myequations{#2} \centering \small \textit{#2} \normalsize \justify }
%%%%%%%%%%%%%%%%%%%%%%%%%%%%%%%%%%%%%%%%%%%%%%%%%%%%%%%%%%%%%%%%%%%%%%%%%%%%%%%%%%%%
% Theorems
%%%%%%%%%%%%%%%%%%%%%%%%%%%%%%%%%%%%%%%%%%%%%%%%%%%%%%%%%%%%%%%%%%%%%%%%%%%%%%%%%%%%
\newtheorem{definition}{Definition}
\newtheorem{theorem}{Theorem}
\newtheorem{lemma}{Lemma}
\newtheorem{corollary}{Corollary}


%%%\begin{document}

\chapter{Determinants}%
\label{cha:determinants}


\LARGE\textsc{Date: 2020-09-21} \\ \\ \LARGE\textsc{Announcements:} \\
\small

\textbf{assignment: 2.1 1-5 odd, 15-25 odd}


\paragraph \hrule \paragraph \\ \fancyhead[R]{Lesson 14} \fancyhead[L]{Week 6}
%  %  %  %  %  %  %  %  %  %  %  %  %  %  %  %  %  %  %  %  %  %  %  %  %  %  %  %

\section{Determinants by Co-factor Expansion}%
\label{sec:determinants_by_co_factor_expansion}

\begin{definition}[Determinant]
	If $A$ is an $n\times n$ matrix, the the number obtained by multiplying the
	entries of any row or any column of $A$ by the corresponding cofactors
	and adding the results together is called the determinant of $A$ and
	the sums are called the cofactor expansion of $A$.

	\[detA = a_{i1}c_{i1} + a_{i2}c_{i2} + \ldots + a_{in}c_{in}\]
	-cofactor expansion about the $i^{th}$ row.
	\[detA = a_{1j}c_{1j} + a_{2j}c_{2j} + \ldots + a_{nj}c_{nj}\]
	-cofactor expansion about the $j^{th}$ column.
\smallskip\hfill$\bullet$\end{definition}

\begin{definition}[Minor]
	If $A$ is a square matrix, then minor entry of $a_{ij}$ is denoted $M_{ij}$ and
	is defined to be the determinant of the submatrix obtained when the
	$i^{th}$ row and $j^{th}$ column of $A$ are deleted.
\smallskip\hfill$\bullet$\end{definition}

\begin{definition}[Co-factor]
	The number $(-1)^{i+j}M_{ij}$ is denoted $C_{ij}$ and is called the co-factor
	entry of $a_{ij}$.
\smallskip\hfill$\bullet$\end{definition}

\begin{example}[3x3]
	\[A = \begin{bmatrix}  1 & 2 & 3 \\ 4 & 5 & 6 \\ 7 & 8 & 9 \end{bmatrix} \]
	\[M_{23} = 8 - 14 = -6\]
	\[C_{23} = (-1)^{5}(-6) = -1 \cdot -6 = 6\]
\smallskip\hfill$\bullet$\end{example}

\begin{example}[Determinant]
	\[A = \begin{bmatrix} a_{11} & a_{12} \\ a_{21} & a_{22} \end{bmatrix} \]
	\begin{align*}
	detA &=  a_{11}c_{11} + a_{12}c_{12} \\
		 &= a_{11}(-1)^{1+1}a_{22} + a_{12} (-1)^{1+2}a_{21} \\
		 &= a_{11}a_{22} - a_{12}a_{21} \\
	\end{align*}
	\[A = \begin{bmatrix} a_{11} & a_{12} \\ a_{21} & a_{22} \end{bmatrix} \]
	\begin{align*}
	detA &=  a_{11}c_{11} + a_{21}c_{21} \\
		 &= a_{11}(-1)^{1+1}a_{22} + a_{21} (-1)^{1+2}a_{12} \\
		 &= a_{11}a_{22} - a_{12}a_{21} \\
	\end{align*}

\smallskip\hfill$\bullet$\end{example}

\begin{example}[3X3 Determinant]

	\[A = \begin{bmatrix} a_{11} & a_{12} & a_{13} \\ a_{21} & a_{22} & a_{23} \\ a_{31} & a_{32} & a_{33} \end{bmatrix} \]
	\begin{align*}
		detA &= a_{11}c_{11} + a_{12}c_{12} + a_{13}c_{13} \\
			 &= a_{11}(-1)^{2}\left[ a_{22}a_{33} - a_{23}a_{32} \right] \\
			 &+ a_{12}(-1)^{3}\left[ a_{21}a_{33}- a_{23}a_{31}\right] \\
			 &+ a_{13}(-1)^{4}[a_{21}a_{32}-a_{22}a_{31}]   \\
			 &= a_{11}a_{22}a_{33} - a_{11} a_{23}a_{32} - a_{12}a_{21}a_{33} + a_{12}a_{23}a_{31} + a_{13}a_{21}a32 -a_{13}a_{22}a_{31} \\
	\end{align*}

\smallskip\hfill$\bullet$\end{example}

\begin{example}[3X3 Determinant]
	"Using the minor entries and the coeficient with a sign, knowing what the cofactor would be"
	\[A = \begin{bmatrix}   1 & 3 & 11 \\ -2 & 4 & -3 \\ -1 & 5 & -7\end{bmatrix}\]
	\begin{align*}
		detA &= 1[-28 + 15] -3[14-3] + 1[-10-(-4)] \\
			 &= 1[-13] - 3[11] + [-6] \\
			 &= -52 \\
	\end{align*}
	"Doing cofactor expansion about second row"
	\begin{align*}
		detA &=   +2[-21 -5] + 4[-7-(-1)] -3[5-(-3)]\\
		&= 2(-26) +4(-6) +3(8) \\
		&= -52 \\
	\end{align*}
	"Doing cofactor expansion about third column"
	\begin{align*}
		detA &=   +1[-10 +4] + 3[5-(-3)] -7[4-(-3)]\\
		&= 1(-6) +3(8) -7(10) \\
		&= -52 \\
	\end{align*}
\smallskip\hfill$\bullet$\end{example}

\begin{example}[Making 3X3 Determinant Easy]
	\[A = \begin{bmatrix}  2 & 0 & 1 \\ 3 & 0 & 4 \\ -1 & 5 & 9 \end{bmatrix} \]
	\[detA = 0 + 0 -5[8-3] = -25\]

\smallskip\hfill$\bullet$\end{example}





%  %  %  %  %  %  %  %  %  %  %  %  %  %  %  %  %  %  %  %  %  %  %  %  %  %  %  %
\newpage
%%%\end{document}


%LEAVE EMPTY ROW ABOVE THIS ONE
