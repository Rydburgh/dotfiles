%%%\documentclass[a4paper,11pt,twoside]{report}

%%%% ____  ____  _____    _    __  __ ____  _     _____
%|  _ \|  _ \| ____|  / \  |  \/  | __ )| |   | ____|
%| |_) | |_) |  _|   / _ \ | |\/| |  _ \| |   |  _|
%|  __/|  _ <| |___ / ___ \| |  | | |_) | |___| |___
%|_|   |_| \_\_____/_/   \_\_|  |_|____/|_____|_____|
%
% last updated: 2020-05-21
%%%%%%%%%%%%%%%%%%%%%%%%%%%%%%%%%%%%%%%%%%%%%%%%%%%%%%%%%%%%%%%%%%%%%%%%%%%%%%%%%%%%%%%%%%
%%% PACKAGES
%%%%%%%%%%%%%%%%%%%%%%%%%%%%%%%%%%%%%%%%%%%%%%%%%%%%%%%%%%%%%%%%%%%%%%%%%%%%%%%%%%%%%%%%%%


\pdfminorversion=7						% to prevent errors when building pdf

% some basic packages
\usepackage{amsmath, amsthm, amssymb}
\usepackage{mhchem}						% for chemical symbols
\usepackage{url}						% to format hyperlink text
\usepackage{float}						% for custom figure/table environment
\usepackage{xifthen}					% to handle tests
\usepackage{booktabs}					% for table commands and optimisation
\usepackage{enumitem}					% to format enumerate, itemize, and description
\usepackage{textcomp}					% to support different glyphs
\usepackage{graphicx}					% to support \includegraphics
\usepackage[T1]{fontenc}				% for unicode encoding
\usepackage[utf8]{inputenc}				% for unicode input
\setlength{\headheight}{13.6pt}
\usepackage[top=1.5in,bottom=1in,right=1in,left=1in,headheight=45pt]{geometry}
\usepackage{fancyhdr}					% for adding different headers
\pagestyle{fancy}
% for list of equations
\usepackage{tocloft}					% for custom lists
\usepackage{ragged2e} 					% to undo \centering
\usepackage{hyperref} 					% to make references hyperlinks
\usepackage{glossaries}

% for figures
\usepackage{import}						% for file control
\usepackage{pdfpages}					% for pdf, graphics, and hypertext
\usepackage{transparent}				% for color stack transparency
\usepackage{xcolor}						% for arbitrary color mixing



\author{Jasper Runco}
\date{2020 // Fall}

%%%%%%%%%%%%%%%%%%%%%%%%%%%%%%%%%%%%%%%%%%%%%%%%%%%%%%%%%%%%%%%%%%%%%%%%%%%%%%%%%%%%
% Commands
%%%%%%%%%%%%%%%%%%%%%%%%%%%%%%%%%%%%%%%%%%%%%%%%%%%%%%%%%%%%%%%%%%%%%%%%%%%%%%%%%%%%

% to make a new figure
\newcommand{\incfig}[2][scale=1]{%
	% \def\svgwidth{#1\columnwidth}
	\import{./figures/}{#2.pdf_tex}
}
\pdfsuppresswarningpagegroup=1

% define list of equations
\newcommand{\listequationsname}{\Large{List of Equations}}
\newlistof{myequations}{equ}{\listequationsname}
\newcommand{\myequations}[1]{
	\phantomsection
	\addcontentsline{equ}{myequations}{\protect\numberline{\theequation}#1}
}

\setlength{\cftmyequationsnumwidth}{2.3em}
\setlength{\cftmyequationsindent}{1.5em}

% command to box, label, reference, and include
% noteworthy equations in list of equations
\newcommand{\noteworthy}[2]{
\begin{align} \label{#2} \ensuremath{\boxed{#1}} \end{align}
\myequations{#2} \centering \textit{#2} \justify}

\newtheorem{definition}{Definition}
\newtheorem{theorem}{Theorem}
\newtheorem{lemma}{Lemma}
\newtheorem{corollary}{Corollary}
\newtheorem{example}{Example}
\newtheorem{solution}{Solution}
\newtheorem{constant}{Constant}
\newtheorem{note}{Note}


%%%\begin{document}

% cha
\LARGE\textsc{Date: 2020-09-09} \\ \\ \LARGE\textsc{Announcements:} \\
\small
\textbf{Assignment: 1.6 1-19 odd}

\textbf{Turn in posted Friday, due Monday at 4 pm}

\textbf{Text chapter 1: 9/16}


\paragraph \hrule \paragraph \\ \fancyhead[R]{Lesson 10} \fancyhead[L]{Week 4}
%  %  %  %  %  %  %  %  %  %  %  %  %  %  %  %  %  %  %  %  %  %  %  %  %  %  %  %

\section{More on Linear Systems and Invertible Matrices}%
\label{sec:1_6}



\begin{example}[Solving system using the inverse]
	Solve:
	\begin{align*}
		2x + 6y + 6z &= 1 \\
		2x + 7y + 6z &= 2 \\
		2x + 7y + 7z &= 3 \\
	\end{align*}

	\[Ax = b\]

	\[\begin{bmatrix} 2 & 6 & 6 \\ 2 & 7 & 6 \\ 2 & 7 & 7 \end{bmatrix} \begin{bmatrix} x \\ y \\ z \end{bmatrix} = \begin{bmatrix} 1 \\ 2 \\ 3  \end{bmatrix} \]

	If $A^{-1}$ exists, \[A^{-1} Ax = A^{-1}b\]
	\[x = A^{-1}b\]

	\[ \frac{1}{2} R_1 \to R_1\]
	\[-R_2 + R_3 \to R_3\]

	\[\begin{bmatrix} 2 & 6 & 6 & | & 1& 0 & 0 \\ 2 & 7 & 6 &  | & 0 & 1 & 1 \\ 2 & 7 & 7 & | & 0 & 0 & 1 \end{bmatrix} \]
	\[\to \begin{bmatrix} 1 & 3 & 3 & | & \frac{1}{2}& 0 & 0 \\ 2 & 7 & 6 &  | & 0 & 1 & 1 \\ 0 & 0 & 1 & | & 0 & -1 & 0 \end{bmatrix} \]

	\[\to \begin{bmatrix} 1 & 3 & 0 & | & \frac{1}{2}& 3 & -3 \\ 0 & 1 & 0 &  | & -1 & 1 & 0 \\ 0 & 0 & 1 & | & 0 & -1 & 0 \end{bmatrix} \]
	\[\to \begin{bmatrix} 1 & 0 & 0 & | & \frac{7}{2}& 0 & -3 \\ 0 & 1 & 0 &  | & -1 & 1 & 0 \\ 0 & 0 & 1 & | & 0 & -1 & 0 \end{bmatrix} \]
	\[A^{-1} = \begin{bmatrix} \frac{7}{2} & 0 & -3 \\ -1 & 1 & 0 \\ 0 & -1 & 0 \end{bmatrix} \]

	\[\begin{bmatrix} \frac{7}{2} & 0 & -3 \\ -1 & 1 & 0 \\ 0 & -1 & 0 \end{bmatrix} \begin{bmatrix} 2 & 6 & 6 \\ 2 & 7 & 6 \\ 2 & 7 & 7 \end{bmatrix} \begin{bmatrix} x \\ y \\ z  \end{bmatrix} = \begin{bmatrix} \frac{7}{2} & 0 & -3 \\ -1 & 1 & 0 \\ 0 & -1 & 0 \end{bmatrix} \begin{bmatrix} 1 \\ 2 \\ 3 \end{bmatrix} \]
	check
	\[\begin{bmatrix} 1 & 0 & 0 \\ 0 & 1 & 0 \\ 0 & 0 & 1 \end{bmatrix} \begin{bmatrix} x \\ y \\ z \end{bmatrix} = \begin{bmatrix} -\frac{11}{2} \\ 1 \\ 1 \end{bmatrix}  \]
	\[\boxed{\left(  -\frac{11}{2}, 1, 1 \right)}\]
\end{example}

\begin{theorem}[1.6.1]
	A system of linear equations has zero, one or an infinite number of solutions.

	\[Ax = b\]
	\begin{description}
		\item[A] has Reduced Row Eschelon form that is $I_{n}$.

			or

		\item[A] has Reduced Row Eschelon form that is not.

	\end{description}
	Suppose that $Ax = b$ has \textbf{two} solutions $x_1$ and $x_2$ (proof by contradiction)

	We know
	\begin{align*}
		Ax_1 &= b \\
		Ax_2 &= b \\
		Ax_1 - Ax_2 &=  A(x_1-x_2) \\
		b - b &= A (x_1 - x_2) \implies \\
		A(x_1-x_2) &= 0 \\
	\end{align*}
	By equivalence principal, the matrix $x_1-x_2=0$ has only trivial solution $\implies \boxed{x_1 = x_2}$


\end{theorem}

If A is an $n \times n$ matrix, the following are equivalent
\begin{enumerate}
	\item A is invertible
	\item $Ax = 0$ has only the trivial solution
	\item The reduced row eschelon form is I
	\item A is expressible as a product of Elementary matrices.
	\item $Ax=b$ is consistent for every $n\times 1$ matrix b. (1 solution or an infinite)
	\item $Ax = b$ has exactly one solution for every $n\times 1$ matrix b:
		\begin{align*}
			Ax &= b \\
			x &= \boxed{A^{-1}b} \\
		\end{align*}
\end{enumerate}


\subsection{A Fundamental Problem}%
\label{sub:a_fundamental_problem}

\begin{example}[A fundamental Problem]
	Let A be a fixed $m \times n $ matrix. Find all $m\times 1$ matrices b such that
	the system of equations $Ax = b$ is consistent.

	\begin{align*}
		 2 + y &=  b_1 \\
		 -2y + 4z &= b_2 \\
		 3x -2z &=   b_3\\
		 \begin{bmatrix}  2 & 1 & 0 & | & b_1 \\0 & -2 & 4  & | & b_2 \\ 3 & 0 & -2 & | & b_3 \end{bmatrix} &\to \\
		 \frac{1}{2} R_1 &\to R_1 \\
		 \frac{-1}{2} R_2 &\to R_2 \\
		 \begin{bmatrix}  1 & \frac{1}{2} & 0 & | &\frac{1}{2} b_1 \\0 & 1 & -2  & | & -\frac{1}{2}b_2 \\ 3 & 0 & -2 & | & b_3 \end{bmatrix} &\to \\
		 -3 R_1 + R_3 &\to R_3 \\
		 \begin{bmatrix}  1 & \frac{1}{2} & 0 & | &\frac{1}{2} b_1 \\0 & 1 & -2  & | & -\frac{1}{2}b_2 \\ 0 & -\frac{3}{2} & -2 & | & b_3 - \frac{3}{2}b_1 \end{bmatrix} &\to \\
		 \begin{bmatrix}  1 & \frac{1}{2} & 0 & | &\frac{1}{2} b_1 \\0 & 1 & -2  & | & -\frac{1}{2}b_2 \\ 0 & 0 & -5 & | & b_3 - \frac{3}{2}b_1 - \frac{3}{4} b_2 \end{bmatrix} &\to \\
	\end{align*}
\end{example}

\begin{example}[Different End of Fundamental problem]
	\[\begin{bmatrix} 1 & 2 & 0 & b_1 \\ 9 & 1 & 0 & 2b_1 - 3b_2\\ 0 & 0 & 0 & b_1 - 3b_2 + b_3 \end{bmatrix} \]
	\[\to b_1 - 3b_2 + b_3 = 0\]
	\[\to   b_3 =  3b_2 - b_1 \]
	\[\boxed{\begin{bmatrix} b_1 \\ b_2 \\ 3b_2 - b_1 \end{bmatrix} }\]
\end{example}






%  %  %  %  %  %  %  %  %  %  %  %  %  %  %  %  %  %  %  %  %  %  %  %  %  %  %  %
\newpage
%%%\end{document}


%LEAVE EMPTY ROW ABOVE THIS ONE
