%%%\documentclass[a4paper,11pt,twoside]{report}

%%%% ____  ____  _____    _    __  __ ____  _     _____
%|  _ \|  _ \| ____|  / \  |  \/  | __ )| |   | ____|
%| |_) | |_) |  _|   / _ \ | |\/| |  _ \| |   |  _|
%|  __/|  _ <| |___ / ___ \| |  | | |_) | |___| |___
%|_|   |_| \_\_____/_/   \_\_|  |_|____/|_____|_____|
%
% last updated: 2020-05-21
%%%%%%%%%%%%%%%%%%%%%%%%%%%%%%%%%%%%%%%%%%%%%%%%%%%%%%%%%%%%%%%%%%%%%%%%%%%%%%%%%%%%%%%%%%
%%% PACKAGES
%%%%%%%%%%%%%%%%%%%%%%%%%%%%%%%%%%%%%%%%%%%%%%%%%%%%%%%%%%%%%%%%%%%%%%%%%%%%%%%%%%%%%%%%%%


\pdfminorversion=7						% to prevent errors when building pdf

% some basic packages
\usepackage{amsmath, amsthm, amssymb}
\usepackage{mhchem}						% for chemical symbols
\usepackage{url}						% to format hyperlink text
\usepackage{float}						% for custom figure/table environment
\usepackage{xifthen}					% to handle tests
\usepackage{booktabs}					% for table commands and optimisation
\usepackage{enumitem}					% to format enumerate, itemize, and description
\usepackage{textcomp}					% to support different glyphs
\usepackage{graphicx}					% to support \includegraphics
\usepackage[T1]{fontenc}				% for unicode encoding
\usepackage[utf8]{inputenc}				% for unicode input
\setlength{\headheight}{13.6pt}
\usepackage[top=1.5in,bottom=1in,right=1in,left=1in,headheight=45pt]{geometry}
\usepackage{fancyhdr}					% for adding different headers
\pagestyle{fancy}
% for list of equations
\usepackage{tocloft}					% for custom lists
\usepackage{ragged2e} 					% to undo \centering
\usepackage{hyperref} 					% to make references hyperlinks
\usepackage{glossaries}

% for figures
\usepackage{import}						% for file control
\usepackage{pdfpages}					% for pdf, graphics, and hypertext
\usepackage{transparent}				% for color stack transparency
\usepackage{xcolor}						% for arbitrary color mixing



\author{Jasper Runco}
\date{2020 // Fall}

%%%%%%%%%%%%%%%%%%%%%%%%%%%%%%%%%%%%%%%%%%%%%%%%%%%%%%%%%%%%%%%%%%%%%%%%%%%%%%%%%%%%
% Commands
%%%%%%%%%%%%%%%%%%%%%%%%%%%%%%%%%%%%%%%%%%%%%%%%%%%%%%%%%%%%%%%%%%%%%%%%%%%%%%%%%%%%

% to make a new figure
\newcommand{\incfig}[2][scale=1]{%
	% \def\svgwidth{#1\columnwidth}
	\import{./figures/}{#2.pdf_tex}
}
\pdfsuppresswarningpagegroup=1

% define list of equations
\newcommand{\listequationsname}{\Large{List of Equations}}
\newlistof{myequations}{equ}{\listequationsname}
\newcommand{\myequations}[1]{
	\phantomsection
	\addcontentsline{equ}{myequations}{\protect\numberline{\theequation}#1}
}

\setlength{\cftmyequationsnumwidth}{2.3em}
\setlength{\cftmyequationsindent}{1.5em}

% command to box, label, reference, and include
% noteworthy equations in list of equations
\newcommand{\noteworthy}[2]{
\begin{align} \label{#2} \ensuremath{\boxed{#1}} \end{align}
\myequations{#2} \centering \textit{#2} \justify}

\newtheorem{definition}{Definition}
\newtheorem{theorem}{Theorem}
\newtheorem{lemma}{Lemma}
\newtheorem{corollary}{Corollary}
\newtheorem{example}{Example}
\newtheorem{solution}{Solution}
\newtheorem{constant}{Constant}
\newtheorem{note}{Note}


%%%\begin{document}


\LARGE\textsc{Date: 2020-09-06} \\ \\ \LARGE\textsc{Announcements:} \\
\small



\paragraph \hrule \paragraph \\ \fancyhead[R]{Lesson 5} \fancyhead[L]{Week 1}
%  %  %  %  %  %  %  %  %  %  %  %  %  %  %  %  %  %  %  %  %  %  %  %  %  %  %  %

\subsection{Integrals}%
\label{sub:integrals}

\begin{definition}[Integral]
	\[ \int_{a}^{b} \overline{r}(t)dt = \lim_{n \to \infty} \sum_{i=1}^{n} \overline{r}(t_{i}^{*})\Delta t \]
	\[= \left< \int_{a}^{b} f(t)dt, \int_{a}^{b} g(t)dt, \int_{a}^{b} h(t)dt    \right>\]
\end{definition}

\begin{example}[Integral]
	\[\int_{1}^{4} \left<4 t^{\frac{3}{2}}, t^2, \cos t \right> dt \]
	\[= \left<\int_{1}^{4} 4 t^{\frac{3}{2}} dt, \int_{1}^{4} t^2 dt, \int_{1}^{4} \cos t dt    \right>\]
	\[= \frac{8}{5}t^{\frac{5}{2}} \biggr\rvert_{1}^{4}, \frac{1}{3} t^3 \biggr\rvert_{1}^{4}, \sin t \biggr\rvert_{1}^{4}    \]
	\[= \left< \frac{256}{5} - \frac{8}{5}, \frac{64}{3} - \frac{1}{3}, \sin 4 - \sin 1\right>\]
	\[= \boxed{\left<\frac{248}{5}, 21, \sin 4 - \sin 1 \right>}\]
\end{example}

\begin{example}[Integral]
	\[ \int \left<t e^{t^2}, t e^{t}, \cos (5t) \right> dt\]
	\[= \left< \int t e^{t^2} dt, \int t e^{t} dt, \int \cos (5t) dt \right>\]
	\[= \left<\frac{1}{2} e^{t^2}+ C_1, t e^{t} - \int e^{t}dt, \frac{1}{5} \sin (5t) + C_3 \right>\]
	\[= \left<\frac{1}{2} e^{t^2}+ C_1, t e^{t} - -e^{t} + C_2, \frac{1}{5} \sin (5t) + C_3 \right>\]

	Sometimes, you want to pull out the constants:

	\[= \left<\frac{1}{2} e^{t^2}, t e^{t} - e^{t}, \frac{1}{5} \sin (5t) \right> + \left<C_1, C_2, C_3 \right>\]
	\[= \left<\frac{1}{2} e^{t^2}, t e^{t} - e^{t}, \frac{1}{5} \sin (5t) \right> + \overline{C}\]

\end{example}

\begin{example}[Integral]
	\[r'(t) = \left<t^2, e^{3t}, \sqrt{t}  \right>\]
	\[\overline{r}(0) = \left<4, 1, 5 \right>\]
	Find $\overline{r}(t)$

	\begin{align*}
		\overline{r}(t) &= \int \left<t^2, e^{3t}, \sqrt{t}  \right> dt\\
		&= \left<\int t^2 dt, \int e^{3t} dt, \int \sqrt{t} dt \right> \\
		&= \left<\frac{1}{3} t^3 + C_1, \frac{1}{3} e^{3t} + C_2, \frac{2}{3} t^{\frac{3}{2}} + C_3\right> \\
	\end{align*}

	Use initial condition:
	\begin{align*}
		&= \left< C_1, \frac{1}{3} + C_2, C_3 \right> = \left<4, 1, 5 \right> \implies\\
		C_1 &= 4 \\
		C_2 &=  \frac{2}{3} \\
		C_3 &=  5 \implies \\
		\overline{r}(t) &= \boxed{\left<\frac{1}{3} t^3 + 4, \frac{1}{3 e^{3t} + \frac{2}{3}, \frac{2}{3} t^{\frac{3}{2}}+ 5} \right>} \\
	\end{align*}
\end{example}



%  %  %  %  %  %  %  %  %  %  %  %  %  %  %  %  %  %  %  %  %  %  %  %  %  %  %  %
\newpage
%%%\end{document}


%LEAVE EMPTY ROW ABOVE THIS ONE
