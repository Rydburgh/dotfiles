%%%\documentclass[a4paper,11pt,twoside]{report}

%%%%| |_) | |_) |  _|   / _ \ | |\/| |  _ \| |   |  _|
%|  __/|  _ <| |___ / ___ \| |  | | |_) | |___| |___
%|_|   |_| \_\_____/_/   \_\_|  |_|____/|_____|_____|
% Jasper Runco
% last updated: 2020-05-20
%%%%%%%%%%%%%%%%%%%%%%%%%%%%%%%%%%%%%%%%%%%%%%%%%%%%%%%%%%%%%%%%%%%%%%%%%%%%%%%%%%%%%%%%%%
%%% PACKAGES
%%%%%%%%%%%%%%%%%%%%%%%%%%%%%%%%%%%%%%%%%%%%%%%%%%%%%%%%%%%%%%%%%%%%%%%%%%%%%%%%%%%%%%%%%%
\pdfminorversion=7						% to prevent errors when building pdf
% some basic packages
\usepackage{amsthm, amsmath}
\usepackage{url}						% to format hyperlink text
\usepackage{float}						% for custom figure/table environment
\usepackage{xifthen}					% to handle tests
\usepackage{booktabs}					% for table commands and optimisation
\usepackage{enumitem}					% to format enumerate, itemize, and description
\usepackage{textcomp}					% to support different glyphs
\usepackage{graphicx}					% to support \includegraphics
\usepackage[T1]{fontenc}				% for unicode encoding
\usepackage[utf8]{inputenc}				% for unicode input
\setlength{\headheight}{13.6pt}
\usepackage[top=1.5in,bottom=1in,right=1in,left=1in,headheight=45pt]{geometry}
\usepackage{fancyhdr}					% for adding different headers
\pagestyle{fancy}
% for list of equations
\usepackage{tocloft}					% for custom lists
\usepackage{ragged2e} 					% to undo \centering
\usepackage{hyperref} 					% to make references hyperlinks
\usepackage{glossaries}
% for figures
\usepackage{import}						% for file control
\usepackage{pdfpages}					% for pdf, graphics, and hypertext
\usepackage{transparent}				% for color stack transparency
\usepackage{xcolor}						% for arbitrary color mixing
%%%%%%%%%%%%%%%%%%%%%%%%%%%%%%%%%%%%%%%%%%%%%%%%%%%%%%%%%%%%%%%%%%%%%%%%%%%%%%%%%%%%
% Commands
%%%%%%%%%%%%%%%%%%%%%%%%%%%%%%%%%%%%%%%%%%%%%%%%%%%%%%%%%%%%%%%%%%%%%%%%%%%%%%%%%%%%
% to make a new figure
\newcommand{\incfig}[2][1]{%
	\def\svgwidth{#1\columnwidth}
	\import{./figures/}{#2.pdf_tex}
}
\pdfsuppresswarningpagegroup=1
% define list of equations
\newcommand{\listequationsname}{\Large{List of Equations}}
\newlistof{myequations}{equ}{\listequationsname}
\newcommand{\myequations}[1]{
	\addcontentsline{equ}{myequations}{\protect\numberline{\theequation}#1}
}
\setlength{\cftmyequationsnumwidth}{2.3em}
\setlength{\cftmyequationsindent}{1.5em}
% command to box, label, refference, and include
% noteworthy equations in list of equations
\newcommand{\noteworthy}[2]{
\begin{align} \label{#2} \ensuremath{\boxed{#1}} \end{align}
\myequations{#2} \centering \small \textit{#2} \normalsize \justify }
%%%%%%%%%%%%%%%%%%%%%%%%%%%%%%%%%%%%%%%%%%%%%%%%%%%%%%%%%%%%%%%%%%%%%%%%%%%%%%%%%%%%
% Theorems
%%%%%%%%%%%%%%%%%%%%%%%%%%%%%%%%%%%%%%%%%%%%%%%%%%%%%%%%%%%%%%%%%%%%%%%%%%%%%%%%%%%%
\newtheorem{definition}{Definition}
\newtheorem{theorem}{Theorem}
\newtheorem{lemma}{Lemma}
\newtheorem{corollary}{Corollary}


%%%\begin{document}

\LARGE\textsc{Date: 2020-09-12} \\ \\ \LARGE\textsc{Announcements:} \\
\small



\paragraph \hrule \paragraph \\ \fancyhead[R]{Lesson 10} \fancyhead[L]{Week 2}
%  %  %  %  %  %  %  %  %  %  %  %  %  %  %  %  %  %  %  %  %  %  %  %  %  %  %  %

\subsection{Oscilating Circle}%
\label{sub:oscilating_circle}

\begin{definition}[Oscilating Circle]
	The circle that lies in the oscilating plane of the curve at a point and has
	the same tangent, lies on the concave side of the curve and has a radius of $\frac{1}{\kappa}$
\smallskip\hfill$\bullet$\end{definition}

To find a circle, you would need a center and the Radius. We know the radius is $\frac{1}{\kappa}$.
To find the center, we must move the radius distance to the center from the edge touching the curve.
This is in the direction of the unit normal vector.

Scale the unit normal vector by the length of the radius:
\[\frac{1}{\kappa}\overline{N}(t)\]

So, the vector from the origin to the center is the vector from the origin to the point on the curve + $\frac{1}{\kappa} \overline{N}$


\begin{example}[oscilating circle]
	Given $y=4-x^2$, $\overline{r}(t) 0 \left<t, 4-t^2, 0 \right>$, find
	the equation of the oscilating circle when t=0, at $(0,4)$
\smallskip\hfill$\bullet$\end{example}

\begin{solution}[oscilating circle]
	\[\overline{r}(t) = \left<t, 4-t^2, 0 \right>\]
	\[\kappa = \frac{\left| \overline{r}' \times  \overline{r}'' \right| }{\left| \overline{r}' \right|^3 }\]
	\[\overline{r}' = \left< 1, -2t, 0 \right>\]
	\[\overline{r}'' = \left< 0, -2, 0 \right>\]
	\[\overline{r}' \times  \overline{r}'' = \begin{bmatrix} \hat{i} & \hat{j} & \hat{k} \\ 1 & -2t & 0 \\ 0 & -2 & 0\end{bmatrix} \]
	\[= \left<0, 0, -2 \right>\]
	\[\left| \overline{r}' \times \overline{r}'' \right| = 2\]
	\[\left| \overline{r}'(t) \right| = \sqrt{1+4t^2} \]
	\[\kappa = \frac{2}{(\sqrt{1+4t^2} )^3}\]
	\[\kappa(0) = 2\]
	\[\:\text{radius}\:= \frac{1}{2}\]
	We know the normal vector is in the negative-y direction because this is a prabola centered on the axis.
	\[\left<0, -1, 0 \right>\]

	\[\left<0,4,0 \right> + \frac{1}{2}\left<0, -1, 0 \right> = \left<0, \frac{7}{2}, 0 \right>\]
	\[(0,\frac{7}{2})\]
	\[(x-h)^2 + (y-k)^2 = r^2\]
	\[\boxed{x^2 + (y-\frac{7}{2})^2 = \frac{1}{4}}\]

\smallskip\hfill$\bullet$\end{solution}

\begin{example}[oscilating circle]
	Given same as above, find oscilating circle at $(2, 0)$
\smallskip\hfill$\bullet$\end{example}
\begin{solution}[oscilating circle]
	\[\kappa(2) = \frac{2}{(\sqrt{1+4(2)^2} )^3}\]
	\[\frac{17\sqrt{17} }{2}\]
	Center: $\left<2, 0, 0 \right> + \frac{17\sqrt{17} }{2}\overline{N}(t)$
	We need principal normal unit vector at t=2.
	\[\overline{T}(t) = \frac{1}{\sqrt{1+4t^2} }\left<1, -2t, 0 \right>\]
	\[\overline{T}'(t) = \frac{1}{\sqrt{1+4t^2} }\left< 0, -2, 0 \right> - \frac{1}{2} ({1+4t^2})^{-\frac{3}{2}}(8t)\left< 1, -2t, 0 \right>\]
	\[\overline{T}'(t) = \left<0,-\frac{2}{\sqrt{1+4t^2} }, 0 \right> + \frac{-4t}{\left( \sqrt{1+4t^2}  \right)^{3} }\left<1, -2t, 0 \right>\]
	\[\overline{T}'(2) = \left<0,-\frac{2}{\sqrt{1+4(2)^2} }, 0 \right> + \frac{-4(2)}{\left( \sqrt{1+4(2)^2}  \right)^{3} }\left<1, -2(2), 0 \right>\]
	\[\overline{T}'(2) = \left<-\frac{8}{17\sqrt{17} }, \frac{-2}{17\sqrt{17} }, 0 \right>\]
	\[\left| \overline{T}'(2) \right| = \sqrt{\frac{64}{17^2*17} + \frac{4}{17^2*17}} = \frac{2}{17}\]
	\[\overline{N}(2) = \frac{1}{\frac{2}{17}} \left<-\frac{8}{17\sqrt{17} }, -\frac{2}{17\sqrt{17} }, 0 \right>\]
	\[\left<-\frac{4}{\sqrt{17} }, -\frac{1}{\sqrt{17} }, 0 \right>\]
	Center: $\left<2,0,0 \right> + \frac{17\sqrt{17} }{2} \left<-\frac{4}{\sqrt{17} }, -\frac{1}{\sqrt{17} }, 0 \right>$
	\[=\left<2, 0, 0 \right>+ \left<-34, -\frac{17}{2}, 0 \right>\]
	\[=\left<-32, -\frac{17}{2}, 0 \right>\]
	\[(x-h)^2 + (y-k)^2 = r^2\]
	\[\boxed{(x+32)^2 + (y +\frac{17}{2})^2 = \frac{4913}{2}}\]

\smallskip\hfill$\bullet$\end{solution}




%  %  %  %  %  %  %  %  %  %  %  %  %  %  %  %  %  %  %  %  %  %  %  %  %  %  %  %
\newpage
%%%\end{document}


%LEAVE EMPTY ROW ABOVE THIS ONE
