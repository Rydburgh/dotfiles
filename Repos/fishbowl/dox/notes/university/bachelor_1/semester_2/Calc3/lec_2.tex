%%%\documentclass[a4paper,11pt,twoside]{report}

%%%% ____  ____  _____    _    __  __ ____  _     _____
%|  _ \|  _ \| ____|  / \  |  \/  | __ )| |   | ____|
%| |_) | |_) |  _|   / _ \ | |\/| |  _ \| |   |  _|
%|  __/|  _ <| |___ / ___ \| |  | | |_) | |___| |___
%|_|   |_| \_\_____/_/   \_\_|  |_|____/|_____|_____|
%
% last updated: 2020-05-21
%%%%%%%%%%%%%%%%%%%%%%%%%%%%%%%%%%%%%%%%%%%%%%%%%%%%%%%%%%%%%%%%%%%%%%%%%%%%%%%%%%%%%%%%%%
%%% PACKAGES
%%%%%%%%%%%%%%%%%%%%%%%%%%%%%%%%%%%%%%%%%%%%%%%%%%%%%%%%%%%%%%%%%%%%%%%%%%%%%%%%%%%%%%%%%%


\pdfminorversion=7						% to prevent errors when building pdf

% some basic packages
\usepackage{amsmath, amsthm, amssymb}
\usepackage{mhchem}						% for chemical symbols
\usepackage{url}						% to format hyperlink text
\usepackage{float}						% for custom figure/table environment
\usepackage{xifthen}					% to handle tests
\usepackage{booktabs}					% for table commands and optimisation
\usepackage{enumitem}					% to format enumerate, itemize, and description
\usepackage{textcomp}					% to support different glyphs
\usepackage{graphicx}					% to support \includegraphics
\usepackage[T1]{fontenc}				% for unicode encoding
\usepackage[utf8]{inputenc}				% for unicode input
\setlength{\headheight}{13.6pt}
\usepackage[top=1.5in,bottom=1in,right=1in,left=1in,headheight=45pt]{geometry}
\usepackage{fancyhdr}					% for adding different headers
\pagestyle{fancy}
% for list of equations
\usepackage{tocloft}					% for custom lists
\usepackage{ragged2e} 					% to undo \centering
\usepackage{hyperref} 					% to make references hyperlinks
\usepackage{glossaries}

% for figures
\usepackage{import}						% for file control
\usepackage{pdfpages}					% for pdf, graphics, and hypertext
\usepackage{transparent}				% for color stack transparency
\usepackage{xcolor}						% for arbitrary color mixing



\author{Jasper Runco}
\date{2020 // Fall}

%%%%%%%%%%%%%%%%%%%%%%%%%%%%%%%%%%%%%%%%%%%%%%%%%%%%%%%%%%%%%%%%%%%%%%%%%%%%%%%%%%%%
% Commands
%%%%%%%%%%%%%%%%%%%%%%%%%%%%%%%%%%%%%%%%%%%%%%%%%%%%%%%%%%%%%%%%%%%%%%%%%%%%%%%%%%%%

% to make a new figure
\newcommand{\incfig}[2][scale=1]{%
	% \def\svgwidth{#1\columnwidth}
	\import{./figures/}{#2.pdf_tex}
}
\pdfsuppresswarningpagegroup=1

% define list of equations
\newcommand{\listequationsname}{\Large{List of Equations}}
\newlistof{myequations}{equ}{\listequationsname}
\newcommand{\myequations}[1]{
	\phantomsection
	\addcontentsline{equ}{myequations}{\protect\numberline{\theequation}#1}
}

\setlength{\cftmyequationsnumwidth}{2.3em}
\setlength{\cftmyequationsindent}{1.5em}

% command to box, label, reference, and include
% noteworthy equations in list of equations
\newcommand{\noteworthy}[2]{
\begin{align} \label{#2} \ensuremath{\boxed{#1}} \end{align}
\myequations{#2} \centering \textit{#2} \justify}

\newtheorem{definition}{Definition}
\newtheorem{theorem}{Theorem}
\newtheorem{lemma}{Lemma}
\newtheorem{corollary}{Corollary}
\newtheorem{example}{Example}
\newtheorem{solution}{Solution}
\newtheorem{constant}{Constant}
\newtheorem{note}{Note}


%%%\begin{document}

\chapter{Cylinders and Quatric Surfaces}%
\label{cha:cylinders_and_quatric_surfaces}


\LARGE\textsc{Date: 2020-09-03} \\ \\ \LARGE\textsc{Announcements:} \\
\small
<++>


\paragraph \hrule \paragraph \\ \fancyhead[R]{Lesson 1} \fancyhead[L]{Week 1}
%  %  %  %  %  %  %  %  %  %  %  %  %  %  %  %  %  %  %  %  %  %  %  %  %  %  %  %

\begin{definition}[Cylinder]
	A cylinder is a surface that consists of al lines (called rulings) that are parallel
	to a given line and pass through a given plane curve.
\end{definition}

\begin{definition}[plane curve]
	A plane curve is any graph in a 2-d space.
\end{definition}

\begin{figure}[ht]
    \centering
    \incfig{how-to-draw-a-circle-in-calc-3}
    \caption{How to draw a circle in Calc 3}
    \label{fig:how-to-draw-a-circle-in-calc-3}
\end{figure}

\section{Quadric Surfaces}%
\label{sec:section_name_quadric_surfaces}

\begin{definition}[Quadric surface]
	The most general form is $Ax^2+By^2+Cz^2+Dxy+Eyz+Fxz+Gx+Hy+Iz+J=0$
	where A-J are constants.
\end{definition}

Using rotation and translation will be brought into one of the two standard forms:

\begin{definition}[Standard quadric forms]
	\[Ax^2+By^2+Cz^2+J = 0\] or
	\[Ax^2+By^2+Iz=0\]
\end{definition}

\begin{example}[What we do with this]
	$z = \frac{x^2}{9} + \frac{y^2}{4}$
\end{example}

\begin{solution}[]
	You should know this is an Elliptic Paraboloid and what that looks like.
\end{solution}

\paragraph{What if you dont recognize the quadric surface form?}

\paragraph{}
	Looking at the equation, figure out \textbf{TRACES}, either \textbf{horizontal} (z = constant) or \textbf{vertical}

\begin{figure}[ht]
    \centering
    \incfig{graphs-of-quadric-surfaces}
    \caption{Graphs of Quadric Surfaces}
    \label{fig:graphs-of-quadric-surfaces}
\end{figure}


















%  %  %  %  %  %  %  %  %  %  %  %  %  %  %  %  %  %  %  %  %  %  %  %  %  %  %  %
\newpage
%%%\end{document}


%LEAVE EMPTY ROW ABOVE THIS ONE
