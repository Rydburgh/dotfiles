%%%\documentclass[a4paper,11pt,twoside]{report}

%%%% ____  ____  _____    _    __  __ ____  _     _____
%|  _ \|  _ \| ____|  / \  |  \/  | __ )| |   | ____|
%| |_) | |_) |  _|   / _ \ | |\/| |  _ \| |   |  _|
%|  __/|  _ <| |___ / ___ \| |  | | |_) | |___| |___
%|_|   |_| \_\_____/_/   \_\_|  |_|____/|_____|_____|
%
% last updated: 2020-05-21
%%%%%%%%%%%%%%%%%%%%%%%%%%%%%%%%%%%%%%%%%%%%%%%%%%%%%%%%%%%%%%%%%%%%%%%%%%%%%%%%%%%%%%%%%%
%%% PACKAGES
%%%%%%%%%%%%%%%%%%%%%%%%%%%%%%%%%%%%%%%%%%%%%%%%%%%%%%%%%%%%%%%%%%%%%%%%%%%%%%%%%%%%%%%%%%


\pdfminorversion=7						% to prevent errors when building pdf

% some basic packages
\usepackage{amsmath, amsthm, amssymb}
\usepackage{mhchem}						% for chemical symbols
\usepackage{url}						% to format hyperlink text
\usepackage{float}						% for custom figure/table environment
\usepackage{xifthen}					% to handle tests
\usepackage{booktabs}					% for table commands and optimisation
\usepackage{enumitem}					% to format enumerate, itemize, and description
\usepackage{textcomp}					% to support different glyphs
\usepackage{graphicx}					% to support \includegraphics
\usepackage[T1]{fontenc}				% for unicode encoding
\usepackage[utf8]{inputenc}				% for unicode input
\setlength{\headheight}{13.6pt}
\usepackage[top=1.5in,bottom=1in,right=1in,left=1in,headheight=45pt]{geometry}
\usepackage{fancyhdr}					% for adding different headers
\pagestyle{fancy}
% for list of equations
\usepackage{tocloft}					% for custom lists
\usepackage{ragged2e} 					% to undo \centering
\usepackage{hyperref} 					% to make references hyperlinks
\usepackage{glossaries}

% for figures
\usepackage{import}						% for file control
\usepackage{pdfpages}					% for pdf, graphics, and hypertext
\usepackage{transparent}				% for color stack transparency
\usepackage{xcolor}						% for arbitrary color mixing



\author{Jasper Runco}
\date{2020 // Fall}

%%%%%%%%%%%%%%%%%%%%%%%%%%%%%%%%%%%%%%%%%%%%%%%%%%%%%%%%%%%%%%%%%%%%%%%%%%%%%%%%%%%%
% Commands
%%%%%%%%%%%%%%%%%%%%%%%%%%%%%%%%%%%%%%%%%%%%%%%%%%%%%%%%%%%%%%%%%%%%%%%%%%%%%%%%%%%%

% to make a new figure
\newcommand{\incfig}[2][scale=1]{%
	% \def\svgwidth{#1\columnwidth}
	\import{./figures/}{#2.pdf_tex}
}
\pdfsuppresswarningpagegroup=1

% define list of equations
\newcommand{\listequationsname}{\Large{List of Equations}}
\newlistof{myequations}{equ}{\listequationsname}
\newcommand{\myequations}[1]{
	\phantomsection
	\addcontentsline{equ}{myequations}{\protect\numberline{\theequation}#1}
}

\setlength{\cftmyequationsnumwidth}{2.3em}
\setlength{\cftmyequationsindent}{1.5em}

% command to box, label, reference, and include
% noteworthy equations in list of equations
\newcommand{\noteworthy}[2]{
\begin{align} \label{#2} \ensuremath{\boxed{#1}} \end{align}
\myequations{#2} \centering \textit{#2} \justify}

\newtheorem{definition}{Definition}
\newtheorem{theorem}{Theorem}
\newtheorem{lemma}{Lemma}
\newtheorem{corollary}{Corollary}
\newtheorem{example}{Example}
\newtheorem{solution}{Solution}
\newtheorem{constant}{Constant}
\newtheorem{note}{Note}


%%%\begin{document}

\LARGE\textsc{Date: 2020-09-20} \\ \\ \LARGE\textsc{Announcements:} \\
\small



\paragraph \hrule \paragraph \\ \fancyhead[R]{Lesson 14} \fancyhead[L]{Week 3}
%  %  %  %  %  %  %  %  %  %  %  %  %  %  %  %  %  %  %  %  %  %  %  %  %  %  %  %
\section{Partial Derivatives}%
\label{sec:partial_derivatives}

\begin{definition}[Partial Derivative]
\[z = f(x,y)\]

The partial derivative of $f$ with respect to $x$
\[f_{x} =  \frac{\partial f}{\partial x} = \frac{\partial z}{\partial x} = z_{x}\]

How the output $z$ changes with respect to $x$ - the $y$ stays constant.

\[\frac{\partial z}{\partial x} =\lim_{h \to 0} \frac{f(x +h, y) - f(x,y)}{h}\]
\smallskip\hfill$\bullet$\end{definition}

\begin{example}[Partial Derivative]
	\[z = f(x,y) = x^2y^3\]
	\[\frac{\partial z}{\partial x} = \lim_{h \to 0} \frac{(x+h)^2 y^3 - x^2y^3}{h}\]
	\[= \lim_{h \to 0} \frac{y^3\left(  (x+h)^2 - x^2\right) }{h}\]
	\[= y^3\lim_{h \to 0} \frac{  x^2+ 2xh +h^2 - x^2 }{h}\]
	\[= y^3\lim_{h \to 0} \frac{ h( 2x +h ) }{h}\]
	\[= y^3\lim_{h \to 0}  2x +h  = \boxed{2xy^3}\]
\smallskip\hfill$\bullet$\end{example}

\subsection{Higher Order Derivatives}%
\label{sub:higher_order_derivatives}

\begin{theorem}[Clairaut's Theorem]
	Suppose that $f$ is defined on a disk $D$ that contains the point (a,b). If the
	function $f_{xy}$ and $f_{yx}$ are both continuous on $D$, then the \[
	f_{xy}(a,b) = f_{yx}(a,b)\]
\smallskip\hfill$\bullet$\end{theorem}

\begin{corollary}[Clairaut's Corollary]
	\[f_{ x x y} = f_{xyx}= f_{yx x}\]
	\[f_{ y y x} = f_{yxy}= f_{xy y}\]

\smallskip\hfill$\bullet$\end{corollary}









%  %  %  %  %  %  %  %  %  %  %  %  %  %  %  %  %  %  %  %  %  %  %  %  %  %  %  %
\newpage
%%%\end{document}


%LEAVE EMPTY ROW ABOVE THIS ONE
