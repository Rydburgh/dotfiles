%%%
\documentclass[a4paper,11pt,twoside]{report}
%%%
% ____  ____  _____    _    __  __ ____  _     _____
%|  _ \|  _ \| ____|  / \  |  \/  | __ )| |   | ____|
%| |_) | |_) |  _|   / _ \ | |\/| |  _ \| |   |  _|
%|  __/|  _ <| |___ / ___ \| |  | | |_) | |___| |___
%|_|   |_| \_\_____/_/   \_\_|  |_|____/|_____|_____|
%
% last updated: 2020-05-21
%%%%%%%%%%%%%%%%%%%%%%%%%%%%%%%%%%%%%%%%%%%%%%%%%%%%%%%%%%%%%%%%%%%%%%%%%%%%%%%%%%%%%%%%%%
%%% PACKAGES
%%%%%%%%%%%%%%%%%%%%%%%%%%%%%%%%%%%%%%%%%%%%%%%%%%%%%%%%%%%%%%%%%%%%%%%%%%%%%%%%%%%%%%%%%%


\pdfminorversion=7						% to prevent errors when building pdf

% some basic packages
\usepackage{amsmath, amsthm, amssymb}
\usepackage{mhchem}						% for chemical symbols
\usepackage{url}						% to format hyperlink text
\usepackage{float}						% for custom figure/table environment
\usepackage{xifthen}					% to handle tests
\usepackage{booktabs}					% for table commands and optimisation
\usepackage{enumitem}					% to format enumerate, itemize, and description
\usepackage{textcomp}					% to support different glyphs
\usepackage{graphicx}					% to support \includegraphics
\usepackage[T1]{fontenc}				% for unicode encoding
\usepackage[utf8]{inputenc}				% for unicode input
\setlength{\headheight}{13.6pt}
\usepackage[top=1.5in,bottom=1in,right=1in,left=1in,headheight=45pt]{geometry}
\usepackage{fancyhdr}					% for adding different headers
\pagestyle{fancy}
% for list of equations
\usepackage{tocloft}					% for custom lists
\usepackage{ragged2e} 					% to undo \centering
\usepackage{hyperref} 					% to make references hyperlinks
\usepackage{glossaries}

% for figures
\usepackage{import}						% for file control
\usepackage{pdfpages}					% for pdf, graphics, and hypertext
\usepackage{transparent}				% for color stack transparency
\usepackage{xcolor}						% for arbitrary color mixing



\author{Jasper Runco}
\date{2020 // Fall}

%%%%%%%%%%%%%%%%%%%%%%%%%%%%%%%%%%%%%%%%%%%%%%%%%%%%%%%%%%%%%%%%%%%%%%%%%%%%%%%%%%%%
% Commands
%%%%%%%%%%%%%%%%%%%%%%%%%%%%%%%%%%%%%%%%%%%%%%%%%%%%%%%%%%%%%%%%%%%%%%%%%%%%%%%%%%%%

% to make a new figure
\newcommand{\incfig}[2][scale=1]{%
	% \def\svgwidth{#1\columnwidth}
	\import{./figures/}{#2.pdf_tex}
}
\pdfsuppresswarningpagegroup=1

% define list of equations
\newcommand{\listequationsname}{\Large{List of Equations}}
\newlistof{myequations}{equ}{\listequationsname}
\newcommand{\myequations}[1]{
	\phantomsection
	\addcontentsline{equ}{myequations}{\protect\numberline{\theequation}#1}
}

\setlength{\cftmyequationsnumwidth}{2.3em}
\setlength{\cftmyequationsindent}{1.5em}

% command to box, label, reference, and include
% noteworthy equations in list of equations
\newcommand{\noteworthy}[2]{
\begin{align} \label{#2} \ensuremath{\boxed{#1}} \end{align}
\myequations{#2} \centering \textit{#2} \justify}

\newtheorem{definition}{Definition}
\newtheorem{theorem}{Theorem}
\newtheorem{lemma}{Lemma}
\newtheorem{corollary}{Corollary}
\newtheorem{example}{Example}
\newtheorem{solution}{Solution}
\newtheorem{constant}{Constant}
\newtheorem{note}{Note}

%%%
\begin{document}

\LARGE\textsc{Date: 2020-09-22} \\ \\ \LARGE\textsc{Announcements:} \\
\small



\paragraph \hrule \paragraph \\ \fancyhead[R]{Lesson 16} \fancyhead[L]{Week 4}
%  %  %  %  %  %  %  %  %  %  %  %  %  %  %  %  %  %  %  %  %  %  %  %  %  %  %  %
\section{The Chain Rule}%
\label{sec:the_chain_rule}

\[y = f(g(x))\]

let
\begin{align*}
	u = g(x) &\to \frac{\partial u}{\partial x} = g'(x) \\
	y = f(u) &\to \frac{\partial y}{\partial u} = f'(u) \\
\end{align*}

\subsection{The Chain Rule (Case 1)}%
\label{sub:the_chain_rule_case_1_}

Suppose $z = f(x,y)$ is a differentiable function of $x$ and $y$, where $x$ and $y$ are
differentiable functions of $t$, Then $z$ is a differentiable function of $t$ and \[
\frac{dz}{dt} = \frac{\partial z}{\partial x} \cdot \frac{d x}{d t} + \frac{\partial z}{\partial y}\cdot \frac{d y}{d t} \]


\begin{example}[Case 1 Chain Rule]
	\begin{align*}
	z&= x^2y + 3x^3y^{4} \\
	x&= e^{2t} \\
	y&= \cos t \\
	\end{align*}
	\begin{align*}
		\frac{\partial z}{\partial x} &=  2xy + 9 x^2 y^{4} \\
		\frac{\partial z}{\partial y} &=  x^2 + 12 x^3 y^{3} \\
	\end{align*}
	\begin{align*}
		\frac{dx}{dt}&= 2e^{2t}\\
		\frac{dy}{dt}&= -\sin t \\
	\end{align*}

	\begin{align*}
		\frac{dz}{dt} &= (2xy + 9x^2y^{4})2e^{2t} + (x^2 + 12x^3y^3)\cdot -\sin t \\
	\end{align*}
\smallskip\hfill$\bullet$\end{example}

\subsection{The Chain Rule (Case 2)}%
\label{sub:the_chain_rule_case_2_}

Suppose $z = f(x,y)$ is a differentiable function of $x$ and $y$, where $x= g(s,t)$ and $y = h(s,t)$ are
differentiable functions of $t$, Then $z$ is a differentiable function of $s$ and $t$, Then








%  %  %  %  %  %  %  %  %  %  %  %  %  %  %  %  %  %  %  %  %  %  %  %  %  %  %  %
\newpage
%%%
\end{document}

%LEAVE EMPTY ROW ABOVE THIS ONE
