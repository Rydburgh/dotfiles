%%%\documentclass[a4paper,11pt,twoside]{report}

%%%% ____  ____  _____    _    __  __ ____  _     _____
%|  _ \|  _ \| ____|  / \  |  \/  | __ )| |   | ____|
%| |_) | |_) |  _|   / _ \ | |\/| |  _ \| |   |  _|
%|  __/|  _ <| |___ / ___ \| |  | | |_) | |___| |___
%|_|   |_| \_\_____/_/   \_\_|  |_|____/|_____|_____|
%
% last updated: 2020-05-21
%%%%%%%%%%%%%%%%%%%%%%%%%%%%%%%%%%%%%%%%%%%%%%%%%%%%%%%%%%%%%%%%%%%%%%%%%%%%%%%%%%%%%%%%%%
%%% PACKAGES
%%%%%%%%%%%%%%%%%%%%%%%%%%%%%%%%%%%%%%%%%%%%%%%%%%%%%%%%%%%%%%%%%%%%%%%%%%%%%%%%%%%%%%%%%%


\pdfminorversion=7						% to prevent errors when building pdf

% some basic packages
\usepackage{amsmath, amsthm, amssymb}
\usepackage{mhchem}						% for chemical symbols
\usepackage{url}						% to format hyperlink text
\usepackage{float}						% for custom figure/table environment
\usepackage{xifthen}					% to handle tests
\usepackage{booktabs}					% for table commands and optimisation
\usepackage{enumitem}					% to format enumerate, itemize, and description
\usepackage{textcomp}					% to support different glyphs
\usepackage{graphicx}					% to support \includegraphics
\usepackage[T1]{fontenc}				% for unicode encoding
\usepackage[utf8]{inputenc}				% for unicode input
\setlength{\headheight}{13.6pt}
\usepackage[top=1.5in,bottom=1in,right=1in,left=1in,headheight=45pt]{geometry}
\usepackage{fancyhdr}					% for adding different headers
\pagestyle{fancy}
% for list of equations
\usepackage{tocloft}					% for custom lists
\usepackage{ragged2e} 					% to undo \centering
\usepackage{hyperref} 					% to make references hyperlinks
\usepackage{glossaries}

% for figures
\usepackage{import}						% for file control
\usepackage{pdfpages}					% for pdf, graphics, and hypertext
\usepackage{transparent}				% for color stack transparency
\usepackage{xcolor}						% for arbitrary color mixing



\author{Jasper Runco}
\date{2020 // Fall}

%%%%%%%%%%%%%%%%%%%%%%%%%%%%%%%%%%%%%%%%%%%%%%%%%%%%%%%%%%%%%%%%%%%%%%%%%%%%%%%%%%%%
% Commands
%%%%%%%%%%%%%%%%%%%%%%%%%%%%%%%%%%%%%%%%%%%%%%%%%%%%%%%%%%%%%%%%%%%%%%%%%%%%%%%%%%%%

% to make a new figure
\newcommand{\incfig}[2][scale=1]{%
	% \def\svgwidth{#1\columnwidth}
	\import{./figures/}{#2.pdf_tex}
}
\pdfsuppresswarningpagegroup=1

% define list of equations
\newcommand{\listequationsname}{\Large{List of Equations}}
\newlistof{myequations}{equ}{\listequationsname}
\newcommand{\myequations}[1]{
	\phantomsection
	\addcontentsline{equ}{myequations}{\protect\numberline{\theequation}#1}
}

\setlength{\cftmyequationsnumwidth}{2.3em}
\setlength{\cftmyequationsindent}{1.5em}

% command to box, label, reference, and include
% noteworthy equations in list of equations
\newcommand{\noteworthy}[2]{
\begin{align} \label{#2} \ensuremath{\boxed{#1}} \end{align}
\myequations{#2} \centering \textit{#2} \justify}

\newtheorem{definition}{Definition}
\newtheorem{theorem}{Theorem}
\newtheorem{lemma}{Lemma}
\newtheorem{corollary}{Corollary}
\newtheorem{example}{Example}
\newtheorem{solution}{Solution}
\newtheorem{constant}{Constant}
\newtheorem{note}{Note}


%%%\begin{document}

\chapter{(14) Partial Derivatives}%
\label{cha:_14_partial_derivatives}


\LARGE\textsc{Date: 2020-09-20} \\ \\ \LARGE\textsc{Announcements:} \\
\small



\paragraph \hrule \paragraph \\ \fancyhead[R]{Lesson 12} \fancyhead[L]{Week 3}
%  %  %  %  %  %  %  %  %  %  %  %  %  %  %  %  %  %  %  %  %  %  %  %  %  %  %  %
\section{Functions of Several Variables}%
\label{sec:functions_of_several_variables}

A function of two variables is a rule that assigns to each ordered pair $(x,y)$ in a set
$D$ a unique real number denoted by $f(x,y)$. The set $D$ is the Domain of $f$ and its
Range is the set of values that $f$ takes on, that is $\{f(x,y) | (x,y) \in D\}$.

\subsubsection{Single Variables}%
\label{ssub:single_variables}


\[y = f(x)\]
\[(x,y)\]

\[f(x) = x^2 - 2x + 3\]
\[f(-3) = (-3)^2 -2(-3) + 3 = 18\]
\[(-3,18)\]
\subsubsection{Two Variables}%
\label{ssub:two_variables}

\[z = f(x,y)\]
\[(x,y,z)\]

\[f(x,y) = x^2y + xy\]
\[f(-3,2) = (-3)^2(2) + (-3)(2) = 12\]

\[(-3,2,12)\]


These are both polynomial functions.

A. Domain (1 variable): $(-\infty,\infty)$, All real numbers.

B. Domain (2 variables): $\mathbb{R}^2$

\[f(x) = \frac{\sqrt{x-3} }{x-7}\]
need $x \ge 3$, $x \neq 7$
\[[3,7)\cup (7,\infty)\]

\[f(x,y) = \frac{\sqrt{x + y -1} }{x-y} \]
need $y \ge -x +1$, $x \neq y$

You could graph this domain in the x,y-plane

\[f(x,y,z) = \ln (1 -x^2 -y^2 -z^2)\]
need $x^2 + y^2 + z^2 < 1$

Recognize special quadric surface, an elipsoid that is a sphere, and graph it

with dotted surface to represent the domain.

\subsection{Level Curve}%
\label{sub:level_curve}

The level curves of a function of two variables are the curves with equations $f(x,y) = k$ where
$k$ is a constant in the Range of $f$.

\begin{definition}[Contour Map]
	A 2-d representation of a surface drawn with level curves.
\smallskip\hfill$\bullet$\end{definition}

\begin{example}[Unfamiliar surface]
	\[f(x,y) = e^{x-y}\]
	This will never output a value of 0 or negative.

	\[1 = e^{x-y}\]
	\[\ln1 = x-y\]
	\[y=x\]

	\[2 = e^{x-y}\]
	\[\ln2 = x-y\]
	\[y=x - \ln2\]

	The larger the value $k$ gets, the more negative the y-intercept gets.

	This resembles a slide that slopes towards the x-y plane.
\smallskip\hfill$\bullet$\end{example}

\begin{example}[]
	\[f(x,y,z) = \sqrt{x^2+y^2-z} \]
\textbf{Level surfaces}
	\[0 = \sqrt{x^2+y^2-z} \]
\[z = x^2 + y ^2 \]
(paraboloid that maps to zero in 4-D) $(x,y,z,0)$
	\[1 = \sqrt{x^2+y^2-z} \]
\[z = x^2 + y ^2 -1 \]
(paraboloid that maps to one in 4-D) $(x,y,z,1)$
This paraboloid would shift down from input 0, and this could continue\ldots
\[(x,y,z,k)\]
\smallskip\hfill$\bullet$\end{example}

%  %  %  %  %  %  %  %  %  %  %  %  %  %  %  %  %  %  %  %  %  %  %  %  %  %  %  %
\newpage
%%%\end{document}


%LEAVE EMPTY ROW ABOVE THIS ONE
