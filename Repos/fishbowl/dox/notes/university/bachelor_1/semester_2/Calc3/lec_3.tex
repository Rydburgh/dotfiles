%%%\documentclass[a4paper,11pt,twoside]{report}

%%%% ____  ____  _____    _    __  __ ____  _     _____
%|  _ \|  _ \| ____|  / \  |  \/  | __ )| |   | ____|
%| |_) | |_) |  _|   / _ \ | |\/| |  _ \| |   |  _|
%|  __/|  _ <| |___ / ___ \| |  | | |_) | |___| |___
%|_|   |_| \_\_____/_/   \_\_|  |_|____/|_____|_____|
%
% last updated: 2020-05-21
%%%%%%%%%%%%%%%%%%%%%%%%%%%%%%%%%%%%%%%%%%%%%%%%%%%%%%%%%%%%%%%%%%%%%%%%%%%%%%%%%%%%%%%%%%
%%% PACKAGES
%%%%%%%%%%%%%%%%%%%%%%%%%%%%%%%%%%%%%%%%%%%%%%%%%%%%%%%%%%%%%%%%%%%%%%%%%%%%%%%%%%%%%%%%%%


\pdfminorversion=7						% to prevent errors when building pdf

% some basic packages
\usepackage{amsmath, amsthm, amssymb}
\usepackage{mhchem}						% for chemical symbols
\usepackage{url}						% to format hyperlink text
\usepackage{float}						% for custom figure/table environment
\usepackage{xifthen}					% to handle tests
\usepackage{booktabs}					% for table commands and optimisation
\usepackage{enumitem}					% to format enumerate, itemize, and description
\usepackage{textcomp}					% to support different glyphs
\usepackage{graphicx}					% to support \includegraphics
\usepackage[T1]{fontenc}				% for unicode encoding
\usepackage[utf8]{inputenc}				% for unicode input
\setlength{\headheight}{13.6pt}
\usepackage[top=1.5in,bottom=1in,right=1in,left=1in,headheight=45pt]{geometry}
\usepackage{fancyhdr}					% for adding different headers
\pagestyle{fancy}
% for list of equations
\usepackage{tocloft}					% for custom lists
\usepackage{ragged2e} 					% to undo \centering
\usepackage{hyperref} 					% to make references hyperlinks
\usepackage{glossaries}

% for figures
\usepackage{import}						% for file control
\usepackage{pdfpages}					% for pdf, graphics, and hypertext
\usepackage{transparent}				% for color stack transparency
\usepackage{xcolor}						% for arbitrary color mixing



\author{Jasper Runco}
\date{2020 // Fall}

%%%%%%%%%%%%%%%%%%%%%%%%%%%%%%%%%%%%%%%%%%%%%%%%%%%%%%%%%%%%%%%%%%%%%%%%%%%%%%%%%%%%
% Commands
%%%%%%%%%%%%%%%%%%%%%%%%%%%%%%%%%%%%%%%%%%%%%%%%%%%%%%%%%%%%%%%%%%%%%%%%%%%%%%%%%%%%

% to make a new figure
\newcommand{\incfig}[2][scale=1]{%
	% \def\svgwidth{#1\columnwidth}
	\import{./figures/}{#2.pdf_tex}
}
\pdfsuppresswarningpagegroup=1

% define list of equations
\newcommand{\listequationsname}{\Large{List of Equations}}
\newlistof{myequations}{equ}{\listequationsname}
\newcommand{\myequations}[1]{
	\phantomsection
	\addcontentsline{equ}{myequations}{\protect\numberline{\theequation}#1}
}

\setlength{\cftmyequationsnumwidth}{2.3em}
\setlength{\cftmyequationsindent}{1.5em}

% command to box, label, reference, and include
% noteworthy equations in list of equations
\newcommand{\noteworthy}[2]{
\begin{align} \label{#2} \ensuremath{\boxed{#1}} \end{align}
\myequations{#2} \centering \textit{#2} \justify}

\newtheorem{definition}{Definition}
\newtheorem{theorem}{Theorem}
\newtheorem{lemma}{Lemma}
\newtheorem{corollary}{Corollary}
\newtheorem{example}{Example}
\newtheorem{solution}{Solution}
\newtheorem{constant}{Constant}
\newtheorem{note}{Note}


%%%\begin{document}

\chapter{Vector Functions}%
\label{cha:vector_functions}


\LARGE\textsc{Date: 2020-09-06} \\ \\ \LARGE\textsc{Announcements:} \\
\small



\paragraph \hrule \paragraph \\ \fancyhead[R]{Lesson 2} \fancyhead[L]{Week 1}
%  %  %  %  %  %  %  %  %  %  %  %  %  %  %  %  %  %  %  %  %  %  %  %  %  %  %  %
\section{Vector Functions and Space Curves}%
\label{sec:vector_functions_and_space_curves}


\subsubsection{Vectors}%
\label{ssub:vectors}

\begin{align*}
	\overline{r(t)} &= \left<f(t), g(t), h(t) \right> \\
					&= f(t)\hat{i} + g(t)\hat{j} + h(t)\hat{k} \\
					\overline{r(t)} &= \left<t^2 + 1, \frac{1}{t-1}, \sqrt{t + 3}  \right> \\
					\overline{r(2)} &=   \left<5,1,\sqrt{5}  \right>\\
\end{align*}
\subsubsection{Domain}%
\label{ssub:domain}

\begin{align*}
	t^2 + 1 &\to \:\text{polonomial, Domain:}\: (-\infty, \infty) \\
	\frac{1}{t+	1} &\to \:\text{Rational, Domain:}\: (-\infty,1)\cup(1,\infty) \\
	\sqrt{t+3} &\to \:\text{Radical, Domain:}\: [-3,\infty)\\
\:\text{Domain of }\:\overline{r(t)}&=  [-3,1)\cup(1,\infty) \\
\end{align*}

\begin{example}[Domain]
	\[\overline{r(t)} = \left< \cos t, \ln(t+r), \frac{t}{t^2-23} \right>\]
\end{example}

\begin{align*}
	\cos t &\to (-\infty,\infty) \\
	\ln(t+4) &\to (-4, \infty) \\
	\frac{t}{t^2-25} &\to (-\infty,-5) \cup (-5,5) \cup(5,\infty)\\
	\:\text{Domain}\: &= (-4,5)\cup(5,\infty) \\
\end{align*}

\subsubsection{Graphs}%
\label{ssub:graphs}

\begin{description}
	\item[2-D] plane curve
		\item[3-D] space curve
\end{description}

\subsubsection{Plane curve}%
\label{ssub:plane_curve}


\[\overline{r(t)} = \left<t-3, t^2 \right>\]

\begin{table}[htpb]
	\centering
	\label{tab:label}
	\begin{tabular}{l | r}
		$\ldots$ \\
	$t=-2$ & \ $\left<-5,4 \right>$ \\
	$t=-1$ & \ $\left<-4,1 \right>$ \\
	$t=0$ & \ $\left<-3,0 \right>$ \\
	$t=1$ & \ $\left<-2, 1 \right>$ \\
	$t=2$ & \ $\left<-1, 4 \right>$ \\
		$\ldots$ \\
	\end{tabular}
\end{table}
\begin{figure}[ht]
    \centering
    \incfig{2-d-vector-graph}
    \caption{2-D vector graph}
    \label{fig:2-d-vector-graph}
\end{figure}

\subsubsection{Space Curve}%
\label{ssub:space_curve}


\[\overline{r(t)} = \left<3+t, 4-2t, t-5 \right>\]

\begin{align*}
	x &=  3 + t \\
	y &=  4 - 2t \\
	z &=  t-5 \\
\end{align*}
\[\:\text{(Parametric form)}\:\]

\[\overline{r(t)} = \left<3, 4t, -5 \right> + \left<t, -2t, t \right>\]
\[\left(   \overline{r(t)} = \left<3, 4t, -5 \right> + t\left<1, -2, 1 \right>\right)\:\text{}\:\]
\[\:\text{(Equation of line in 3-D)}\:\]

\[x-3 =\frac{y-4}{-2} = z+5\]
\[\left( \:\text{Symetric form}\: \right) \]

\subsection{Limit}%
\label{sub:limit}

\begin{definition}[Limit of a Vector]
	If $\overline{r(t)} = \left<f(t), g(t), h(t), \right>$, then
	\[\lim_{t \to a} \overline{r(t)} = \left<\lim_{t \to a} f(t), \lim_{t \to a}g(t), \lim_{t \to a}h(t) \right> \]

	provided the limits of the component functions exist.
\end{definition}

\begin{example}[Limit of a Vector]
	\[\overline{r(t)} = \left< \frac{\sin^2 t}{t^2}, t e^{-t}, t^2+1 \right>\]

	\begin{align*}
	\lim_{t \to 0}\overline{r(t)} &=  \lim_{t \to 0}\left< \frac{\sin^2 t}{t^2}, t e^{-t}, t^2+1 \right> \\
	&=  \left< \lim_{t \to 0} \frac{\sin^2 t}{t^2}, \lim_{t \to 0}t e^{-t}, \lim_{t \to 0}t^2+1 \right> \\
	&=  \left< a, b ,c \right> \\
	\end{align*}
	\begin{align*}
		a &=  \frac{0}{0}(\:\text{L'H Rule}\:) \\
		a &= \lim_{t \to 0}\frac{2 \sin t \cos t}{2 t} \\
		a &= \lim_{t \to 0}\frac{\sin t \cos t}{t} \\
		a &=  \frac{0}{0}(\:\text{L'H Rule}\:) \\
		a &=  \lim_{t \to 0 \frac{-\sin^2 t + \cos^2 t}{1}} \\
		a &=  1 \\
		b &=  0 \\
		c &=  1 \implies\\
	\end{align*}
	\[ = \boxed{ \left<1,0,1 \right>}\]

\end{example}











%  %  %  %  %  %  %  %  %  %  %  %  %  %  %  %  %  %  %  %  %  %  %  %  %  %  %  %
\newpage
%%%\end{document}


%LEAVE EMPTY ROW ABOVE THIS ONE
