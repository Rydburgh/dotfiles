%%%\documentclass[a4paper,11pt,twoside]{report}

%%%% ____  ____  _____    _    __  __ ____  _     _____
%|  _ \|  _ \| ____|  / \  |  \/  | __ )| |   | ____|
%| |_) | |_) |  _|   / _ \ | |\/| |  _ \| |   |  _|
%|  __/|  _ <| |___ / ___ \| |  | | |_) | |___| |___
%|_|   |_| \_\_____/_/   \_\_|  |_|____/|_____|_____|
%
% last updated: 2020-05-21
%%%%%%%%%%%%%%%%%%%%%%%%%%%%%%%%%%%%%%%%%%%%%%%%%%%%%%%%%%%%%%%%%%%%%%%%%%%%%%%%%%%%%%%%%%
%%% PACKAGES
%%%%%%%%%%%%%%%%%%%%%%%%%%%%%%%%%%%%%%%%%%%%%%%%%%%%%%%%%%%%%%%%%%%%%%%%%%%%%%%%%%%%%%%%%%


\pdfminorversion=7						% to prevent errors when building pdf

% some basic packages
\usepackage{amsmath, amsthm, amssymb}
\usepackage{mhchem}						% for chemical symbols
\usepackage{url}						% to format hyperlink text
\usepackage{float}						% for custom figure/table environment
\usepackage{xifthen}					% to handle tests
\usepackage{booktabs}					% for table commands and optimisation
\usepackage{enumitem}					% to format enumerate, itemize, and description
\usepackage{textcomp}					% to support different glyphs
\usepackage{graphicx}					% to support \includegraphics
\usepackage[T1]{fontenc}				% for unicode encoding
\usepackage[utf8]{inputenc}				% for unicode input
\setlength{\headheight}{13.6pt}
\usepackage[top=1.5in,bottom=1in,right=1in,left=1in,headheight=45pt]{geometry}
\usepackage{fancyhdr}					% for adding different headers
\pagestyle{fancy}
% for list of equations
\usepackage{tocloft}					% for custom lists
\usepackage{ragged2e} 					% to undo \centering
\usepackage{hyperref} 					% to make references hyperlinks
\usepackage{glossaries}

% for figures
\usepackage{import}						% for file control
\usepackage{pdfpages}					% for pdf, graphics, and hypertext
\usepackage{transparent}				% for color stack transparency
\usepackage{xcolor}						% for arbitrary color mixing



\author{Jasper Runco}
\date{2020 // Fall}

%%%%%%%%%%%%%%%%%%%%%%%%%%%%%%%%%%%%%%%%%%%%%%%%%%%%%%%%%%%%%%%%%%%%%%%%%%%%%%%%%%%%
% Commands
%%%%%%%%%%%%%%%%%%%%%%%%%%%%%%%%%%%%%%%%%%%%%%%%%%%%%%%%%%%%%%%%%%%%%%%%%%%%%%%%%%%%

% to make a new figure
\newcommand{\incfig}[2][scale=1]{%
	% \def\svgwidth{#1\columnwidth}
	\import{./figures/}{#2.pdf_tex}
}
\pdfsuppresswarningpagegroup=1

% define list of equations
\newcommand{\listequationsname}{\Large{List of Equations}}
\newlistof{myequations}{equ}{\listequationsname}
\newcommand{\myequations}[1]{
	\phantomsection
	\addcontentsline{equ}{myequations}{\protect\numberline{\theequation}#1}
}

\setlength{\cftmyequationsnumwidth}{2.3em}
\setlength{\cftmyequationsindent}{1.5em}

% command to box, label, reference, and include
% noteworthy equations in list of equations
\newcommand{\noteworthy}[2]{
\begin{align} \label{#2} \ensuremath{\boxed{#1}} \end{align}
\myequations{#2} \centering \textit{#2} \justify}

\newtheorem{definition}{Definition}
\newtheorem{theorem}{Theorem}
\newtheorem{lemma}{Lemma}
\newtheorem{corollary}{Corollary}
\newtheorem{example}{Example}
\newtheorem{solution}{Solution}
\newtheorem{constant}{Constant}
\newtheorem{note}{Note}


%%%\begin{document}

 \chapter{Gauss's Law}%
 \label{cha:gauss_s_law}


\LARGE\textsc{Date: 2020-08-31} \\ \\ \LARGE\textsc{Announcements:} \\
\small
Work on homework, reading quiz, and Lab 2


\paragraph \hrule \paragraph \\ \fancyhead[R]{Lesson 3} \fancyhead[L]{Week 3}
%  %  %  %  %  %  %  %  %  %  %  %  %  %  %  %  %  %  %  %  %  %  %  %  %  %  %  %

\noteworthy{
	\Phi _{E} = \oint \overline{E} \cdot d\overline{A} = Q_{encl}\frac{Q_{encl}}{\epsilon_{0}}
}{Flux Equation}

\begin{description}
	\item[$\Phi_{E}$ -] Electric flux through a closed surface of area A = surface integral of $\overline{E}$
	\item[$Q_{encl}$ -] Total charge enclosed by surface
	\item[$\epsilon_{0}$ -] Electric constant
\end{description}

\begin{example}[Simplest case: Spherical symmetry]
	Use Gauss's Law to calculate electric field due to a point charge \[\overline{E} = \frac{kq}{r^2}\hat{r}\]
\end{example}

\begin{solution}[]
	$\left| E \right| $ same at all points on Gaussian surface

	\[
		\oint \overline{E}\cdot d\overline{A} = \frac{q_{enc}}{\epsilon_{0}}
	\]

	\[
		\oint E \cos \phi  d\overline{A} = \frac{q_{enc}}{\epsilon_{0}}
	\]

	\[
		d \overline{A}
	\]

	points normal and out from Gaussian surface


	Because $\overline{E}$ and $d \overline{A}$ are both normal to the round surface, their angle is 0

	\[
		\oint E dA = \frac{q}{\epsilon_0}
	\]

	E constant on Gaussian surface
	\[
		E\oint  dA = \frac{q}{\epsilon_0}
	\]
	\[
		E 4\pi r^2 = \frac{q}{\epsilon_0}
	\]
	\[
	E = \frac{1}{4\pi \epsilon_0} \frac{q}{r^2} = \frac{kq}{r^2}\] radially outward.
\end{solution}

\subsection{Applications}%
\label{sub:applications}


\paragraph{Conductor}

\begin{itemize}
	\item Suppose we construct a Gaussian surface inside a conductor.
	\item Because $\overline{E}=0$ everywhere on the surface, Gauss's law
		requires the net charge inside the surface be zero.
	\item Under \textbf{electrostatic} conditions (charges not in motion), any
		excess charge on a solid conductor resides entirely on the conductor's surface.
	\item \textbf{Therefore, the electric field inside a conductor is always zero}
\end{itemize}
\paragraph{Conductor with cavity}


\begin{itemize}
	\item Cavity inside a conductor
	\item If a charge of $1nC$ resides inside the cavity, and the field inside the conductor
		is always zero, then there must be a $-1nC$ charge distributed around the surface
		of the cavity.
\end{itemize}

\paragraph{Review}
\begin{itemize}
	\item Electric field due to a point charge: $\overline{E} = \frac{kq}{r^2}\hat{r}$
	\item Electric field due to an infinite line of charge: $\overline{E} = \frac{\lambda}{2\pi \epsilon_0 r} \hat{r}$
		($\lambda $ linear charge density)
	\item Electric field due to an infinite sheet of charge: $\overline{E} = \frac{\sigma}{2 \epsilon_0}$ uniform field
		($\sigma$ surface charge density)
\end{itemize}

\paragraph{Field of a charged conducting sphere}
\begin{itemize}
	\item Positive charge Q is distributed on spherical conductor with radius R. Find magnitude of
		electric field at point P and distance r from the center of the sphere.
	\item for $r<R$, we have symetry so we chose a Gaussian surface that is spherical with radus r.
		\[ E = 0\]
	\item for r>R, we construct a Gaussian surface enclosing the charged sphere.
		\[\oint \overline{E} \cdot d\overline{A} = \frac{q_{enc}}{\epsilon_0}\]
		\[\oint E dA \cos \phi = q_{enc}/\epsilon_0\]
		\[\oint E dA \cos 0 = q_{enc}/\epsilon_0\]
		\[\oint E dA  = Q/\epsilon_0\]
		\[E\oint  dA  = Q/\epsilon_0\]
		\[E 4\pi r^2  = Q/\epsilon_0\]
		\[E = Q/4\pi r^2  \epsilon_0\]
		\[E = kQ/ r^2  \]
	\item The result is the same as for a point charge.
\end{itemize}

\paragraph{Field of uniformly charged sphere (insulator)}
\begin{itemize}
	\item Positive charge Q uniformly throughtout volume of \textbf{insulating} sphere with radius R.
		Find magnitude of field at point P a distance r from center.

		\[ \rho_{\:\text{sphere}\:} = \:\text{folume charge density, charge/volume}\:\]
	\item r<R
		\[\oint \overline{E} \cdot d\overline{A} = q_{enc}/\epsilon_0\]
	\item The symmetry says the E-field points out and is constant across a gaussian surface.
		\[\oint E d A \cos \phi = q_{enc}/\epsilon_0\]
		\[\oint E d A = q_{enc}/\epsilon_0\]
		\[E\oint  d A = q_{enc}/\epsilon_0\]
		\[E 4 \pi r^2 = q_{enc}/\epsilon_0\]
		\[E 4 \pi r^2 = q_{enc}/\epsilon_0\]
		\item the question: what is $q_{enc}$?
			\[q_{enc} = \rho V_{enc} = \frac{Q}{\frac{4}{3}\pi R^3}\frac{4}{3}\pi r^3\]
			\[q_{enc} = \frac{Qr^3}{R^3}\]

		\[E = \frac{1}{4 \pi r^2 \epsilon_0 }\frac{Qr}{R^3}\:\text{radially out}\:\]
	\item Now, what is the field outside?
		\[\oint \overline{E} \cdot d \overline{A} = q_{enc}/\epsilon_0 \]
		\[E 4 \pi r^2 = \frac{Q}{\epsilon_0}\]
		\[E = \frac{kQ}{r^2}\:\text{radially out}\:\]
\end{itemize}

		\paragraph{Field of a Uniform Line Charge}

		\begin{itemize}
		\item We need to chose a Gaussian surface that is symetrical, so use a \textbf{cylinder}
		\[ \oint \overline{E} \cdot d \overline{A} = q_{enc}/\epsilon_0\]
		\item This has three components, the end caps and the round
			\[\int_{end} \overline{E} d \overline{A} +\int_{end} \overline{E} d \overline{A} +\int_{round} \overline{E} d \overline{A} \]
		\item For the end caps, the field lines perpendicular to the area, so they contribute nothing to the flux

			\[\int E dA \cos \phi  = q_{enc}/\epsilon_0\]
			\[E\int  dA = q_{enc}/\epsilon_0\]
			\[q_{enc} = \lambda l\]
			\[E \cdot 2\pi r l = \lambda \frac{l}{\epsilon_0}\]

		\end{itemize}
		\paragraph{Infinite plane of charge}
		\begin{itemize}
			\item planar symmetry
				\[ \sigma = \:\text{charge}\:/\:\text{area}\:\]
				\item the Gaussian surface that takes advantage of this symmetry is a cylinder bisecting the plane
					\item cylinder: endcaps area A
						\[\oint  \overline{E} \cdot d \overline{A} = q_{enc}/\epsilon_0\]
						\[\int_{end}  \overline{E} \cdot d \overline{A}+ \int_{end}  \overline{E} \cdot d \overline{A} + \int_{round}  \overline{E} \cdot d \overline{A} = q_{enc}/\epsilon_0\]
						\[\int  \overline{E} \cdot d \overline{A} \cos \phi   + \int E dA \cos \phi = q_{enc}/\epsilon_0\]
						\[2\int E dA= q_{enc}/\epsilon_0\]
						\[2E A= \sigma A/\epsilon_0\]
						\[E = \sigma / 2\epsilon_0 \:\text{uniform outward from plane}\:\]
		\end{itemize}

		\subsection{Conductors}%
		\label{sub:conductors}

		\noteworthy{
		E_{\perp} = \frac{\sigma}{\epsilon_0}
	}{Electric field at surface of a conductor}

\textbf{Electrostatic shielding}
\begin{itemize}
	\item A conducting box immersed in uniform field
	\item the field of induced charges combines with the uniform field to give \textbf{zero} total field
\end{itemize}









%  %  %  %  %  %  %  %  %  %  %  %  %  %  %  %  %  %  %  %  %  %  %  %  %  %  %  %
\newpage
%%%\end{document}


%LEAVE EMPTY ROW ABOVE THIS ONE
