%%%\documentclass[a4paper,11pt,twoside]{report}

%%%%| |_) | |_) |  _|   / _ \ | |\/| |  _ \| |   |  _|
%|  __/|  _ <| |___ / ___ \| |  | | |_) | |___| |___
%|_|   |_| \_\_____/_/   \_\_|  |_|____/|_____|_____|
% Jasper Runco
% last updated: 2020-05-20
%%%%%%%%%%%%%%%%%%%%%%%%%%%%%%%%%%%%%%%%%%%%%%%%%%%%%%%%%%%%%%%%%%%%%%%%%%%%%%%%%%%%%%%%%%
%%% PACKAGES
%%%%%%%%%%%%%%%%%%%%%%%%%%%%%%%%%%%%%%%%%%%%%%%%%%%%%%%%%%%%%%%%%%%%%%%%%%%%%%%%%%%%%%%%%%
\pdfminorversion=7						% to prevent errors when building pdf
% some basic packages
\usepackage{amsthm, amsmath}
\usepackage{url}						% to format hyperlink text
\usepackage{float}						% for custom figure/table environment
\usepackage{xifthen}					% to handle tests
\usepackage{booktabs}					% for table commands and optimisation
\usepackage{enumitem}					% to format enumerate, itemize, and description
\usepackage{textcomp}					% to support different glyphs
\usepackage{graphicx}					% to support \includegraphics
\usepackage[T1]{fontenc}				% for unicode encoding
\usepackage[utf8]{inputenc}				% for unicode input
\setlength{\headheight}{13.6pt}
\usepackage[top=1.5in,bottom=1in,right=1in,left=1in,headheight=45pt]{geometry}
\usepackage{fancyhdr}					% for adding different headers
\pagestyle{fancy}
% for list of equations
\usepackage{tocloft}					% for custom lists
\usepackage{ragged2e} 					% to undo \centering
\usepackage{hyperref} 					% to make references hyperlinks
\usepackage{glossaries}
% for figures
\usepackage{import}						% for file control
\usepackage{pdfpages}					% for pdf, graphics, and hypertext
\usepackage{transparent}				% for color stack transparency
\usepackage{xcolor}						% for arbitrary color mixing
%%%%%%%%%%%%%%%%%%%%%%%%%%%%%%%%%%%%%%%%%%%%%%%%%%%%%%%%%%%%%%%%%%%%%%%%%%%%%%%%%%%%
% Commands
%%%%%%%%%%%%%%%%%%%%%%%%%%%%%%%%%%%%%%%%%%%%%%%%%%%%%%%%%%%%%%%%%%%%%%%%%%%%%%%%%%%%
% to make a new figure
\newcommand{\incfig}[2][1]{%
	\def\svgwidth{#1\columnwidth}
	\import{./figures/}{#2.pdf_tex}
}
\pdfsuppresswarningpagegroup=1
% define list of equations
\newcommand{\listequationsname}{\Large{List of Equations}}
\newlistof{myequations}{equ}{\listequationsname}
\newcommand{\myequations}[1]{
	\addcontentsline{equ}{myequations}{\protect\numberline{\theequation}#1}
}
\setlength{\cftmyequationsnumwidth}{2.3em}
\setlength{\cftmyequationsindent}{1.5em}
% command to box, label, refference, and include
% noteworthy equations in list of equations
\newcommand{\noteworthy}[2]{
\begin{align} \label{#2} \ensuremath{\boxed{#1}} \end{align}
\myequations{#2} \centering \small \textit{#2} \normalsize \justify }
%%%%%%%%%%%%%%%%%%%%%%%%%%%%%%%%%%%%%%%%%%%%%%%%%%%%%%%%%%%%%%%%%%%%%%%%%%%%%%%%%%%%
% Theorems
%%%%%%%%%%%%%%%%%%%%%%%%%%%%%%%%%%%%%%%%%%%%%%%%%%%%%%%%%%%%%%%%%%%%%%%%%%%%%%%%%%%%
\newtheorem{definition}{Definition}
\newtheorem{theorem}{Theorem}
\newtheorem{lemma}{Lemma}
\newtheorem{corollary}{Corollary}


%%%\begin{document}

\chapter{(24) Capacitors and Dielectrics}%
\label{cha:_24_}


\LARGE\textsc{Date: 2020-09-14} \\ \\ \LARGE\textsc{Announcements:} \\
\small

\textbf{Exam: Can start on Tuesday, (4 hour time limit) Due Sunday 11:59 PM (Chapter 21-23) 23 questions - 14 conceptual 9 problem solving
	E field for point charges and charge distributions, relationship between force charge and electric field,
	use gausses law for e field for symmetric charge distribution and find charge distribution, find potential
	due to point charges, relationship between electric potential and electric potential energy, relationship
between potential and electric field}

\textbf{Lab: due Sep 21}

\paragraph \hrule \paragraph \\ \fancyhead[R]{Lesson 5} \fancyhead[L]{Week 5}
%  %  %  %  %  %  %  %  %  %  %  %  %  %  %  %  %  %  %  %  %  %  %  %  %  %  %  %


\section{Capacitors and Capacitance}%
\label{sec:capacitors_and_capacitance}

\begin{definition}[Capacitor].

	\begin{itemize}
		\item Two conductors separated y an insulator (or vacume)
		\item When \textbf{charged}, two conductors have equal magnitude and opposite sign
		\item Zero net charge
	\end{itemize}
\smallskip\hfill$\bullet$\end{definition}

\noteworthy{
	C = \frac{Q}{V_{ab}}
}{Capacitance of a capacitor}

\begin{description}
	\item[Q] Magnitude of charge on each conductor
	\item[$V_{ab}$] Potential difference between $+Q$ and $-Q$
\end{description}

\noteworthy{
	F = \frac{C}{V}
}{Farad - Unit of Capacitance}

\begin{definition}[Parallel plate capacitors].

	\begin{itemize}
		\item Two parallel conducting plates separated by a small distance compared to dimensions.
		\item Uniform field and charge distribution over opposing surfaces.
	\end{itemize}
\smallskip\hfill$\bullet$\end{definition}


\noteworthy{
	C = \frac{Q}{V_{ab}} = \epsilon_0 \frac{A}{d}
}{Capacitance of Parallel Plates}

\begin{description}
	\item[Q] Magnitude of charge on each plate
	\item[A] Area of each plate
	\item[d] Distance between plates
\end{description}

.

The Electric field of parallel plates:
\begin{align*}
	\sigma &=   \:\text{charge}\:/\:\text{area}\:\\
	E &= \frac{\sigma}{\epsilon_0} \\
\end{align*}

\section{Capacitors in Series and Parallel}%
\label{sec:capacitors_in_series_and_parallel}


\subsection{Capacitors in Series}%
\label{sub:capacitors_in_series}

\begin{itemize}
	\item The capacitors have the same charge Q.
	\item Their potential differences add:
		\[V_{ac}+ V_{c b} = V_{ab}\]
\end{itemize}

\textbf{Finding Equivalent Capacitance: }

The potential drop across both capacitors must be the sum of the potential drop across each.
The charge on $C_1$ and $C_2$ must be the same for the individual capacitors.

\begin{align*}
	C &= \frac{Q}{V} \\
	Q_1 &=  Q_2 \\
	V_{ab} &=   V_1 + V_2\\
	V &=  \frac{Q_1}{C_1}+ \frac{Q_2}{C_2} \\
	V &= Q\left(  \frac{1}{C_1} + \frac{1}{C_2} \right) \\
	\frac{V}{Q} &=  \frac{1}{C_1} + \frac{1}{C_2} \\
	\frac{1}{C_{eq}} &=  \frac{1}{C_1} + \frac{1}{C_2} \\
\end{align*}
\noteworthy{
	\frac{1}{C_{eq}}=  \frac{1}{C_1} + \frac{1}{C_2}
}{Series Equivalent Capacitance}

\subsection{Capacitors in Parallel}%
\label{sub:capacitors_in_parallel}

\begin{itemize}
	\item Have same potential drop $V$.
	\item Charge on each depends on its capacitance:
		\[Q_1 = C_1V, Q_2 = C_2V\]
\end{itemize}

\textbf{Finding Equivalent Capacitance: }

The potential drop across both capacitors must be the same. The charge on $C_1$ and $C_2$ must
sum to the total charge.

\begin{align*}
	C &= \frac{Q}{V}\\
	Q &=Q_1 + Q_2  \\
	V_1 &= V_2 \\
	Q &= C_1V_1 + C_2V_2 \\
	C_{eq}V &= C_1V + C_2V \\
	C_{eq} &= C_1 + C_2 \\
\end{align*}

\noteworthy{
	C_{eq} = C_1 + C_2
}{Parallel Equivalent Capacitance}

\section{Energy Storage in Capacitors and Electric-Field Energy}%
\label{sec:energy_storage_in_capacitors_and_electric_field_energy}


\subsection{Energy Stored in a Capacitor}%
\label{sub:energy_stored_in_a_capacitor}


\noteworthy{
	U = \frac{Q^2}{2C} = \frac{1}{2} C V^2 = \frac{1}{2}QV
}{Capacitor Potential Energy}

\begin{description}
	\item[U] Potential energy stored in a capacitor
	\item[Q] Magnitude of charge on each plate
	\item[C] Capacitance
	\item[V] Potential difference between plates
\end{description}

\subsubsection{Derivation of Capacitor Potential Energy}%
\label{ssub:derivation_of_capacitor_potential_energy}

Energy stored in capacitor? $\to$ Calculate work done to charge capacitor.

$V = \frac{Q}{C}$ (Final charge is Q and final potential is V after charging)

$v = \frac{q}{C}$ (v and q at some intermediate time)

\begin{align*}
	W &=  \Delta u = Q\Delta V \\
	dW &=  v dq \\
	\int dW &=  \int_{0}^{Q} \frac{q}{C}dq  \\
W &=  \frac{q^2}{2C}\biggr\rvert_{0}^{Q}   \\
W &= \frac{Q^2}{2C} \\
\end{align*}

\subsection{Electric-Field Energy}%
\label{sub:electric_field_energy}

energy density $= \frac{\:\text{energy}\:}{\:\text{volume}\:} = \frac{\frac{1}{2}CV^2}{Ad} = \frac{\frac{1}{2}\frac{\epsilon_0 A}{d}}{Ad}=$

\noteworthy{
	u = \frac{1}{2}\epsilon_0 E^2
}{Electrical Density in Vacuum}

\begin{description}
	\item[u] Energy density for a vacuum
	\item[E] Magnitude of electric field
\end{description}


\section{Dielectrics}%
\label{sec:dielectrics}

Most capacitors have noncuducting material between plates, such as Mylar.

\subsubsection{Increasing Capacitance}%
\label{ssub:increasing_capacitance}

\begin{itemize}
	\item Connect electrometer across a charged capacitor, with magnitude Q on each and potential
		difference $V_0$
	\item An uncharged sheet of dielectric between the plates, the potential difference decreases
		to smaller V value.
	\item Since Q is unchanged, capacitance $C = \frac{Q}{V}$ \textbf{increased}.
	\item The field decreases due to \textbf{polarization} within the dielectric,
		induced surface charges.
		\[E = \frac{E_0}{K}\]
		\begin{description}
			\item[E] Decreased E-field
		\end{description}
		\begin{description}
			\item[K] dielectric constant, unitless
		\end{description}
		\[V = Ed\]
		\[\implies V = \frac{V_0}{K}\]
		\[\implies C = KC_0\]
\end{itemize}

\noteworthy{
\epsilon = K \epsilon_0
}{Permitivity of a Dielectric}
		\begin{description}
			\item[K] dielectric constant, unitless
		\end{description}


\noteworthy{
C = K C_0 = K \epsilon_0 \frac{A}{d = \epsilon \frac{A}{d}}
}{Capacitance of parallel-plate capacotor, dielectric between plates}


\noteworthy{
u = \frac{1}{2}K \epsilon_0 E^2 = \frac{1}{2} \epsilon  E^2
}{Electric energy density in a dielectric}











%  %  %  %  %  %  %  %  %  %  %  %  %  %  %  %  %  %  %  %  %  %  %  %  %  %  %  %
\newpage
%%%\end{document}


%LEAVE EMPTY ROW ABOVE THIS ONE
