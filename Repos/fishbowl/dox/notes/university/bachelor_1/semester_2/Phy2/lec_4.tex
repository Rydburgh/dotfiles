%%%\documentclass[a4paper,11pt,twoside]{report}

%%%% ____  ____  _____    _    __  __ ____  _     _____
%|  _ \|  _ \| ____|  / \  |  \/  | __ )| |   | ____|
%| |_) | |_) |  _|   / _ \ | |\/| |  _ \| |   |  _|
%|  __/|  _ <| |___ / ___ \| |  | | |_) | |___| |___
%|_|   |_| \_\_____/_/   \_\_|  |_|____/|_____|_____|
%
% last updated: 2020-05-21
%%%%%%%%%%%%%%%%%%%%%%%%%%%%%%%%%%%%%%%%%%%%%%%%%%%%%%%%%%%%%%%%%%%%%%%%%%%%%%%%%%%%%%%%%%
%%% PACKAGES
%%%%%%%%%%%%%%%%%%%%%%%%%%%%%%%%%%%%%%%%%%%%%%%%%%%%%%%%%%%%%%%%%%%%%%%%%%%%%%%%%%%%%%%%%%


\pdfminorversion=7						% to prevent errors when building pdf

% some basic packages
\usepackage{amsmath, amsthm, amssymb}
\usepackage{mhchem}						% for chemical symbols
\usepackage{url}						% to format hyperlink text
\usepackage{float}						% for custom figure/table environment
\usepackage{xifthen}					% to handle tests
\usepackage{booktabs}					% for table commands and optimisation
\usepackage{enumitem}					% to format enumerate, itemize, and description
\usepackage{textcomp}					% to support different glyphs
\usepackage{graphicx}					% to support \includegraphics
\usepackage[T1]{fontenc}				% for unicode encoding
\usepackage[utf8]{inputenc}				% for unicode input
\setlength{\headheight}{13.6pt}
\usepackage[top=1.5in,bottom=1in,right=1in,left=1in,headheight=45pt]{geometry}
\usepackage{fancyhdr}					% for adding different headers
\pagestyle{fancy}
% for list of equations
\usepackage{tocloft}					% for custom lists
\usepackage{ragged2e} 					% to undo \centering
\usepackage{hyperref} 					% to make references hyperlinks
\usepackage{glossaries}

% for figures
\usepackage{import}						% for file control
\usepackage{pdfpages}					% for pdf, graphics, and hypertext
\usepackage{transparent}				% for color stack transparency
\usepackage{xcolor}						% for arbitrary color mixing



\author{Jasper Runco}
\date{2020 // Fall}

%%%%%%%%%%%%%%%%%%%%%%%%%%%%%%%%%%%%%%%%%%%%%%%%%%%%%%%%%%%%%%%%%%%%%%%%%%%%%%%%%%%%
% Commands
%%%%%%%%%%%%%%%%%%%%%%%%%%%%%%%%%%%%%%%%%%%%%%%%%%%%%%%%%%%%%%%%%%%%%%%%%%%%%%%%%%%%

% to make a new figure
\newcommand{\incfig}[2][scale=1]{%
	% \def\svgwidth{#1\columnwidth}
	\import{./figures/}{#2.pdf_tex}
}
\pdfsuppresswarningpagegroup=1

% define list of equations
\newcommand{\listequationsname}{\Large{List of Equations}}
\newlistof{myequations}{equ}{\listequationsname}
\newcommand{\myequations}[1]{
	\phantomsection
	\addcontentsline{equ}{myequations}{\protect\numberline{\theequation}#1}
}

\setlength{\cftmyequationsnumwidth}{2.3em}
\setlength{\cftmyequationsindent}{1.5em}

% command to box, label, reference, and include
% noteworthy equations in list of equations
\newcommand{\noteworthy}[2]{
\begin{align} \label{#2} \ensuremath{\boxed{#1}} \end{align}
\myequations{#2} \centering \textit{#2} \justify}

\newtheorem{definition}{Definition}
\newtheorem{theorem}{Theorem}
\newtheorem{lemma}{Lemma}
\newtheorem{corollary}{Corollary}
\newtheorem{example}{Example}
\newtheorem{solution}{Solution}
\newtheorem{constant}{Constant}
\newtheorem{note}{Note}


%%%\begin{document}

\chapter{(23) Electrical Potential}%
\label{cha:_23_electrical_potential}


\LARGE\textsc{Date: 2020-09-07} \\ \\ \LARGE\textsc{Announcements:} \\
\small



\paragraph \hrule \paragraph \\ \fancyhead[R]{Lesson 4} \fancyhead[L]{Week 4}
%  %  %  %  %  %  %  %  %  %  %  %  %  %  %  %  %  %  %  %  %  %  %  %  %  %  %  %

\section{Electric Potential Energy}%
\label{sec:electric_potential_energy}

\subsection{Electrical Potential Energy in a Uniform field}%
\label{sub:electrical_potential_energy_in_a_uniform_field}

\begin{itemize}
	\item If a positive charge moves in the direction of the field, the field does \textbf{positive}
		work on the charge
	\item The potential energy \textbf{decreases}.
	\item If the positive charge moves opposite the direction of the field, the field
		does \textbf{negative} work on the charge.
	\item The potential energy increases.
	\item If a negative charge moves in the direction of the field, the field
		does \textbf{negative} work on the charge.
	\item The potential energy \textbf{increases}.
	\item If the negative charge moves opposite the direction of the field, the
		field does \textbf{positive} work on the charge.
	\item The potential energy \textbf{decreases}.
\end{itemize}

\subsubsection{Electric Potential Energy of Two Point Charges}%
\label{ssub:electric_potential_energy_of_two_point_charges}

\begin{itemize}
	\item Doesn't depend on path taken.
	\item Electric potential energy only depends on the distance between the charges.
	\item Defined to be zero when the charges are infinitely far apart.
	\item Charges with the same sign have positive electric potential energy.
	\item Charges with opposite signs have negative electric potential energy.
\end{itemize}


\noteworthy{
U = \frac{1}{4\pi \epsilon_0} \frac{qq_0}{r}
}{Electric potenetial energy of two point charges}

\begin{description}
	\item[U] Potential Energy
	\item[$\epsilon_0$] Electric constant
	\item[$q,q_0$] Values of two charges
	\item[r] Distance between two charges
\end{description}

\subsubsection{Electrical Potential with Several Point Charges}%
\label{ssub:electrical_potential_with_several_point_charges}

\begin{itemize}
	\item The potential energy with $q_0$ depends on the other charges and their distances.
		\item Electric potential energy is \textbf{Algebraic sum}.
\end{itemize}

\noteworthy{
U = \frac{q_0}{4\pi \epsilon_0}\left( \frac{q_1}{r_1}+\frac{q_2}{r_2}+\frac{q_3}{r_3}+\ldots \right) =\frac{q_0}{4\pi \epsilon_0}\sum_{i} q_\frac{i}{r_i}
}{Electric potential  energy $q_0$ due to a collection of charges}

\section{Electric Potential}%
\label{sec:electric_potential}

\begin{definition}[Electric Potential]
	Potential is potential energy per unit charge. The potential of a with respect to b $(V_{ab} = V_{a} - V_{b} $
	 equals the work done by the electric force when a unit charge moves from a to b.
\smallskip\hfill$\bullet$\end{definition}

\noteworthy{
V = \frac{1}{4\pi \epsilon_0}\frac{q}{r}
}{Electric potential due to a point charge}

\begin{description}
	\item[q] value of point charge.
		\item[r] distance from point charge to where potential is measured.
\end{description}

\noteworthy{
	V = \frac{1}{4\pi \epsilon_0} \sum_{i}^{} \frac{q_{i}}{r_{i}}
}{Electric potential due to a collection of point charges}

\subsection{Electric Potential and Electric Field}%
\label{sub:electric_potential_and_electric_field}

\begin{itemize}
	\item Moving in the direction of the electric field, the electric potential \textbf{decreases}.
	\item The direction of the electric field is the direction of decreasing V.
	\item To move against the E-field, an external force per unit charge must be applied opposite the
		electric force per unit charge.
	\item The electric force per unit charge is the E-field.
		\item The \textbf{potential difference} $V_{a}-V_{b}$ equals the work done per unit
			charge by the external force to move from b to a:
			\[V_{a} - V_{b} = -\int_{a}^{b} \overline{E} \cdot d\overline{r} \]
		\item Electric field can be expressed as  $1 \frac{N}{C} = 1 \frac{V}{m}$
\end{itemize}

\subsection{Electron Volt}%
\label{sub:electron_volt}

\begin{itemize}
	\item Change in potential energy U, when a charge moves from a potential of $V_{b}$ to a
		potential of $V_{a}$ is \[U_{a} - U_{b} = q(V_{a} - V_{b}).\]
		\item When the potential difference is 1V for a charge q with magnitude e of the electron
			charge, the change in energy is defined as one electron volt (eV).
\end{itemize}
\noteworthy{
1 eV = 1.602 \times 10^{-19}J
}{Electron volt}

\subsection{Electric Potential and Field of a Charged Conductor}%
\label{sub:electric_potential_and_field_of_a_charged_conductor}

\begin{itemize}
	\item A solid conducting sphere of radius R has a total charge of q.
	\item The electric field \textbf{inside} the sphere is zero everywhere.
		\item The potential is the \textbf{same} at every point inside the sphere
			and is equal to the value at the surface.
\end{itemize}

\begin{figure}[ht]
    \centering
    \incfig{electric-field-in-a-conducting-sphere}
    \caption{Electric field in a conducting sphere}
    \label{fig:electric-field-in-a-conducting-sphere}
\end{figure}


\begin{figure}[ht]
    \centering
    \incfig{electric-potential-in-a-conducting-sphere}
    \caption{Electric potential in a conducting sphere}
    \label{fig:electric-potential-in-a-conducting-sphere}
\end{figure}


\subsubsection{Ionization and Corona Discharge}%
\label{ssub:ionization_and_corona_discharge}

\begin{itemize}
	\item Air becomes ionized (conductor) at or above $3\times 10^{6} \frac{V}{m}$.
		\item For a charged sphere, $V_{\:\text{surface}\: }= E_{\:\text{surface}\:}R$
		\item If $E_{m}$ is the E-field magnitude at which air becomes conductive
			\textbf{(dielectric strength)}, then $V_{m}$ is the maximum potential to
			which it can be raised: $V_{m} = RE_{m}$.
\end{itemize}

\subsubsection{Oppositely Charged Parallel Plates}%
\label{ssub:oppositely_charged_parallel_plates}

\noteworthy{
V = Ey
}{Potential at any height between two large oppositely charged parallel plates}
\begin{description}
	\item[V] Potential (units $V = \frac{J}{C}$)
	\item[E] Electric field magnitude (units $\frac{N}{C}$)
	\item[y] height (m)
\end{description}

\begin{figure}[ht]
    \centering
    \incfig{potential-between-parallel-plates}
    \caption{Potential between parallel plates}
    \label{fig:potential-between-parallel-plates}
\end{figure}

\section{Equipotential Surfaces}%
\label{sec:equipotential_surfaces}

\subsection{Equipotential Surfaces and Field lines}%
\label{sub:equipotential_surfaces_and_field_lines}

\begin{itemize}
	\item \textbf{Equipotential surfaces} have constant electric potential.
		\item Field lines and equipotential surfaces are always mutually perpendicular.
\end{itemize}

\begin{figure}[ht]
    \centering
    \incfig{equipotential-surfaces-for-monopole}
    \caption{Equipotential surfaces for monopole}
    \label{fig:equipotential-surfaces-for-monopole}
\end{figure}


\begin{figure}[ht]
    \centering
    \incfig{equipotential-surfaces-for-dipole}
    \caption{Equipotential surfaces for dipole}
    \label{fig:equipotential-surfaces-for-dipole}
\end{figure}

\begin{figure}[ht]
    \centering
    \incfig{equipotential-surfaces-for-two-equal-charges}
    \caption{Equipotential surfaces for two equal charges}
    \label{fig:equipotential-surfaces-for-two-equal-charges}
\end{figure}


\subsubsection{Equipotentials and Conductors}%
\label{ssub:equipotentials_and_conductors}

\begin{itemize}
	\item When all charges are at rest
		\begin{itemize}
			\item The surface of a conductor is always an equipotential surface
			\item The electric field just outside a conductor is always perpendicular to the surface.
		\end{itemize}
	\item If the electric field had a tangential component at the surface of a conductor, a
		net amount of work would be done on a test charge by moving it around a loop, which
		is impossible because the E-force is conservative.
\end{itemize}

\begin{figure}[ht]
    \centering
    \incfig{equipotential-surface-and-conductor}
    \caption{Equipotential surface and conductor}
    \label{fig:equipotential-surface-and-conductor}
\end{figure}

\subsection{Potential Gradient}%
\label{sub:potential_gradient}

\begin{itemize}
	\item The components of the electric field can be found by partial derivatives of the electric potential.
	\item The electric field is the negative gradient of the potential.
\end{itemize}

\noteworthy{
E_{x} = -\frac{\partial V}{\partial x},
E_{y} = -\frac{\partial V}{\partial y},
E_{z} = -\frac{\partial V}{\partial z}
}{Components of electric field}

\noteworthy{
\overline{E} = -\Delta \overline{V}
}{Electric field, negative gradient of potential}




%  %  %  %  %  %  %  %  %  %  %  %  %  %  %  %  %  %  %  %  %  %  %  %  %  %  %  %
\newpage
%%%\end{document}


%LEAVE EMPTY ROW ABOVE THIS ONE
