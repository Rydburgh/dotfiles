%%%\documentclass[a4paper,11pt,twoside]{report}

%%%% ____  ____  _____    _    __  __ ____  _     _____
%|  _ \|  _ \| ____|  / \  |  \/  | __ )| |   | ____|
%| |_) | |_) |  _|   / _ \ | |\/| |  _ \| |   |  _|
%|  __/|  _ <| |___ / ___ \| |  | | |_) | |___| |___
%|_|   |_| \_\_____/_/   \_\_|  |_|____/|_____|_____|
%
% last updated: 2020-05-21
%%%%%%%%%%%%%%%%%%%%%%%%%%%%%%%%%%%%%%%%%%%%%%%%%%%%%%%%%%%%%%%%%%%%%%%%%%%%%%%%%%%%%%%%%%
%%% PACKAGES
%%%%%%%%%%%%%%%%%%%%%%%%%%%%%%%%%%%%%%%%%%%%%%%%%%%%%%%%%%%%%%%%%%%%%%%%%%%%%%%%%%%%%%%%%%


\pdfminorversion=7						% to prevent errors when building pdf

% some basic packages
\usepackage{amsmath, amsthm, amssymb}
\usepackage{mhchem}						% for chemical symbols
\usepackage{url}						% to format hyperlink text
\usepackage{float}						% for custom figure/table environment
\usepackage{xifthen}					% to handle tests
\usepackage{booktabs}					% for table commands and optimisation
\usepackage{enumitem}					% to format enumerate, itemize, and description
\usepackage{textcomp}					% to support different glyphs
\usepackage{graphicx}					% to support \includegraphics
\usepackage[T1]{fontenc}				% for unicode encoding
\usepackage[utf8]{inputenc}				% for unicode input
\setlength{\headheight}{13.6pt}
\usepackage[top=1.5in,bottom=1in,right=1in,left=1in,headheight=45pt]{geometry}
\usepackage{fancyhdr}					% for adding different headers
\pagestyle{fancy}
% for list of equations
\usepackage{tocloft}					% for custom lists
\usepackage{ragged2e} 					% to undo \centering
\usepackage{hyperref} 					% to make references hyperlinks
\usepackage{glossaries}

% for figures
\usepackage{import}						% for file control
\usepackage{pdfpages}					% for pdf, graphics, and hypertext
\usepackage{transparent}				% for color stack transparency
\usepackage{xcolor}						% for arbitrary color mixing



\author{Jasper Runco}
\date{2020 // Fall}

%%%%%%%%%%%%%%%%%%%%%%%%%%%%%%%%%%%%%%%%%%%%%%%%%%%%%%%%%%%%%%%%%%%%%%%%%%%%%%%%%%%%
% Commands
%%%%%%%%%%%%%%%%%%%%%%%%%%%%%%%%%%%%%%%%%%%%%%%%%%%%%%%%%%%%%%%%%%%%%%%%%%%%%%%%%%%%

% to make a new figure
\newcommand{\incfig}[2][scale=1]{%
	% \def\svgwidth{#1\columnwidth}
	\import{./figures/}{#2.pdf_tex}
}
\pdfsuppresswarningpagegroup=1

% define list of equations
\newcommand{\listequationsname}{\Large{List of Equations}}
\newlistof{myequations}{equ}{\listequationsname}
\newcommand{\myequations}[1]{
	\phantomsection
	\addcontentsline{equ}{myequations}{\protect\numberline{\theequation}#1}
}

\setlength{\cftmyequationsnumwidth}{2.3em}
\setlength{\cftmyequationsindent}{1.5em}

% command to box, label, reference, and include
% noteworthy equations in list of equations
\newcommand{\noteworthy}[2]{
\begin{align} \label{#2} \ensuremath{\boxed{#1}} \end{align}
\myequations{#2} \centering \textit{#2} \justify}

\newtheorem{definition}{Definition}
\newtheorem{theorem}{Theorem}
\newtheorem{lemma}{Lemma}
\newtheorem{corollary}{Corollary}
\newtheorem{example}{Example}
\newtheorem{solution}{Solution}
\newtheorem{constant}{Constant}
\newtheorem{note}{Note}



%%%\begin{document}

\chapter{Electric Fields Cont'd}%
\label{cha:electric_fields_cont_d}


\LARGE\textsc{Date: 2020-08-24} \\ \\ \LARGE\textsc{Announcements:} \\
\small

\paragraph \hrule \paragraph \\ \fancyhead[R]{Lesson 2} \fancyhead[L]{Week 2}
%  %  %  %  %  %  %  %  %  %  %  %  %  %  %  %  %  %  %  %  %  %  %  %  %  %  %  %

\section{Electric field lines}%
\label{sec:electric_field_lines}

\begin{definition}[Electric field line]
	An imaginary line or curve whose tangent at any point is the direction of the electric field vector
	at that point.
\end{definition}

% \begin{figure}[ht]
%     \centering
%     \incfig{electric-field-line}
%     \caption{electric field line}
%     \label{fig:electric-field-line}
% \end{figure}

\begin{itemize}
	\item  Field lines always start at a positive charge and end at a negative charge and can never cross.
	\item Show the direction of the electric field at each point.
	\item Spacing gives idea of magnitude of field at each point.
\end{itemize}

\subsection{Electric field lines of a dipole}%
\label{sub:electric_field_lines_of_a_dipole}

\begin{definition}[Electric dipole]
	A pair of equal and opposite electric charges.
\end{definition}

\begin{itemize}
	\item The field radiates outward from the positive and towards the negative.
\end{itemize}

\begin{figure}[ht]
	\centering
	\incfig{dipole}
	\caption{Dipole}
	\label{fig:dipole}
\end{figure}

\begin{example}[1]
	A positive point charge +Q is released from rest in an electric field. At any later time, the velocity
	of the point charge

	A. in the direction of the electric field?
	B. opposite the direction of the electric field?
	C. not enough information.
\end{example}
\begin{solution}[1]
	The question does not state a uniform electric field, and remember field lines are not trajectories. C.
\end{solution}

\begin{example}[2]
	What happens when I place an electric dipole in a uniform electric field pointing to the right.

	Net Force? Net Torque?
\end{example}

\begin{solution}[2]
	A.
	The force on the positive point charge is $F_{+} = q E $ right.

	The force on the negative point charge pole is $F_{-} = qE$ left.

	$\implies F_{net} = 0$

	B.
	$ \tau = \overline{r} \times \overline{F} = r F \sin \theta $

	$\tau_{+} = \left( \frac{d}{2} \right) (qE) \sin \theta $ into page

	$\tau_{-}= \overline{r} \times \overline{F} = \left( \frac{d}{2} \right) (qE) \sin \theta$ into page

	$\tau_{net}= qdE \sin \theta $ into page

	\begin{definition}[Electric dipole moment]
		\[\boxed{p = qd}\]
		$\overline{p}$ points from $\boxed{-}$ to $\boxed{+}$

		$\overline{p} \perp \overline{E}$ , max torque

		$\overline{p} \parallel \overline{E}$ , minimum torque
	\end{definition}

\end{solution}

$\tau _{net} = \overline{p} \times  \overline{E}$

\subsection{Electric field due to a charge distribution}%
\label{sub:electric_field_due_to_a_charge_distribution}

\begin{itemize}
	\item line of charge
	\item ring of charge
	\item disk of charge
\end{itemize}

\subsubsection{Line of charge}%
\label{ssub:line_of_charge}

\begin{figure}[ht]
	\centering
	\incfig{line-of-charge}
	\caption{line of charge}
	\label{fig:line-of-charge}
\end{figure}

\begin{example}[3]
	What is the e-field? at point P?
\end{example}

\begin{solution}[3]
	\begin{align*}
		dE &=  \frac{k d Q}{r^2} \\
		\int dE_{y} &= 0 \:\text{(symetry)}\: \\
		dE_{x} &=  \frac{kdQ}{r^2} \cos \alpha \\
		E_{x} &=  \int \frac{kdQ}{r^2} \cos \alpha \\
		r^2 &=  x^2 + y^2 \\
		\cos \alpha &= \frac{x}{r}  = \frac{x}{\sqrt{x^2 + y^2} }\\
		\:\text{define linear charge density}\: \lambda &= \:\text{charge/length}\: \\
		\lambda &= \frac{Q}{2a} \\
		dQ &= \lambda dy \\
		E_{x} &= \int \frac{k \lambda dy x}{(x^2+y^2)\sqrt{x^2+y^2} } \\
			  &= \int_{-a}^{a}   \frac{k \lambda  x dy}{(x^2 + y^2)^{\frac{3}{2}}} \\
		E_{x}&= \boxed{\frac{KQ}{x \sqrt{a^2+x^2} }} \\
	\end{align*}

\end{solution}

\begin{figure}[ht]
	\centering
	\incfig{ring-of-charge}
	\caption{ring of charge}
	\label{fig:ring-of-charge}
\end{figure}
\begin{example}[4]
	Ring of charge
\end{example}

\begin{solution}[4]
	Practice problem.
\end{solution}

\begin{figure}[ht]
	\centering
	\incfig{disk-of-charge}
	\caption{disk of charge}
	\label{fig:disk-of-charge}
\end{figure}

\begin{example}[Disk of charge]

	Find the electric field at point P.
\end{example}

\begin{solution}[Disk of charge]
	\begin{align*}
		dE_{x}&=  \frac{kdQ}{r^2} \cos \theta \\
		r'^2 &= r^2 + x^2 \\
		\cos \theta = \frac{x}{r'} &= \frac{x}{\sqrt{r^2  + x^2} } \\
		\:\text{surface charge density}\: \sigma &=\frac{\:\text{charge}\:}{\:\text{area}\:} \\
		dQ &= \sigma dA = \sigma 2\pi r dr \\
		dA &= 2\pi r dr \\
		\int d E_{x} &= \int_{0}^{R} \frac{k \sigma 2 \pi r dr}{r^2 + x ^2} \frac{x}{\sqrt{r^2 + x^2} }  \\
		E_{x}&= \frac{\sigma}{2 \epsilon_{0}}\left[ 1 - \frac{1}{\sqrt{\frac{R^2}{x^2}+1} } \right]  \\
	\end{align*}
	further: think about R>>x
\end{solution}

\break
\section{Chapter 22: Gauss's Law}%
\label{sec:chapter_22_gauss_s_law}

\subsection{Objectives}%
\label{sub:objectives}

\begin{itemize}
	\item Define electric flux and calculate through surfaces
	\item Define gauss's law
		\begin{itemize}
			\item used to determine charge distribution given known e-field
			\item to determine field given charge distribution
		\end{itemize}
	\item know when Gauss's law can be used
	\item Do calculations with Gauss's law
\end{itemize}

\subsubsection{Electric field due to a charged conducting sphere}%
\label{ssub:electric_field_due_to_a_charged_conducting_sphere}

We could use the electric field equations, but things get messy. Gauss's Law gives simplicity through symetry.

\subsection{Intro to Gauss's law}%
\label{sub:intro_to_gauss_s_law}

\begin{itemize}
	\item Given any general charge distribution, we surround it with an imaginary surface
	\item We look at the field at various points on imaginary surface
	\item GL is a relationship between the field at all points and total charge enclosed. it helps find the field
		for symmetric charge distributions.
\end{itemize}

\subsection{Charge and electric flux}%
\label{sub:charge_and_electric_flux}

In boxes, there are positive charges within producing outward electric flux.

When boxes have negative charge inside, there is an inward electric flux.

When $\overline{E} = 0$ there is no electric flux in or out of the box.

What happens if there is no net charge inside the box?
\begin{itemize}
	\item There is an electric field, but it flows in and out on either half
	\item Thus no net electric flux into or out of box.
\end{itemize}

What happens if there is charge near the box, but not inside?

\begin{itemize}
	\item The net electric flux through the box is zero.
	\end{itemize}\begin{figure}[ht]
	\centering
	\incfig{zero-net-flux}
	\caption{zero net flux}
	\label{fig:zero-net-flux}
\end{figure}

\subsubsection{Quantifying electric flux}%
\label{ssub:quantifying_electric_flux}

Net electric flux is \textbf{directly proportional} to net amount of charge within surface.

Net electric flux is \textbf{independent} of the size of the closed surface.

\begin{example}[]
	Suppose a Gaussian surface with rectangular sides and positive point charge +q at it's center,
	and the surface doubles, but charge remains +q, what happens to the flux?
\end{example}

\begin{solution}[]
	Remains the same
\end{solution}

\begin{example}[]
	Spherical Gaussian surface 1 has +q at it's center. Spherical Gaussian surface 2, same size encloses
	the charge but is not centered on it. Compare flux through surface 1 and 2.
\end{example}

\begin{solution}[]
	The same
\end{solution}

\subsubsection{Calculating electric flux}%
\label{ssub:calculating_electric_flux}

\begin{itemize}
	\item Consider a surface
	\item What affects amount of flux passing through surface?
	\item "net" analogy
\end{itemize}

\begin{definition}[Electric flux]
	\begin{align*}
		\Phi_{E} &= \Sigma \overline{E}_{i}\cdot d \overline{A}_{i} \\
				 &= \int E \cos \phi dA \\
				 &= \int E_{\perp}dA \\
				 &= \int \overline{E} \cdot d\overline{A}\\
				 &\to \int \overline{E} \cdot d \overline{A} \left( \frac{Nm^2}{C} \right) \\
	\end{align*}
	\begin{description}
		\item[$\Phi_{E}$ -] Electric flux through a surface
		\item[$E$ -] Magnitude of Electric Field
		\item[$\phi$ -] Angle between $\overline{E}$ and normal to surface
			\item[$dA$ -] Element of surface area
				\item[$E_{\perp}$ -] Component of $\overline{E}$ perpendicular to surface
					\item[$d\overline{A}$ -] Vector element of surface area
	\end{description}
\end{definition}


%  %  %  %  %  %  %  %  %  %  %  %  %  %  %  %  %  %  %  %  %  %  %  %  %  %  %  %
\newpage
%%%\end{document}


%LEAVE EMPTY ROW ABOVE THIS ONE
