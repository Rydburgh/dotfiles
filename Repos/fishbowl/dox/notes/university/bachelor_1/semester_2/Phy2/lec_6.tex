%%%\documentclass[a4paper,11pt,twoside]{report}

%%%%| |_) | |_) |  _|   / _ \ | |\/| |  _ \| |   |  _|
%|  __/|  _ <| |___ / ___ \| |  | | |_) | |___| |___
%|_|   |_| \_\_____/_/   \_\_|  |_|____/|_____|_____|
% Jasper Runco
% last updated: 2020-05-20
%%%%%%%%%%%%%%%%%%%%%%%%%%%%%%%%%%%%%%%%%%%%%%%%%%%%%%%%%%%%%%%%%%%%%%%%%%%%%%%%%%%%%%%%%%
%%% PACKAGES
%%%%%%%%%%%%%%%%%%%%%%%%%%%%%%%%%%%%%%%%%%%%%%%%%%%%%%%%%%%%%%%%%%%%%%%%%%%%%%%%%%%%%%%%%%
\pdfminorversion=7						% to prevent errors when building pdf
% some basic packages
\usepackage{amsthm, amsmath}
\usepackage{url}						% to format hyperlink text
\usepackage{float}						% for custom figure/table environment
\usepackage{xifthen}					% to handle tests
\usepackage{booktabs}					% for table commands and optimisation
\usepackage{enumitem}					% to format enumerate, itemize, and description
\usepackage{textcomp}					% to support different glyphs
\usepackage{graphicx}					% to support \includegraphics
\usepackage[T1]{fontenc}				% for unicode encoding
\usepackage[utf8]{inputenc}				% for unicode input
\setlength{\headheight}{13.6pt}
\usepackage[top=1.5in,bottom=1in,right=1in,left=1in,headheight=45pt]{geometry}
\usepackage{fancyhdr}					% for adding different headers
\pagestyle{fancy}
% for list of equations
\usepackage{tocloft}					% for custom lists
\usepackage{ragged2e} 					% to undo \centering
\usepackage{hyperref} 					% to make references hyperlinks
\usepackage{glossaries}
% for figures
\usepackage{import}						% for file control
\usepackage{pdfpages}					% for pdf, graphics, and hypertext
\usepackage{transparent}				% for color stack transparency
\usepackage{xcolor}						% for arbitrary color mixing
%%%%%%%%%%%%%%%%%%%%%%%%%%%%%%%%%%%%%%%%%%%%%%%%%%%%%%%%%%%%%%%%%%%%%%%%%%%%%%%%%%%%
% Commands
%%%%%%%%%%%%%%%%%%%%%%%%%%%%%%%%%%%%%%%%%%%%%%%%%%%%%%%%%%%%%%%%%%%%%%%%%%%%%%%%%%%%
% to make a new figure
\newcommand{\incfig}[2][1]{%
	\def\svgwidth{#1\columnwidth}
	\import{./figures/}{#2.pdf_tex}
}
\pdfsuppresswarningpagegroup=1
% define list of equations
\newcommand{\listequationsname}{\Large{List of Equations}}
\newlistof{myequations}{equ}{\listequationsname}
\newcommand{\myequations}[1]{
	\addcontentsline{equ}{myequations}{\protect\numberline{\theequation}#1}
}
\setlength{\cftmyequationsnumwidth}{2.3em}
\setlength{\cftmyequationsindent}{1.5em}
% command to box, label, refference, and include
% noteworthy equations in list of equations
\newcommand{\noteworthy}[2]{
\begin{align} \label{#2} \ensuremath{\boxed{#1}} \end{align}
\myequations{#2} \centering \small \textit{#2} \normalsize \justify }
%%%%%%%%%%%%%%%%%%%%%%%%%%%%%%%%%%%%%%%%%%%%%%%%%%%%%%%%%%%%%%%%%%%%%%%%%%%%%%%%%%%%
% Theorems
%%%%%%%%%%%%%%%%%%%%%%%%%%%%%%%%%%%%%%%%%%%%%%%%%%%%%%%%%%%%%%%%%%%%%%%%%%%%%%%%%%%%
\newtheorem{definition}{Definition}
\newtheorem{theorem}{Theorem}
\newtheorem{lemma}{Lemma}
\newtheorem{corollary}{Corollary}


%%%\begin{document}

\chapter{Currents, Resistance, and Electromotive Force}%
\label{cha:currents_resistance_and_electromotive_force}


\LARGE\textsc{Date: 2020-09-21} \\ \\ \LARGE\textsc{Announcements:} \\
\small



\paragraph \hrule \paragraph \\ \fancyhead[R]{Lesson 6} \fancyhead[L]{Week 6}
%  %  %  %  %  %  %  %  %  %  %  %  %  %  %  %  %  %  %  %  %  %  %  %  %  %  %  %

\section{Current}%
\label{sec:current}

\begin{definition}[Current]
	A current is any motion of charge form one region to another.
\smallskip\hfill$\bullet$\end{definition}
\subsubsection{Quantifying Current}%
\label{ssub:quantifying_current}


\begin{itemize}
	\item Model a wire as a cylinder:
		\item $n = \frac{\:\text{\# of moving charges}\:}{\:\text{volume}\:}$
			\item $\:\text{Volume}\: = dx A$
			\item $\:\text{Volume}\: = (\overline{v}_{d}dt)A$
				\item Amount of charge that folows through cylinder:
					\[dQ = q \cdot \:\text{number of charges}\: = q \cdot n \cdot V\]
					\[dQ = q n \overline{v}_{d}dt A\]
					\[\frac{dQ}{dt} = I = q n \overline{v}_{d} A\]
\end{itemize}

\noteworthy{
I =  n |q| v_{d} A
}{Current}

\begin{description}
	\item[I] Current
	\item[q] charge per particle
	\item[n] \# of moving Charges
	\item[$v_{d}$] Drift velocity
	\item[A] Cross sectional area of conductor
	\item[units] Ampers, Amps, A
	\item Scalar quantity, cw or ccw in a curcuit
\end{description}

\noteworthy{
	\overline{J} = \frac{I}{A} = \left| q \right| n \overline{v}_d
}{Vector Current Density}
\begin{description}
	\item[I] Current
	\item[A] Cross sectional area
	\item[units] $\frac{A}{m^2}$
\end{description}

\subsection{Direction of Current Flow}%
\label{sub:direction_of_current_flow}

\begin{itemize}
	\item \textbf{Conventional Current} is treated as a flow of positive charges.
	\item I a metallic conductor, the charges moving are electrons, but the
		conventional current points in the opposite direction.
	\item The vector \textbf{Current Density} is always in the same direction
		as the electric field.
\end{itemize}

\section{Resistivity}%
\label{sec:resistivity}

\begin{definition}[Resistivity]
	The resistivity of a material is the ratio of the electric field in the material to the
	current density it causes. A measure of a materia's opposition to flow. Depends on
	material and temperature. \textbf{Conductivity} is the recipricol of resistivity
\smallskip\hfill$\bullet$\end{definition}

\noteworthy{
\rho = \frac{E}{J}
}{Resistivity}

\begin{description}
	\item[rho] Resistivity
	\item[E] Electric field
		\item[J] Current densityCurrent density
\end{description}

\noteworthy{
\rho (T)v = \rho_0 [1 + \alpha(T-T_0)]
}{Temperature dependence of resistivity}

\section{Resistance}%
\label{sec:resistance}

\subsection{Resistance and Ohm's Law}%
\label{sub:resistance_and_ohm_s_law}

The \textbf{resistance} of a conductor is $R = \rho \frac{L}{A}$.

The potential across a conductor is given by Ohm's Law:

\noteworthy{
V = IR
}{Ohm's Law}

\noteworthy{
R = \frac{\rho L}{A}
}{Resistance Equation}
\begin{description}
	\item[L] Length of conductor
		\item[A] Cross sectional  area
\end{description}

\section{Electromotive Froce and Circuits}%
\label{sec:electromotive_froce_and_circuits}

Electromotive force (emf) makes current flow from low to high potential. A circuit that
provides emf is called a \textbf{source of emf}. SI units of $1 V = 1 \frac{J}{C}$.
A flashlight battery has an emf of 1.5V; it does 1.5 J of work on every coulomb of charge that
passes through it.
Use symbol $\mathcal{E}$ (cursive E) for emf.


When a battery is connected, electrons must reach the positive charge. When they do:
\[W_{net} = W_{b} - W_{c} = \Delta KE = 0 (\:\text{steady state}\:)\]
\[W_{b} = W_{e}\]
\[q \mathcal{E} = q V_{ab}\]
\[\mathcal{E} = V_{ab}\]

$U_{i} \implies$ powering circuit elements. Rise in PE = loss in potential energy.

\[\mathcal{E} = V_{ab} = IR\]

This is energy balance, what ever emf is put in is used by the resistors.

\subsection{Internal Resistance}%
\label{sub:internal_resistance}

Real sources contain some \textbf{internal resistance, r}.

\noteworthy{
V_{ab} = \mathcal{E} - Ir
}{Potential of source with internal resistance}

\begin{description}
	\item[V] Terminal voltage, source with internal resistance.
	\item[$\mathcal{E}$] emf of source.
	\item[I] Current through source.
	\item[r] Internal resistance of source.
	\item[-Ir] Loss of potential in battery.
\end{description}

\section{Energy and Power in Electric Circuits}%
\label{sec:energy_and_power_in_electric_circuits}

\begin{itemize}
	\item A circuit element causes a net transfer of energy into or out of the circuit.
		\item the time rate of energy transfer is power, denoted by P
\end{itemize}

\noteworthy{
P = V_{ab} I = I^2 R = \frac{V_{ab}^2}{R}
}{Power delivered to or extracted from a circuit element}

\begin{description}
	\item[V] Voltage across circuit element.
		\item[I] Current in circuit element.
			\item[Unit] Watts, W
\end{description}


%  %  %  %  %  %  %  %  %  %  %  %  %  %  %  %  %  %  %  %  %  %  %  %  %  %  %  %
\newpage
%%%\end{document}


%LEAVE EMPTY ROW ABOVE THIS ONE
