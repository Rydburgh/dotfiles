%%%%%%%%%%%%%%%%%%%%%%%%%%%%%%%%%%%%%%%%%
% University/School Laboratory Report
% LaTeX Template
% Version 3.1 (25/3/14)
%
% This template has been downloaded from:
% http://www.LaTeXTemplates.com
%
% Original author:
% Linux and Unix Users Group at Virginia Tech Wiki
% (https://vtluug.org/wiki/Example_LaTeX_chem_lab_report)
%
% License:
% CC BY-NC-SA 3.0 (http://creativecommons.org/licenses/by-nc-sa/3.0/)
%
%%%%%%%%%%%%%%%%%%%%%%%%%%%%%%%%%%%%%%%%%

%----------------------------------------------------------------------------------------
%	PACKAGES AND DOCUMENT CONFIGURATIONS
%----------------------------------------------------------------------------------------

\documentclass{article}

\usepackage{float} % Required for images to be placed inline
\usepackage[version=3]{mhchem} % Package for chemical equation typesetting
\usepackage{siunitx} % Provides the \SI{}{} and \si{} command for typesetting SI units
\usepackage{graphicx} % Required for the inclusion of images
\usepackage{natbib} % Required to change bibliography style to APA
\usepackage{amsmath} % Required for some math elements

\setlength\parindent{0pt} % Removes all indentation from paragraphs

\renewcommand{\labelenumi}{\alph{enumi}.} % Make numbering in the enumerate environment by letter rather than number (e.g. section 6)

%\usepackage{times} % Uncomment to use the Times New Roman font

%----------------------------------------------------------------------------------------
%	DOCUMENT INFORMATION
%----------------------------------------------------------------------------------------

\title{Coulomb's Law Lab\\ PHY-222-AC01} % Title

\author{Jasper \textsc{Runco}} % Author name

\date{\today} % Date for the report

\begin{document}

\maketitle % Insert the title, author and date

\begin{center}
	\begin{tabular}{l r}
	\end{tabular}
\end{center}

% If you wish to include an abstract, uncomment the lines below
% \begin{abstract}
% Abstract text
% \end{abstract}

%----------------------------------------------------------------------------------------
%	SECTION 1
%----------------------------------------------------------------------------------------

\section{Theory}

Coulomb's law describes a mathematical expression for the interaction of electrically charged objects.
The electrical charge generates a force pair, $F_{E}$, with a magnitude that has been experimentally determined
to such a degree that it is accepted as a universal law.

\subsection{Definitions}
\label{definitions}
\begin{description}
	\item[Electrically charged object -]
		Matter with more or less electrons than protons is negatively or
		positively charged respectively.
	\item[Force pair -]
		The principal of Newton's third law in which every action force has a corresponding reaction force
		equal in magnitude and opposite in direction.
	\item[Coulomb's law -]
		\[F_{E} = k \frac{q_1 q_2}{r^2}\]
		\begin{description}
			\item[r -]
				The distance between the charged objects measured in meters.
			\item[$q_1 q_2$ -]
				The sum of the electrical charges of each object, measured in Coulombs.
			\item[k -]
				The electric constant, the experimentally determined proportionality constant with a value
				of $k = 9.0 \times 10^{9} \frac{Nm^2}{C^2}$

		\end{description}
\end{description}

%----------------------------------------------------------------------------------------
%	SECTION 2
%----------------------------------------------------------------------------------------
\section{Objectives}


% If you have more than one objective, uncomment the below:
\begin{description}
	\item[First Objective] \hfill \\
		Experimentally confirm Coulomb's law.
	\item[Second Objective] \hfill \\
		Study how distance and charge affect the electric force.
	\item[Third Objective] \hfill \\
		Experimentally determine the value of the electric constant, k.
\end{description}


%----------------------------------------------------------------------------------------
%	SECTION 3
%----------------------------------------------------------------------------------------

\section{Experimental Data}

\subsection{Part One}%
\label{sub:part_one}


\begin{table}[htpb]
	\centering
	\caption{}
	\label{tab:label}
	\begin{tabular}{| c |  c | c |  c | }
		\hline
		$q_1= 2 \mu C$ & &  $q_2 = 4 \mu C$  &\\
		\hline
		\textbf{r(cm)} & \textbf{$r^2(m^2)$} & $\frac{1}{r^2}(\frac{1}{m^2})$ & $F_{E}(N)$ \\
		\hline
		10 & $1.0 \times 10^{-2}$ & $1\times 10^{2}$& $7.190$ \\
		\hline
		9 & $8.1 \times  10^{-3}$& $1.2 \times 10^{2}$ & $8.877$\\
		\hline
		8 & $6.4 \times 10^{-3}$& $1.6 \times 10^{2}$ & $11.234$\\
		\hline
		7 & $4.9 \times  10^{-3}$& $2.0 \times  10^{2}$ & $14.674$ \\
		\hline
		6 & $3.6 \times 10^{-3}$& $2.8 \times 10^{2}$& $19.972$\\
		\hline
		5 & $2.5 \times 10^{-3}$ & $4.0 \times  10^2$& $28.760$ \\
		\hline
		4 & $1.6 \times  10^{-3}$& $6.3 \times  10^2$ & $44.938$ \\
		\hline
		3 & $9.0 \times  10^{-4}$& $1.1 \times  10^{3}$ & $ 79.889$\\
		\hline
	\end{tabular}
\end{table}

\subsection{Part Two}%
\label{sub:part_two}

\begin{table}[htpb]
	\centering
	\caption{}
	\label{tab:label}
	\begin{tabular}{| c | c |}
		\hline
		$q_1 = 5 \mu C$ & $ r = 6 cm$   \\
		\hline
		$q_2 (\mu C)$ & $F_{E}(N)$\\
		\hline
		10 & $124.827$  \\
		\hline
		9 &  $112.344$ \\
		\hline
		8 &  $99.862$ \\
		\hline
		7 &  $87.379$ \\
		\hline
		6 &   $74.896$\\
		\hline
		5 &  $62.414$ \\
		\hline
		4 &   $49.931$\\
		\hline
		3 &  $37.448$ \\
		\hline
	\end{tabular}
\end{table}


%----------------------------------------------------------------------------------------
%	SECTION 4
%----------------------------------------------------------------------------------------

\section{Data Analysis}

\subsection{Part One}%
\label{sub:part_one}


\subsubsection{$F_{E}$ with respect to $r$}%
\label{ssub:_f__e_with_respect_to_r_}

\begin{figure}[H]
	\begin{center}
		\includegraphics[width=0.65\textwidth]{plot1} % Include the image placeholder.png
		\caption{Force per distance}
	\end{center}
\end{figure}

\begin{description}
	\item[Comments on Figure 1:] As the charged objects' distance $(r)$ tends towards zero
		from the positive direction, the force $(F)$ tends towards infinity. As $r$ tends
		towards positive infinity, $F$ tends towards zero.
\end{description}

\subsubsection{$F_{E}$ with respect to $\frac{1}{r^2}$ to find $k$}%

\begin{figure}[H]
	\begin{center}
		\includegraphics[width=0.65\textwidth]{plot2} % Include the image placeholder.png
		\caption{Force per inverse square of distance}
	\end{center}
\end{figure}


\begin{description}
	\item[Coulomb's Law -]
		$F_{E} = k \frac{q_1 q_2}{r^2}$
	\item[Linear regression of Figure 2 data -] y = 0.072238x
	\item[$q_1$ -] $2 \mu C$
	\item[$q_2$ -] $4 \mu C$
\end{description}

\begin{align*}
	F_{E} &= k \frac{(2 \mu C)(4 \mu C)}{r^2} \implies \\
	F_{E} (N) &= k \left( 8 \times 10^{-12} C^2 \right)  \left( \frac{1}{r^2} \right) \left( \frac{1}{m^2} \right)  \\
	y &= 0.072238x \implies \\
	F_{E} &= 0.072238 \left( \frac{1}{r^2} \right)  \\
	0.072238 \left( \frac{1}{r^2} \right) (N) &=  k \left( 8 \times 10^{-12} C^2 \right) \left( \frac{1}{r^2} \right) \left( \frac{1}{m^2} \right)\\
	k &= \frac{  0.072238 \left( \frac{1}{r^2} \right) (N)(m^2)}{8 \times 10^{-12} C^2 \left(  \frac{1}{r^2}\right)  } \\
	k &= \boxed{9.02975 \times 10^{9}}\left( \frac{N m^2}{C^2} \right)  \\
\end{align*}


\subsubsection{Percent error in $k$}%
\label{ssub:percentag}

\begin{align*}
	\:\text{\% error}\: &= \left| \frac{\:\text{theoretical}\: - \:\text{experimental}\:}{\:\text{theoretical}\:}  \right| \times 100\%\\
\:\text{\% error}\: &= \left|   \frac{\left( 9.0 \times 10^{9} \frac{N m^2}{C^2} \right) - \left( 9.02975 \times 10^{9}\frac{N m^2}{C^2}  \right) }{\left( 9.0 \times 10^{9} \frac{N m^2}{C^2} \right)} \right| \times 100\%\\
\:\text{\% error}\: &= 0.330556 \%
\end{align*}



\subsection{Part Two}%
\label{sub:part_two}



\subsubsection{$F_{E}$ with respect to $q_{2}$}%
\label{ssub:_f__e_with_respect_to_q__2_}

\begin{figure}[H]
	\begin{center}
		\includegraphics[width=0.65\textwidth]{plot3} % Include the image placeholder.png
		\caption{Force per magnitude charge on $q_{2}$}
	\end{center}
\end{figure}



\begin{description}
	\item[Comments on Figure 3: ]
		There is a positive linear relationship between the magnitude of charge on one object $q_2 (\mu C)$ and
		electric force $F_{E} (N)$. Given the precision of $F_{E}$ measurement, it appears to approach a
		lower bounds of $F_{E} = 0$ at $q_2 = 0$.
\end{description}



\subsubsection{$F_{E}$ with respect to $q_2$ to find $k$}%
\label{ssub:_f__e_with_respect_to_q_2_to_find_k_}


\begin{description}
	\item[Coulomb's Law -]
		$F_{E} = k \frac{q_1 q_2}{r^2}$
	\item[Linear regression of Figure 3 data -] $y = 12.483 x + 0.00021429$
	\item[$q_1$ -] $5 \mu C$
		\item[r -] $6 cm$
\end{description}

\begin{align*}
	F_{E} &= k \frac{(5 \mu C)(q_2)}{(6cm)^2} \implies \\
	F_{E}(N) &=  k q_2(\mu C) \left( \frac{5\times 10^{-6}C}{3.6 \times 10^{-3}m^2} \right)  \\
	y &= 12.483 x + 0.00021429 \implies \\
	F_{E} &= 12.483 q_2 + 0.00021429 \\
	(12.483 q_2 + 0.00021429) (N) &=  k q_2(\mu C) \left( \frac{5\times 10^{-6}C}{3.6 \times 10^{-3}m^2} \right) \implies \\
	k &= \frac{ (12.483 q_2 + 0.00021429)(3.6\times 10^{-3} m^2)(N)}{(q_2)(5\times 10^{-6}C)(10^{-6}C)} \implies  \\
	k &=  \boxed{ 8.98776 \times 10^{9} \left( \frac{Nm^2}{C^2} \right)}  + 1.542888 \times 10^{5}q_2^{-1}\left( \frac{Nm^2}{C^2} \right) \\
\end{align*}


\subsubsection{Percent error in $k$}%
\label{ssub:percentag}

\begin{align*}
	\:\text{\% error}\: &= \left| \frac{\:\text{theoretical}\: - \:\text{experimental}\:}{\:\text{theoretical}\:}  \right| \times 100\%\\
\:\text{\% error}\: &= \left|   \frac{\left( 9.0 \times 10^{9} \frac{N m^2}{C^2} \right) - \left( 8.98776 \times 10^{9}\frac{N m^2}{C^2}  \right) }{\left( 9.0 \times 10^{9} \frac{N m^2}{C^2} \right)} \right| \times 100\%\\
\:\text{\% error}\: &= 0.136 \%
\end{align*}


%----------------------------------------------------------------------------------------
%	SECTION 5
%----------------------------------------------------------------------------------------

\section{Results and Conclusions}

\subsection{First Objective}%
\label{sub:first_objective}

This experiment does confirm Coulomb's law. It is observed that charged objects generate a non-contact force
on each other that is proportional to the sum of their charges and inversely proportional to
the square of their distance.

\subsection{Second Objective}%
\label{sub:second_objective}

According to Figure 1, the magnitude of the force pair experienced by the charged objects increases
as the distance decreases. This is an exponential relationship and the exponent that describes how the
force changes depending on the distance variable is 2, meaning this is an inverse square relationship.

According to Figure 3, the magnitude of the force pair experienced by the charged objects increases as the
charge on either object increases. The variable of charge has a directly proportional effect on the
force experienced by the objects. This is the case when the magnitude of the other charge is held constant,
but the total effect on the force is determined by the sum of their charges.

\subsection{Third Objective}%
\label{sub:third_objective}

Figures 2 and 3 show that that the value of $k = 9.0\times 10^{9}$ $\frac{Nm^2}{C^2}$ was experimentally
confirmed with a maximum percent error of $0.330556\%$

%----------------------------------------------------------------------------------------
%	SECTION 6
%----------------------------------------------------------------------------------------

\section{Discussion of Experimental Uncertainty}

The virtual experiment is meant to show an uncertainty generated by the precision of the force $(F_{E}(N))$
measurement. Tables 1 and 2 show a precision of $F_{E}$ on the $mN$ scale. This is a rounded value that makes
calculation practical but leads to an error in the calculation of $k$ and fitting error in Figure 3 where
the line does not intersect the origin.


\end{document}
