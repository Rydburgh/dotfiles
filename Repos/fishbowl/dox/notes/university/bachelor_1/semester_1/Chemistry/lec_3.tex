%%%\documentclass[a4paper,11pt,twoside]{report}

%%%% ____  ____  _____    _    __  __ ____  _     _____
%|  _ \|  _ \| ____|  / \  |  \/  | __ )| |   | ____|
%| |_) | |_) |  _|   / _ \ | |\/| |  _ \| |   |  _|
%|  __/|  _ <| |___ / ___ \| |  | | |_) | |___| |___
%|_|   |_| \_\_____/_/   \_\_|  |_|____/|_____|_____|
%
% last updated: 2020-05-21
%%%%%%%%%%%%%%%%%%%%%%%%%%%%%%%%%%%%%%%%%%%%%%%%%%%%%%%%%%%%%%%%%%%%%%%%%%%%%%%%%%%%%%%%%%
%%% PACKAGES
%%%%%%%%%%%%%%%%%%%%%%%%%%%%%%%%%%%%%%%%%%%%%%%%%%%%%%%%%%%%%%%%%%%%%%%%%%%%%%%%%%%%%%%%%%


\pdfminorversion=7						% to prevent errors when building pdf

% some basic packages
\usepackage{amsmath, amsthm, amssymb}
\usepackage{mhchem}						% for chemical symbols
\usepackage{url}						% to format hyperlink text
\usepackage{float}						% for custom figure/table environment
\usepackage{xifthen}					% to handle tests
\usepackage{booktabs}					% for table commands and optimisation
\usepackage{enumitem}					% to format enumerate, itemize, and description
\usepackage{textcomp}					% to support different glyphs
\usepackage{graphicx}					% to support \includegraphics
\usepackage[T1]{fontenc}				% for unicode encoding
\usepackage[utf8]{inputenc}				% for unicode input
\setlength{\headheight}{13.6pt}
\usepackage[top=1.5in,bottom=1in,right=1in,left=1in,headheight=45pt]{geometry}
\usepackage{fancyhdr}					% for adding different headers
\pagestyle{fancy}
% for list of equations
\usepackage{tocloft}					% for custom lists
\usepackage{ragged2e} 					% to undo \centering
\usepackage{hyperref} 					% to make references hyperlinks
\usepackage{glossaries}

% for figures
\usepackage{import}						% for file control
\usepackage{pdfpages}					% for pdf, graphics, and hypertext
\usepackage{transparent}				% for color stack transparency
\usepackage{xcolor}						% for arbitrary color mixing



\author{Jasper Runco}
\date{2020 // Fall}

%%%%%%%%%%%%%%%%%%%%%%%%%%%%%%%%%%%%%%%%%%%%%%%%%%%%%%%%%%%%%%%%%%%%%%%%%%%%%%%%%%%%
% Commands
%%%%%%%%%%%%%%%%%%%%%%%%%%%%%%%%%%%%%%%%%%%%%%%%%%%%%%%%%%%%%%%%%%%%%%%%%%%%%%%%%%%%

% to make a new figure
\newcommand{\incfig}[2][scale=1]{%
	% \def\svgwidth{#1\columnwidth}
	\import{./figures/}{#2.pdf_tex}
}
\pdfsuppresswarningpagegroup=1

% define list of equations
\newcommand{\listequationsname}{\Large{List of Equations}}
\newlistof{myequations}{equ}{\listequationsname}
\newcommand{\myequations}[1]{
	\phantomsection
	\addcontentsline{equ}{myequations}{\protect\numberline{\theequation}#1}
}

\setlength{\cftmyequationsnumwidth}{2.3em}
\setlength{\cftmyequationsindent}{1.5em}

% command to box, label, reference, and include
% noteworthy equations in list of equations
\newcommand{\noteworthy}[2]{
\begin{align} \label{#2} \ensuremath{\boxed{#1}} \end{align}
\myequations{#2} \centering \textit{#2} \justify}

\newtheorem{definition}{Definition}
\newtheorem{theorem}{Theorem}
\newtheorem{lemma}{Lemma}
\newtheorem{corollary}{Corollary}
\newtheorem{example}{Example}
\newtheorem{solution}{Solution}
\newtheorem{constant}{Constant}
\newtheorem{note}{Note}


%%%\begin{document}

\chapter{Wave-Particle Duality of Light}%
\label{cha:wave_particle_duality_of_light}


\LARGE\textsc{Date: 2020-06-05} \\ \LARGE\textsc{Announcements:}



\paragraph \hrule \paragraph \\ \fancyhead[R]{Lesson 3} \fancyhead[L]{Week 1}
%  %  %  %  %  %  %  %  %  %  %  %  %  %  %  %  %  %  %  %  %  %  %  %  %  %  %  %

\section{Properties of Waves}%
\label{sec:properties_of_waves}
Waves have a periodic variation of some quntity.

\begin{definition}[Amplitude]
	The deviation from the average level
\end{definition}

\begin{definition}[Wavelength]
	($\lambda$)The distance between succesive maxima
\end{definition}

\begin{definition}[Frequency]
	($\nu$) Number of cycles per unit time. Units of frequency
	are \# of cycles / second = $Hz$ = $s^{-1}$.
\end{definition}

\begin{definition}[Period]
	$\frac{1}{\nu}$ The time it takes for one cycle to occure
\end{definition}

\begin{definition}[Intensity]
	$= a^2$
\end{definition}

\begin{definition}[Speed]
	$\lambda \nu$ (m/sec)
\end{definition}
\begin{constant}[Speed of light]
	$c = \lambda \nu = 2.9979 \times 10^{8}$ m/s
	$c \approx 670,000,000 $ $mph \approx 186,000 $ miles/s
\end{constant}

Note: ROY G. BIV

\subsection{Superposition}%
\label{sub:superposition}

Waves can interfere constructively when they are superpositioned in phase.
Distructive when out of phase.

\subsection{Light as a particle}%
\label{sub:light_as_a_particle}

\subsubsection{Photoelectric effect}%
\label{ssub:photoelectric_effect}

A beam of UV light hitting a metal surface can eject electrons if the frequency
$\nu $ is greater or equal to a threshold frequency $\nu _{0}$

They changed the frequency with a constant intensity: below threshold frequency no
electrons, at threshold there were a certain number of electrons, but above the
threshold there was no change in number of ejections.

They looked at the KE of the light as a function of the frequency, and found they
increased proportionally.

They then looked at how the intensity increased the KE, but it stayed constant.

Looking at the number of electrons as a function of intensity, they were
directly proportional.

These results were all unexpected, untill Einstein looked into the data. Different metals
had different threshold frequencies, but they all had the same slope.  $ m = 6.626 \times 10^{-34}$ Js

Planck came up with the number when studying black body radiation.

\begin{constant}[Planck's constant]
	$h = 6.626 \times 10^{-34}$ Js
\end{constant}

Rewriting the equation of the line, Einstein came up with the frequency being proportional
to the energy

\noteworthy{
K.E. = h \nu - h \nu _{0}
}{Photoelectric effect}

\begin{description}
	\item[$\nu$ -] frequency of incident light
	\item[$h \nu $ -] the \textbf{energy of the incident light $= E_{i}$}
	\item[$\nu _{0}$ -] threshold frequency
	\item[$h\nu_{0}$ -] threshold energy or \textbf{workfunction ($\phi$)}
\end{description}

Ultimately, he realized energy of light is proportional to frequency.

\noteworthy{
E = h \nu
}{Photon Energy - Einstein (1905)}
\begin{note}[Units:]
	$J = (Js)(s^{-1})$
\end{note}

Light is made up of energy "packets" called photons, where the energy
of the photon depends on its frequency.


\begin{definition}[Intensity]
	number of photons per second (Units: Watt, J/s)
\end{definition}





%  %  %  %  %  %  %  %  %  %  %  %  %  %  %  %  %  %  %  %  %  %  %  %  %  %  %  %
\newpage
%%%\end{document}


%LEAVE EMPTY ROW ABOVE THIS ONE
