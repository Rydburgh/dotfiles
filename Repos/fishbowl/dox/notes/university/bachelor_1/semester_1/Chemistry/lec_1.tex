%%%\documentclass[a4paper,11pt,twoside]{report}

%%%% ____  ____  _____    _    __  __ ____  _     _____
%|  _ \|  _ \| ____|  / \  |  \/  | __ )| |   | ____|
%| |_) | |_) |  _|   / _ \ | |\/| |  _ \| |   |  _|
%|  __/|  _ <| |___ / ___ \| |  | | |_) | |___| |___
%|_|   |_| \_\_____/_/   \_\_|  |_|____/|_____|_____|
%
% last updated: 2020-05-21
%%%%%%%%%%%%%%%%%%%%%%%%%%%%%%%%%%%%%%%%%%%%%%%%%%%%%%%%%%%%%%%%%%%%%%%%%%%%%%%%%%%%%%%%%%
%%% PACKAGES
%%%%%%%%%%%%%%%%%%%%%%%%%%%%%%%%%%%%%%%%%%%%%%%%%%%%%%%%%%%%%%%%%%%%%%%%%%%%%%%%%%%%%%%%%%


\pdfminorversion=7						% to prevent errors when building pdf

% some basic packages
\usepackage{amsmath, amsthm, amssymb}
\usepackage{mhchem}						% for chemical symbols
\usepackage{url}						% to format hyperlink text
\usepackage{float}						% for custom figure/table environment
\usepackage{xifthen}					% to handle tests
\usepackage{booktabs}					% for table commands and optimisation
\usepackage{enumitem}					% to format enumerate, itemize, and description
\usepackage{textcomp}					% to support different glyphs
\usepackage{graphicx}					% to support \includegraphics
\usepackage[T1]{fontenc}				% for unicode encoding
\usepackage[utf8]{inputenc}				% for unicode input
\setlength{\headheight}{13.6pt}
\usepackage[top=1.5in,bottom=1in,right=1in,left=1in,headheight=45pt]{geometry}
\usepackage{fancyhdr}					% for adding different headers
\pagestyle{fancy}
% for list of equations
\usepackage{tocloft}					% for custom lists
\usepackage{ragged2e} 					% to undo \centering
\usepackage{hyperref} 					% to make references hyperlinks
\usepackage{glossaries}

% for figures
\usepackage{import}						% for file control
\usepackage{pdfpages}					% for pdf, graphics, and hypertext
\usepackage{transparent}				% for color stack transparency
\usepackage{xcolor}						% for arbitrary color mixing



\author{Jasper Runco}
\date{2020 // Fall}

%%%%%%%%%%%%%%%%%%%%%%%%%%%%%%%%%%%%%%%%%%%%%%%%%%%%%%%%%%%%%%%%%%%%%%%%%%%%%%%%%%%%
% Commands
%%%%%%%%%%%%%%%%%%%%%%%%%%%%%%%%%%%%%%%%%%%%%%%%%%%%%%%%%%%%%%%%%%%%%%%%%%%%%%%%%%%%

% to make a new figure
\newcommand{\incfig}[2][scale=1]{%
	% \def\svgwidth{#1\columnwidth}
	\import{./figures/}{#2.pdf_tex}
}
\pdfsuppresswarningpagegroup=1

% define list of equations
\newcommand{\listequationsname}{\Large{List of Equations}}
\newlistof{myequations}{equ}{\listequationsname}
\newcommand{\myequations}[1]{
	\phantomsection
	\addcontentsline{equ}{myequations}{\protect\numberline{\theequation}#1}
}

\setlength{\cftmyequationsnumwidth}{2.3em}
\setlength{\cftmyequationsindent}{1.5em}

% command to box, label, reference, and include
% noteworthy equations in list of equations
\newcommand{\noteworthy}[2]{
\begin{align} \label{#2} \ensuremath{\boxed{#1}} \end{align}
\myequations{#2} \centering \textit{#2} \justify}

\newtheorem{definition}{Definition}
\newtheorem{theorem}{Theorem}
\newtheorem{lemma}{Lemma}
\newtheorem{corollary}{Corollary}
\newtheorem{example}{Example}
\newtheorem{solution}{Solution}
\newtheorem{constant}{Constant}
\newtheorem{note}{Note}


%%%\begin{document}

\chapter{Average atomic mass}%
\label{cha:average_atomic_mass}


\LARGE\textsc{Date: 2020-05-19} \\ \LARGE\textsc{Anouncements:}

make inkscape always open in workspace 3 \checkmark

get inkscape shortcut manager configured

\paragraph \hrule \paragraph \\ \fancyhead[R]{Lesson 2} \fancyhead[L]{Week 1}
%  %  %  %  %  %  %  %  %  %  %  %  %  %  %  %  %  %  %  %  %  %  %  %  %  %  %  %

\section{Introduction}%
\label{sec:introduction}


The periodic table we take for granted is the result of thousands of years of
work trying to sort out the complexity of the physical world. How do you make
sense of it?

At the end of the day, we realized that the building blocks, elements, that
make up matter are of a limited number, and they are disproportionately present.
They have different properties and are themselves made up of other building blocks.

With chemistry, we can start to make use of the math and physics to understand the
world. On top of that, we build biology, based on molecular interactions.

\section*{Elements and atoms}%
\label{sec:Elements and atoms}
Observe that different substances have different properties and they react with
each other differently. Certain types of air particles\ldots it leads to a question
. If we break down carbon to the smallest possible size that it would still have
the properties of carbon. We call this an atom, the most basic unit of any
element. They are unimaginably small. One hair is one million carbon atoms wide.

\subsubsection*{The elements are related to each other}%
\label{sec:The elements are related to eachother}
A carbon atom is made up of the same subatomic particles as a gold atom.


\subsubsection*{Proton}%
\label{sec:Proton}

The elements are written in order of atomic number, which is the number of
protons in it's nucleus. Protons define the element.

\subsubsection*{Neutrons}%
\label{sec:section name{Neutrons}}
Neutrons are in the nucleus and their number can change without changing the
element. Example: \ce{^{12}_{6}C} and \ce{^{14}_{6}C}.
\subsubsection{Electrons}%
\label{ssub:electrons}

These buzz around the nucleus and determine the net charge of the atom.



To operate at such a small scale, we need to have some units of measurement.
How to we define mass at the atomic scale. Historically, scientists have used the
\textbf{unified atomic mass unit (amu)} and is defined as \[
	u = 1.660540 \times 10^{-27} \text{kg}
.\]

This seems like a strange number. In fact, this deffinition makes it cleaner when
talking about the mass of atoms. The mass of a proton is approximately $1u$ and
a neutron is approximately $1u$

The electron mass is almost $\frac{1}{2000}$ the mass of the proton and neutron.
\section{Periodic table}%
\label{sec:periodic_table}

The number on top is the elements atomic number, i.e. the number of protons in the
nucleus. But what would the mass of hydrogen be? There are different \textbf{isotopes}
of hydrogen, the most common form has 1 proton, 0 neutrons, and 1 electron. It's mass
is $\approx 1.008 u$. But the periodic table takes the weighted average of all isotopes
to give the \textbf{average atomic mass}. Because they don't put units in, it's really the
relative atomic mass.

\section{Worked example: Atomic weight calculation}%
\label{sec:worked_example_atomic_weight_calculation}

If we look at this table, we see carbon-12 is the most common isotope on earth.
\begin{table}[htpb]
	\centering
	\caption{Atomic mass}
	\label{tab:Atomic mass}
	\begin{tabular}{c l r}
	$\ce{^{12}_{}C}$ $98.89\%$ & $12.0000$ amu \\
	$\ce{^{13}_{}C}$ $1.110\%$ & $13.0034$ amu
	\end{tabular}
\end{table}

The periodic table reports the atomic weight of carbon as 12.01. It is callculated
by taking the weighted average of these two isotopes: \[
0.9889 \cdot 12 + 0.0111 \cdot 13.0034 = 12.0111374 \approx 12.01
.\]
The difference in atomic mass is from the number of neutrons the isotope has.

\section{The mole and Avogadro's number}%
\label{sec:the_mole_and_avogadro_s_number}

Here we connect the avg. atomic mass to the masses we will likely handle in a lab.
The chemistry community has come up with the following tool, take Lithium for example:
it has $6.94 \frac{u}{atom Li}$. How many atoms would a sample have it it had the mass
$ 6.94 g$Li. This number, named Avogadro's number, is
\noteworthy{
\approx 6.022 \times 10^{23}
}{Constant: Avogadro's number}

One \textbf{mole} of an element is this many atoms of the element. In chemistry practice
finding the number of moles is a useful thing, and finding the number of atoms is
simply multiplying the number of moles by Avogadro's number. So to go from mass to
moles, devide by molar mass. Then multiply by Avogadr's number if you want to know the
number of atoms.

\section{Atomic number, mass number, and isotopes}%
\label{sec:atomic_number_mass_number_and_isotopes}

\begin{definition}[atomic number (Z)]
	The number of protons in a nucleus, the number of
	electrons (in a neutral atom)
\end{definition}
\begin{definition}[isotope]
	atoms of a single element (same atomic number) that differ
	in the number of neutrons in their nuclei (different masses)
\end{definition}








%  %  %  %  %  %  %  %  %  %  %  %  %  %  %  %  %  %  %  %  %  %  %  %  %  %  %  %
\newpage
%%%\end{document}


%LEAVE EMPTY ROW ABOVE THIS ONE
